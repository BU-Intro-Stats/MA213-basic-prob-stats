%%%%%%%%%%%%%%%%%%%%%%%%%%%%%%%%%%%%
% Lesson Plan (50 minutes)
%%%%%%%%%%%%%%%%%%%%%%%%%%%%%%%%%%%%
\begin{frame}
    \frametitle{Lesson Plan}
    \begin{itemize}
        \item 10 min Lecture: 
        \item 10 min Think/pair/share: Based on these results, do you think dolphins can communicate abstract ideas?
        \item 3 min Slide setting up a Coin Flip Experiment: how could we convince someone else? -- wait for someone to suggest coin flip
        \item 5 min Applet: Coin flips
        \item 10 min Lecture: Case study slides from OpenIntro, Hypothesis Test
        \item 10 min R demo: Simulation-based hypothesis test
        \item 1 min Lecture: wrap up
    \end{itemize}
\end{frame}

%%%%%%%%%%%%%%%%%%%%%%%%%%%%%%%%%%%%
% Learning objectives:
%%%%%%%%%%%%%%%%%%%%%%%%%%%%%%%%%%%%
\begin{frame}
    \frametitle{Learning Objectives}
    \begin{itemize}
        \item \textbf{M1, LO2: Evaluate Study Design and Its Implications:} Identify and explain experimental design choices (observational vs. experimental, sampling methods, blinding, potential biases), and judge whether results can be generalized to a population or used to infer causation. 
        \item \textbf{M1, LO3: Use R for Data Management and Exploration:} Utilize R to load, pre-process, and explore data through visualization and summarization techniques.
        \item \textbf{M1, LO4: Visualize and Describe Data Distributions:} Select appropriate visualizations (scatterplots, histograms, box plots, bar plots) to depict data, and describe distributions qualitatively (shape, center, spread, outliers) and quantitatively (mean, median, mode, range, IQR, standard deviation).
        \item \textbf{M1, L05: Conduct Hypothesis Testing Using Simulation:} Set up null and alternative hypotheses to test for independence between variables, and use simulation techniques to evaluate data support for these hypotheses.
    \end{itemize}
\end{frame}

%%%%%%%%%%%%%%%%%%%%%%%%%%%%%%%%%%%%

