%%%%%%%%%%%%%%%%%%%%%%%%%%%%%%%%%%%%
% Lesson Plan (50 minutes)
%%%%%%%%%%%%%%%%%%%%%%%%%%%%%%%%%%%%
\begin{frame}
    \frametitle{Lesson Plan}
    \begin{itemize}
        \item xx min Lecture: Motivate using the t distribution, could use an example (like traffic flow/Friday 13 in the slides)
        \item xx min Edfinity quiz: like slides 2-3, check intuition about independence and the correct hypotheses
        \begin{itemize}
            \item Maybe add to the quiz an open (not trick, but tricky) question: do you think we have any tools yet for handling one-sample data? How would you try it? Then:
            \item xx min R Demonstration: let's try a method we've learned before (by popular vote?): how does it perform? what are the limitations? (like slides 4-6, in practice)
        \end{itemize}
        \item xx min Lecture/board work: in fact our sample size is too small, but we do have a new tool for this: the t distribution!
        \item xx min Lecture: t-distribution test statistic, p-value
        \item xx min R Demonstration: compute the test statistic and p-value for the HT in item 2 above, then interpret the results (like slides 10-12)
        \item xx min Edfinity quiz (challenge question): how would you determine what the difference is, given all the info (from R) that you need? (like slides 13-16)
        \item xx min Lecture: review the answer/the correct workflow, then interpret the result together (incl. something like slide 18)
    \end{itemize}
\end{frame}
            
%%%%%%%%%%%%%%%%%%%%%%%%%%%%%%%%%%%%
% Learning objectives:
%%%%%%%%%%%%%%%%%%%%%%%%%%%%%%%%%%%%
\begin{frame}
    \frametitle{Learning Objectives}
    \begin{itemize}
        \item \textbf{M1, LO3: Use R for Data Management and Exploration:} Utilize R to load, pre-process, and explore data through visualization and summarization techniques.
        \item \textbf{M3, LO1: Understand Point Estimates and Sampling Variability:} Define a sample statistic (point estimate) for a population parameter, and explain how it varies across different samples.
        \item \textbf{M3, LO3: Calculate and Interpret Standard Error:} Calculate the standard error for proportions and interpret it as a measure of sampling variability.
        \item \textbf{M3, LO4: Explain Hypothesis Testing and Its Limitations:} Discuss the use cases and potential issues with hypothesis testing, including the interpretation of results.
        \item \textbf{M3, LO6: Distinguish Statistical vs. Practical Significance:} Differentiate between statistical significance and practical significance, and explain the implications of each.
        \item \textbf{M4, LO2: Design and Interpret Confidence Intervals:} Design, execute, and interpret confidence intervals for the population proportion.
        \item \textbf{M4, LO5: Explain and Use the t-Distribution:} Explain how the t-distribution differs from the normal distribution and why it is used for population mean inference.
        \item \textbf{M4, LO6: Conduct and Interpret t-Tests:} Design, execute, and interpret t-tests for a single population mean, a difference of paired means, and a difference of independent means, calculating the standard error appropriately for each. Describe how to obtain a p-value for a t-test and a critical t-score for a confidence interval.
    \end{itemize}
\end{frame}
    
%%%%%%%%%%%%%%%%%%%%%%%%%%%%%%%%%%%%
% TODO: Adapt drafted slides