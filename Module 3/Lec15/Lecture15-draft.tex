%%%%%%%%%%%%%%%%%%%%%%%%%%%%%%%%%%%%
% Slide options
%%%%%%%%%%%%%%%%%%%%%%%%%%%%%%%%%%%%

% Option 1: Slides with solutions

\documentclass[t,compress,mathserif]{beamer}
\newcommand{\soln}[1]{\textit{#1}}
\newcommand{\solnGr}[1]{#1}
\def\chp5@path{.}

% Option 2: Handouts without solutions

%\documentclass[11pt,containsverbatim,handout]{beamer}
%\usepackage{pgfpages}
%\pgfpagesuselayout{4 on 1}[letterpaper,landscape,border shrink=5mm]
%\newcommand{\soln}[1]{ }
%\newcommand{\solnGr}{ }


%%%%%%%%%%%%%%%%%%%%%%%%%%%%%%%%%%%%
% Style
%%%%%%%%%%%%%%%%%%%%%%%%%%%%%%%%%%%%

\input{../../lec_style.tex}


%%%%%%%%%%%%%%%%%%%%%%%%%%%%%%%%%%%%
% Preamble
%%%%%%%%%%%%%%%%%%%%%%%%%%%%%%%%%%%%

\title[Chp 5: Foundations for inference]{Chapter 5: Foundations for inference}
\author{OpenIntro Statistics, 4th Edition}
\institute{$\:$ \\ {\footnotesize Slides developed by Mine \c{C}etinkaya-Rundel of OpenIntro. \\
The slides may be copied, edited, and/or shared via the \webLink{http://creativecommons.org/licenses/by-sa/3.0/us/}{CC BY-SA license.} \\
Some images may be included under fair use guidelines (educational purposes).}}
\date{}


%%%%%%%%%%%%%%%%%%%%%%%%%%%%%%%%%%%%
% Begin document
%%%%%%%%%%%%%%%%%%%%%%%%%%%%%%%%%%%%

\begin{document}


%%%%%%%%%%%%%%%%%%%%%%%%%%%%%%%%%%%%
% Title page
%%%%%%%%%%%%%%%%%%%%%%%%%%%%%%%%%%%%

{
\addtocounter{framenumber}{-1} 
{\removepagenumbers 
\usebackgroundtemplate{\includegraphics[width=\paperwidth]{../../OpenIntro_Grid_4_3-01.jpg}}
\begin{frame}

\hfill \includegraphics[width=20mm]{../../oiLogo_highres}

\titlepage

\end{frame}
}
}


%%%%%%%%%%%%%%%%%%%%%%%%%%%%%%%%%%%%
% Sections
%%%%%%%%%%%%%%%%%%%%%%%%%%%%%%%%%%%%

\section{Review}

\begin{frame}
\frametitle{Last time: Distributions and random variables (Ch. 4)}
\begin{itemize}
    \item Continuous vs discrete random variables
    \item Distributions: Normal, Geometric, Poisson
    \item Today we will derive another distribution, used for estimating proportions
    \item But first...
\end{itemize}

\end{frame}

%%%%%%%%%%%%%%%%%%%%%%%%%%%%%%%%%%%%

\section{Point estimates and sampling variability (Ch. 5.1)}

%%%%%%%%%%%%%%%%%%%%%%%%%%%%%%%%%%%%

\begin{frame}
    \frametitle{Point estimates and error}

    \begin{itemize}

        \item We are often interested in \hl{population parameters}.
        \item Complete populations are difficult to collect data on, so we use \hl{sample statistics} as \hl{point estimates} for the unknown population parameters of interest.
        \item \hl{Error} in the estimate = difference between population parameter and sample statistic
        \item \hl{Bias} is systematic tendency to over- or under-estimate the true population parameter.
        \item \hl{Sampling error} describes how much an estimate will tend to vary from one sample to the next.
        \item Much of statistics is focused on understanding and quantifying sampling error, and \hl{sample size} is helpful for quantifying this error.

    \end{itemize}

\end{frame}

%%%%%%%%%%%%%%%%%%%%%%%%%%%%%%%%

\begin{frame}
    \frametitle{Sampling distributions are never observed}

    \begin{itemize}

        \item In real-world applications, we never actually observe the sampling distribution, yet it is useful to always think of a point estimate as coming from such a hypothetical distribution.
        \item Understanding the sampling distribution will help us characterize and make sense of the point estimates that we do observe.

    \end{itemize}

\end{frame}

%%%%%%%%%%%%%%%%%%%%%%%%%%%%%%%%

\section{R Demo: Sampling distributions}

%%%%%%%%%%%%%%%%%%%%%%%%%%%%%%%%%%

%%%%%%%%%%%%%%%%%%%%%%%%%%%%%%%%

\section{Edfinity quiz}

%%%%%%%%%%%%%%%%%%%%%%%%%%%%%%%%%%

%%%%%%%%%%%%%%%%%%%%%%%%%%%%%%%%

\section{Board work: Deriving theoretical sampling distribution in the Binomial case}

%%%%%%%%%%%%%%%%%%%%%%%%%%%%%%%%%%

\section{Introducing bias for small-sample estimation}

%%%%%%%%%%%%%%%%%%%%%%%%%%%%%%%%

\begin{frame}
    \frametitle{Rule of succession}

    \begin{itemize}

        \item (Placeholder for material)

    \end{itemize}

\end{frame}

%%%%%%%%%%%%%%%%%%%%%%%%%%%%%%%%

\section{Board work: deriving two estimators for Binomial case}

%%%%%%%%%%%%%%%%%%%%%%%%%%%%%%%%

\section{R Demo: Rule of succession}

%%%%%%%%%%%%%%%%%%%%%%%%%%%%%%%%%%%%
% End document
%%%%%%%%%%%%%%%%%%%%%%%%%%%%%%%%%%%%

\end{document}