
%%%%%%%%%%%%%%%%%%%%%%%%%%%%%%%%%%%%
% Recap/Agenda 
%%%%%%%%%%%%%%%%%%%%%%%%%%%%%%%%%%%%
% TODO better formatting
\begin{frame}
    \frametitle{Module 1: Exploratory Data Analysis and Study Design}
    \begin{itemize}
        \item \hl{Previously: } Course Introduction, Intro to Data (Chapter 1)
        \item \hl{This time: } Examining numerical data (Chapter 2.1)
        \item \hl{Reading: } Chapter 2.2 for next time
        \item \hl{Deadlines/Announcements: } 
        \begin{itemize}
            \item Skills Lab 1 will be graded for submission (without errors), not correctness
            \item Lecture attendance for weeks 1 and 2 will not affect the final grades
        \end{itemize}
    \end{itemize}
    
\end{frame}

%%%%%%%%%%%%%%%%%%%%%%%%%%%%%%%%%%%%
% Learning objectives:
%%%%%%%%%%%%%%%%%%%%%%%%%%%%%%%%%%%%
\begin{frame}
    \frametitle{Learning Objectives}    
    \begin{itemize}
        \item \textbf{M1, LO2: Evaluate Study Design and Its Implications:} Identify and explain experimental design choices (observational vs. experimental, sampling methods, blinding, potential biases), and judge whether results can be generalized to a population or used to infer causation. 
        \item \textbf{M1, LO3: Use R for Data Management and Exploration:} Utilize R to load, pre-process, and explore data through visualization and summarization techniques.
        \item \textbf{M1, LO4: Visualize and Describe Data Distributions:} Select appropriate visualizations (scatterplots, histograms, box plots, bar plots) to depict data, and describe distributions qualitatively (shape, center, spread, outliers) and quantitatively (mean, median, mode, range, IQR, standard deviation).
    \end{itemize}
\end{frame}
%%%%%%%%%%%%%%%%%%%%%%%%%%%%%%%%%%%%