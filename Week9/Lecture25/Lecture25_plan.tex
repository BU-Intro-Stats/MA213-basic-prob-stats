%%%%%%%%%%%%%%%%%%%%%%%%%%%%%%%%%%%%
% Lesson Plan (50 minutes)
%%%%%%%%%%%%%%%%%%%%%%%%%%%%%%%%%%%%
\begin{frame}
    \frametitle{Lesson Plan}
    \begin{itemize}
        \item xx min Lecture: motivate paired data problems, give an example (to return to throughout lecture, like the school test data)   
        \item xx min Edfinity quiz (short): given these table data (and maybe a histogram), do the observations appear independent?
        \item xx min Lecture: Review quiz answers
        \item xx min R Demonstration: anlayzing the difference of the paired data (like slide 4)
        \item xx min Think/pair/share: how would you set up a hypothesis test for this problem?
        \item xx min Lecture: Review hypothesis test workflow for this example
        \item xx min R Demonstration: compute test statistic and p-value 
        \item xx min Edfinity quiz: Interpret the p-value we got, plus intuition about the CI
        \item xx min Lecture: Review the correct p-value interpretation
        \item xx min Lecture/board work: Compute the confidence interval
        \item xx min Lecture: very brief preview of next lecture on difference of two means (?)
    \end{itemize}

\end{frame}
            
%%%%%%%%%%%%%%%%%%%%%%%%%%%%%%%%%%%%
% Learning objectives:
%%%%%%%%%%%%%%%%%%%%%%%%%%%%%%%%%%%%
\begin{frame}
    \frametitle{Learning Objectives}
    \begin{itemize}
        \item \textbf{M1, LO3: Use R for Data Management and Exploration:} Utilize R to load, pre-process, and explore data through visualization and summarization techniques.
        \item \textbf{M3, LO1: Understand Point Estimates and Sampling Variability:} Define a sample statistic (point estimate) for a population parameter, and explain how it varies across different samples.
        \item \textbf{M3, LO3: Calculate and Interpret Standard Error:} Calculate the standard error for proportions and interpret it as a measure of sampling variability.
        \item \textbf{M3, LO4: Explain Hypothesis Testing and Its Limitations:} Discuss the use cases and potential issues with hypothesis testing, including the interpretation of results.
        \item \textbf{M3, LO6: Distinguish Statistical vs. Practical Significance:} Differentiate between statistical significance and practical significance, and explain the implications of each.
        \item \textbf{M4, LO2: Design and Interpret Confidence Intervals:} Design, execute, and interpret confidence intervals for the population proportion.
        \item \textbf{M4, LO6: Conduct and Interpret t-Tests:} Design, execute, and interpret t-tests for a single population mean, a difference of paired means, and a difference of independent means, calculating the standard error appropriately for each. Describe how to obtain a p-value for a t-test and a critical t-score for a confidence interval.
    \end{itemize}
\end{frame}
    
%%%%%%%%%%%%%%%%%%%%%%%%%%%%%%%%%%%%
% TODO: Add slides for T/P/S on designing a hypothesis test
% TODO: Fix in-slide computation of p-value to be consistent with using p<dist>() function in R (see R demo)
% TODO: Emily please check R p-value computation, the slide is confusing me
% TODO: Once correct, update p-value in Edfinity quiz
% TODO: Board work on computing CI, if doing it that way (there's also a slide for it at the end)