\begin{frame}
    \frametitle{Module 3: Foundations for inference}
    \begin{itemize}
        \item \hl{Previously: }Point estimates and sampling variability (Chapter 5.1)
        \item \hl{This time: }Confidence intervals (Chapter 5.2)
        \item \hl{Reading: }Chapter 5.3 for next time
        \item \hl{Deadlines/Announcements: }
        \begin{itemize}
            \item HW 6 due on Tuesday
            \item Reminder: Monday is a Holiday, Tuesday is Monday schedule
            \item Quiz 2 \hl{in class on Wednesday}, \textbf{Not in discussion sections}
            \begin{itemize}
                \item If you have accommodations through DAS, you can schedule your quiz through Accommodate
            \end{itemize}
            \item No discussion sections next week
        \end{itemize}
    \end{itemize}
    
\end{frame}
    
%%%%%%%%%%%%%%%%%%%%%%%%%%%%%%%%%%%%
% Learning objectives:
%%%%%%%%%%%%%%%%%%%%%%%%%%%%%%%%%%%%
\begin{frame}
\frametitle{Learning Objectives}
\begin{itemize}
    \item \textbf{M3 LO3: Calculate and Interpret Standard Error:} Calculate the standard error for proportions and interpret it as a measure of sampling variability. 
    \item \textbf{M3 LO4: Explain Hypothesis Testing and Its Limitations:} Discuss the use cases and potential issues with hypothesis testing, including the interpretation of results. 
    \item \textbf{M4 LO1: Design and Interpret Confidence Intervals:} Design, execute, and interpret confidence intervals for the population proportion. 
\end{itemize}
\end{frame}
