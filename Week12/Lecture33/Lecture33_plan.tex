%%%%%%%%%%%%%%%%%%%%%%%%%%%%%%%%%%%%
% Lesson Plan (50 minutes)
%%%%%%%%%%%%%%%%%%%%%%%%%%%%%%%%%%%%
\begin{frame}
    \frametitle{Lesson Plan}
    \begin{itemize}
        \item xx min R Demonstration: recap linear regression so far
        \item xx min Lecture: recap (briefly) our strategies for inference so far (tests, CI, p-values, etc)
        \item xx min Lecture: putting it all together: motivate inference for linear regression, with an example to return to throughout the week
        \item xx min Edfinity quiz: testing for the slope: what should the hypotheses be? (like slide 4)
        \item xx min Lecture: review quiz answers, note t-test inference for regression - idea is to make this now a familiar concept
        \item xx min Edfinity quiz: perform a t-test for the slope parameter, step-by-step
        \item xx min R Demonstration: review quiz answers, interpret the test results together
    \end{itemize}
\end{frame}

%%%%%%%%%%%%%%%%%%%%%%%%%%%%%%%%%%%%
% Learning objectives:
%%%%%%%%%%%%%%%%%%%%%%%%%%%%%%%%%%%%
\begin{frame}
    \frametitle{Learning Objectives}
    \begin{itemize}
        \item \textbf{M1, LO1: Classify and Analyze Variables:} Categorize variables based on their types (e.g., numerical/categorical, continuous/discrete, ordinal), assess their association (positive, negative, or independent), and determine which make sense as explanatory vs. response variables.
        \item \textbf{M1, LO3: Use R for Data Management and Exploration:} Utilize R to load, pre-process, and explore data through visualization and summarization techniques.
        \item \textbf{M1, LO4: Visualize and Describe Data Distributions:} Select appropriate visualizations (scatterplots, histograms, box plots, bar plots) to depict data, and describe distributions qualitatively (shape, center, spread, outliers) and quantitatively (mean, median, mode, range, IQR, standard deviation).
        \item \textbf{M5, LO1: Describe and Assess Relationships Between Two Variables:} Describe the association between two numerical variables in a scatter plot in terms of direction, shape (linear or nonlinear), and strength, and assess whether linear regression is an appropriate model.    
        \item \textbf{M5, LO2: Compute and Interpret Correlation and R²:} Compute and interpret correlation coefficients and R² values, while recognizing that correlation does not imply causation. 
        \item \textbf{M5, LO3: Fit and Interpret Linear Models Using Least Squares:} Fit the intercept and slope of a linear model to data using the least squares method, interpret the fitted values, and use the model to predict responses to new inputs.
        \item \textbf{M5, LO4: Explain and Assess Model Fit:} Explain the least squares fitting procedure, assess whether the fit is unduly influenced by any particular points, and distinguish between interpolation and extrapolation in predictions.
        \item \textbf{M5, LO5: Perform Inference for Regression Coefficients:} Use fit summary and parameter estimates (e.g., ˆβ1 and ˆσ2) to perform hypothesis tests or construct confidence intervals for the slope, and interpret the results.
    \end{itemize}
\end{frame}
    
%%%%%%%%%%%%%%%%%%%%%%%%%%%%%%%%%%%%
% TODO: Adapt drafted slides