%%%%%%%%%%%%%%%%%%%%%%%%%%%%%%%%%%%%
% Lesson Plan (50 minutes)
%%%%%%%%%%%%%%%%%%%%%%%%%%%%%%%%%%%%
\begin{frame}
    \frametitle{Lesson Plan}
    \begin{itemize}
        \item 1 min: Present results from Group formation vote
        \item 3 min Lecture (3 frames): data basics (types)
        \item 15 min R demonstration: loading data, examining dataframe, bar plot, scatterplot
        \item 4 min Lecture (4 frames): data basics (Relationships, Explanatory/response, data collection, association/causation)
        \item 3 min Lecture (3 frames): sampling principles and examples
        \item 5 min Edfinity quiz: Population vs sample
        \item 10 min Lecture (12 frames): sampling bias, good samples, Observational/Sampling
        \item 8 min Think/pair/share: sampling, bias concepts
        \item 1 min Final announcements (Plan for next time, Reading)
    \end{itemize}
\end{frame}
    
%%%%%%%%%%%%%%%%%%%%%%%%%%%%%%%%%%%%
% Learning objectives:
%%%%%%%%%%%%%%%%%%%%%%%%%%%%%%%%%%%%
\begin{frame}
    \frametitle{Learning Objectives}
    \begin{itemize}
        \item \textbf{M1, LO1: Classify and Analyze Variables:} Categorize variables based on their types (e.g., numerical/categorical, continuous/discrete, ordinal), assess their association (positive, negative, or independent), and determine which make sense as explanatory vs. response variables.
        \item \textbf{M1, LO2: Evaluate Study Design and Its Implications:} Identify and explain experimental design choices (observational vs. experimental, sampling methods, blinding, potential biases), and judge whether results can be generalized to a population or used to infer causation. 
        \item \textbf{M1, LO3: Use R for Data Management and Exploration:} Utilize R to load, pre-process, and explore data through visualization and summarization techniques.
    \end{itemize}
\end{frame}

%%%%%%%%%%%%%%%%%%%%%%%%%%%%%%%%%%%%