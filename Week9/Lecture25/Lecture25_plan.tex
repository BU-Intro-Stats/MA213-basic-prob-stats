%%%%%%%%%%%%%%%%%%%%%%%%%%%%%%%%%%%%
% Lesson Plan (50 minutes)
%%%%%%%%%%%%%%%%%%%%%%%%%%%%%%%%%%%%
\begin{frame}
    \frametitle{Lesson Plan}
    % ES: I like the idea of making this lecture very interactive, since it can reinforce the one-sample t-test material
    \begin{itemize}
        \item xx min Lecture: motivate paired data problems, give an example (to return to throughout lecture, like the school test data)
        \begin{itemize}
            \item Something like, "last time we learned about how to do inference for one-sample data, now we're returning to the setting with a larger sample size but we have observations that appear to be paired."
        \end{itemize}    
        % ES: I would interleave slides in here to reinforce concepts, since important points may be lost in demos/quizzes
        \item xx min Edfinity quiz (short): given these table data (and maybe a histogram), do the observations appear independent?
        \item xx min R Demonstration: anlayzing the difference of the paired data 
        \item xx min Think/pair/share: how would you set up a hypothesis test for this problem?
        \item xx min Lecture: Review hypothesis test workflow for this example
        \item xx min R Demonstration: compute test statistic and p-value 
        \item xx min Think/pair/share: how would you interpret the p-value we got? now create a confidence interval and interpret that too (like slides 28-29)
        \item xx min R Demonstration (short): computing the correct CI
        \item xx min Lecture: preview of next lecture on difference of two means (?)
    \end{itemize}

\end{frame}
            
%%%%%%%%%%%%%%%%%%%%%%%%%%%%%%%%%%%%
% Learning objectives:
%%%%%%%%%%%%%%%%%%%%%%%%%%%%%%%%%%%%
\begin{frame}
    \frametitle{Learning Objectives}
    \begin{itemize}
        \item \textbf{M1, LO3: Use R for Data Management and Exploration:} Utilize R to load, pre-process, and explore data through visualization and summarization techniques.
        \item \textbf{M3, LO1: Understand Point Estimates and Sampling Variability:} Define a sample statistic (point estimate) for a population parameter, and explain how it varies across different samples.
        \item \textbf{M3, LO3: Calculate and Interpret Standard Error:} Calculate the standard error for proportions and interpret it as a measure of sampling variability.
        \item \textbf{M3, LO4: Explain Hypothesis Testing and Its Limitations:} Discuss the use cases and potential issues with hypothesis testing, including the interpretation of results.
        \item \textbf{M3, LO6: Distinguish Statistical vs. Practical Significance:} Differentiate between statistical significance and practical significance, and explain the implications of each.
        \item \textbf{M4, LO2: Design and Interpret Confidence Intervals:} Design, execute, and interpret confidence intervals for the population proportion.
        \item \textbf{M4, LO6: Conduct and Interpret t-Tests:} Design, execute, and interpret t-tests for a single population mean, a difference of paired means, and a difference of independent means, calculating the standard error appropriately for each. Describe how to obtain a p-value for a t-test and a critical t-score for a confidence interval.
    \end{itemize}
\end{frame}
    
%%%%%%%%%%%%%%%%%%%%%%%%%%%%%%%%%%%%
% TODO: Interleave slides, edfinity, r demos
% TODO: Edfinity quiz on independence assumption
% TODO: R demo on visualizing paired differences
% TODO: Edfinity quiz / TPS on designing a hypothesis test
% TODO: R demo on computing test statistic, p-value
% TODO: Edfinity quiz on interpreting the p-value
% TODO: Board work / R demo on computing CI