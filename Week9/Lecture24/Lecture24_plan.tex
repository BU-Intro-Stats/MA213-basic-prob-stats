%%%%%%%%%%%%%%%%%%%%%%%%%%%%%%%%%%%%
% Lesson Plan (50 minutes)
%%%%%%%%%%%%%%%%%%%%%%%%%%%%%%%%%%%%
\begin{frame}
    \frametitle{Lesson Plan}
    \begin{itemize}
        % ES: I hate the example of traffic, because it is *paired*, and we should do an unpaired example first. Let's come up with a new example (maybe even the Guiness quality control example?)
        % https://www.scientificamerican.com/article/how-the-guinness-brewery-invented-the-most-important-statistical-method-in/
        % LV: sounds good!
        \item xx min Lecture: Motivate using the t distribution, with a new example
        \item xx min Edfinity quiz: like slides 2-3, check intuition about independence, distribution of the data, and the correct hypotheses
        \begin{itemize} 
            % ES: Love this idea! 
            % ES: It's probably best to prepare just one R demo -- as long as one person brings up the one sample proportion test, we can steer the class towards z-test logic
            % ES: Add some board work to transfer the logic from the one-sample proportion to the z-test
            \item Maybe add to the quiz an open (not trick, but tricky) question: do you think we have any tools yet for handling one-sample data? How would you try it? Then:
            \item xx min Board work: Assuming independence, normality, large sample, could use a test very similar to one-sample proportion
        \end{itemize}
        \item xx min Lecture: Examining the normality assumption
        \item xx min Lecture: t-distribution test statistic, p-value
        \item xx min R Demonstration: compute the test statistic and p-value for the HT in item 2 above, then interpret the results (like slides 10-12)
        \item xx min Edfinity quiz (challenge question): how would you determine what the difference is, given all the info (from R) that you need? (like slides 13-16)
        \item xx min Lecture: review the answer/the correct workflow, then interpret the result together (incl. something like slide 18)
    \end{itemize}
\end{frame}
            
%%%%%%%%%%%%%%%%%%%%%%%%%%%%%%%%%%%%
% Learning objectives:
%%%%%%%%%%%%%%%%%%%%%%%%%%%%%%%%%%%%
\begin{frame}
    \frametitle{Learning Objectives}
    \begin{itemize}
        \item \textbf{M1, LO3: Use R for Data Management and Exploration:} Utilize R to load, pre-process, and explore data through visualization and summarization techniques.
        \item \textbf{M3, LO1: Understand Point Estimates and Sampling Variability:} Define a sample statistic (point estimate) for a population parameter, and explain how it varies across different samples.
        \item \textbf{M3, LO3: Calculate and Interpret Standard Error:} Calculate the standard error for proportions and interpret it as a measure of sampling variability.
        \item \textbf{M3, LO4: Explain Hypothesis Testing and Its Limitations:} Discuss the use cases and potential issues with hypothesis testing, including the interpretation of results.
        \item \textbf{M3, LO6: Distinguish Statistical vs. Practical Significance:} Differentiate between statistical significance and practical significance, and explain the implications of each.
        \item \textbf{M4, LO2: Design and Interpret Confidence Intervals:} Design, execute, and interpret confidence intervals for the population proportion.
        \item \textbf{M4, LO5: Explain and Use the t-Distribution:} Explain how the t-distribution differs from the normal distribution and why it is used for population mean inference.
        \item \textbf{M4, LO6: Conduct and Interpret t-Tests:} Design, execute, and interpret t-tests for a single population mean, a difference of paired means, and a difference of independent means, calculating the standard error appropriately for each. Describe how to obtain a p-value for a t-test and a critical t-score for a confidence interval.
    \end{itemize}
\end{frame}
    
%%%%%%%%%%%%%%%%%%%%%%%%%%%%%%%%%%%%
% TODO: (BIG) Rework the slides to use a new example of a one-sample t-test, that isn't paired. 
% TODO: Edfinity quiz on one sample HT design and assumptions, with open question on tools we've used before
% TODO: Plan board work justifying Z test for normal data
% TODO: (BIG) R demo looking at sampling distributions for the sample mean, testing assumptions wrt the standard normal z-test
% TODO: (optional) Consider using the traffic example as a motivation for the pooled test section (end of this lecture or beginning of next), including the discussion of independence between samples in two-sample context (although the next section does have this discussion)
% TODO: R demo usign t dist to perform HT
% TODO: Edfinity challenge question about CI
% TODO: Edfinity question about interpretations of HT/CIs (e.g. Slide 18)