%%%%%%%%%%%%%%%%%%%%%%%%%%%%%%%%%%%%
% Lesson Plan (50 minutes)
%%%%%%%%%%%%%%%%%%%%%%%%%%%%%%%%%%%%
\begin{frame}
    \frametitle{Lesson Plan}
    \begin{itemize}
        % ---------- From Lecture 8 ----------
        \item 8 min Lecture (4 frames): Expectation and variability
        \item 8 min R Demo: Interpreting theoretical mean and variance (sample the card game RV, make a histogram of the samples, compare it to the discrete distribution, compute mean and variance of the samples, compare them to the theoretical mean and variance)
        \item 4 min Lecture (4 frames): Linear combinations
        % \item 3 min Edfinity quiz: Linear combinations

        % ---------- New for Lecture 9 ----------
        \item 12 min Lecture / Board work (5 frames): RVs recap
        \item 8 min Edfinity Quiz / Group work: linear combinations (word problems)
        \item 5 min Lecture (4 frames): Continuous distributions, probability density functions
        \item xx min R Demonstration: Continuous distribution examples
        \item xx min Edfinity quiz: concepts
    \end{itemize}
\end{frame}

%%%%%%%%%%%%%%%%%%%%%%%%%%%%%%%%%%%%
% Learning objectives:
%%%%%%%%%%%%%%%%%%%%%%%%%%%%%%%%%%%%
\begin{frame}
    \frametitle{Learning Objectives}
    \begin{itemize}
        \item \textbf{M1 LO3: Use R for Data Management and Exploration:} Utilize R to load, pre-process, and explore data through visualization and summarization techniques.
        \item \textbf{M2 LO1: Validate and Explain Probability Distributions:} Assess the validity of a probability distribution using the concepts of outcome, sample space, and probability properties (e.g., disjoint outcomes, probabilities between 0 and 1, and total probabilities summing to 1).
        \item \textbf{M2 LO3: Compute Probabilities Using Various Tools:} Use logic, Venn diagrams, and probability rules to compute probabilities for events.
        \item \textbf{M2 LO4: Understand and Compute Expectations and Variances:} Explain the concepts of expectations and variances of random variables, and compute the expectation and variance of a linear combination of random variables.
        \item \textbf{M2 LO6: Assess Data Using the Normal Distribution:} Use the normal distribution to assess the "unusualness" of data points, apply the 68-95-99.7% rule, evaluate normality through histograms and q-q plots, and determine when a normal approximation to the binomial model is valid for calculating binomial probabilities.
    \end{itemize}
\end{frame}

%%%%%%%%%%%%%%%%%%%%%%%%%%%%%%%%%%%%
% Note: We can leave this space here for now in case we are behind at this point in the semester. But be prepared to do 25 minutes of R demo / active learning on continuous distributions
% TODO: consider changing the numbers in the RVs review to make it easier for board work
% TODO: Edfinity quiz on linear combinations (could do it as think/pair/share interpreting word problems as random variables. And/or break up the questions into (1) translate to RVs, then (2) compute the answer using rules)
% TODO: R demo on continuous distributions
    % Normalized Histogram of a continuous RV: increasing sample size, decreasing bin size
    % Plot continuous distribution on top
    % Point out / think/pair/share: probabilities are areas under the curve, total area is 1
    % Examples of different distributions (Normal, Exponential, Uniform, Bimodal mixture of Gaussians?)
% TODO: Edfinity quiz on continuous distributions
