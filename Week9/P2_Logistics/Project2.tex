\documentclass[12pt]{article}
\usepackage{resource/shortex}
\usepackage{mathtools,amssymb}
\usepackage{enumitem}
\usepackage{datetime}
\usepackage{fancyhdr}
\usepackage{ifthen}
\usepackage{pifont}
\usepackage{mathrsfs}
\usepackage[normalem]{ulem}
%\RequirePackage[usenames,dvipsnames]{color}


%% showing solutions 
\newboolean{showsols}
\setboolean{showsols}{false}
\newcommand{\showsolutions}{\setboolean{showsols}{true}}
\newlength{\solspace}
\setlength{\solspace}{16em}
\newcommand{\nosolspace}{\setlength{\solspace}{0em}}

\newcommand{\solution}[2]{\ifthenelse{\boolean{showsols}}{{\par\textcolor{red}{Rubric criterion: #1}\newline\textcolor{blue}{#2}}}{\vspace{\solspace}}}
\newcommand{\npforprint}{\ifthenelse{\boolean{showsols}}{}{\newpage}}
\newcommand{\vsforprint}[1]{\ifthenelse{\boolean{showsols}}{}{\vspace{#1}}}

%% grading
\newcommand{\grades}[1]{~\newline\noindent\framebox[\textwidth][c]{E\quad \quad \quad \quad S \quad \quad \quad \quad R \quad \quad \quad \quad N}}

\newcommand{\questiongrades}{\noindent\emph{Each item is marked $\checkmark$, $\checkmark\!-$, or \ding{55}:}\begin{center}\emph{$\checkmark$ = capable  $~~$ $\checkmark\!-$ = mostly capable  $~~$ \ding{55} = not yet}\end{center} }

\newcommand{\triAssessments}{$\checkmark\quad\checkmark\!-\quad$\ding{55}\quad}
\newcommand{\biAssessments}{$\checkmark\quad$\ding{55}\quad}

%% line for writing name and BUID
\newcommand{\nameBUID}{\noindent\textbf{Name:}\underline{\phantom{XXXXXXXXXXXXXXXXXXX}}\hfill\textbf{BUID:}\underline{\phantom{XXXXXXXXXXXXX}}\newline}


%% formatting 
%\allowdisplaybreaks[3]
\renewcommand{\headrulewidth}{.4mm} % header line width
\renewcommand{\arraystretch}{1.25}


%% page style 
\pagestyle{fancy}
\fancyhf{}
\rhead{Fall 2025}
\lhead{CAS MA 213: Basic Statistics and Probability}
%\rfoot{\thepage}

%--------------------
\begin{document}

\begin{center}
\textbf{\Large Final Project \\ Statistical Data Analysis } 
\end{center}

	

%%%%%%%%%%%%%%%%%%%%%%%%%%%%%%%%%%%%%%%%%%%%%%%%%%%%%%%%%%%%%%%%%%%%%%%%%%%%%%%%%%%%%%%%%
%%%%%%%%%%%%%%%%%%%%%%%%%%%%%%%%%%%%%%%%%%%%%%%%%%%%%%%%%%%%%%%%%%%%%%%%%%%%%%%%%%%%%%%%%

\section*{Final Project Overview}
In the final project, you will produce a written report of a Statistical Data Analysis using topics you learend from this course. 
It should be a report of 10-15 pages (11pt font, 1.5 or single-spaced, standard margins) including tables and figures.
You can choose the study question. The question can include various types of topics including social science, meddicine, sports, natural science, etc. 
You may use datasets of your interest (you need to cite them in a correct way). 
Therefore, finding a good source of data and an interesting question can be very crucial. 
You will work in groups of four people on the project. 


\subsection*{Suggested Outline}
The following is the suggested outline of your report. 
\begin{enumerate}
    \item \textbf{Title/abstract} : Your report a title and write a brief (e.g., 100-200 words)
    abstract summarizing the goal, execution, and conclusions of your project.  
    \item \textbf{Introduction} : The introduction should summarize the motivation of your project. What is your motivation behind your project and the goals of your work. You can talk about your null and alternative hypothesis. 
    \item \textbf{Methods} : Summarize the basic problem you face and talk about data and how they were collected in the begining. 
    \item \textbf{Results} : Briefly outline the main steps you followed in modeling and analyzing your data, along with the key findings. This may include descriptive statistics, visualizations, diagnostic checks, and important stages in developing your final model. Where relevant, use graphs such as histograms, scatterplots as well as like point estimates, confidence intervals, and hypothesis tests. Ensure all visuals are clearly labeled and easy to interpret. 
    \item \textbf{Discussion} : You can revisit how your goals were acheived or it could have been acheived by improving or modifiying it next time as a future work. You could imply also what future work can be added on to see better result etc. 
    \item \textbf{Supplement Section }: You may include your R code, extra figures and tables that are not part of the main part of the report. From this page, there is no page limitations. 
    \item \textbf{References }: Place any referecnes at the end of the document. Also, please place the citations on the appopriate part of the text. 
\end{enumerate}
    

% \section*{Project Requirements}
% \begin{itemize}
%     \item Maximum 10 pages or report (11pt font, 1.5 or single-spaced, standard margins) including tables and figures.
%     \item Source of data must be cited
%     \item R code should be included (Rscript or Rmd files)
%     \item References should be provided at the end
% \end{itemize}

% \begin{center}\rule{16cm}{1pt}\end{center}
    
        
\section*{Deliverables}
\begin{itemize}
    \item Deliverable 1 : Project outline (due next Friday)
    \item Deliverable 2 : Project progress report
    \item Deliverable 3 : Final project report (due at the end of the lab session)
\end{itemize}




% \section*{Deadlines [todo]}
% \begin{table}[h!]
%     \centering
%     \begin{tabularx}{\textwidth}{|l|X|l|}
%     \hline
%     \textbf{Item} & \textbf{Description} & \textbf{Due} \\
%     \hline
%     Deliverable & Brief introduction of the analysis and roles of group members &  \\
%     \hline
%     Final project report &  &  \\
%     \hline
%     \end{tabularx}
%     \caption{Deliverable and Deadlines}
% \end{table}

% \newpage


\subsection*{Possible examples of the project}
\begin{itemize}
    \item Predicting Body Weight from Height using a Linear Regression Model
    \item Dependency between Smoking Habits and Age Groups using Categorical Data Analysis
    \item Analyzing the Effect of Exercise on Heart Rate using a Linear Regression Model 
    \item Relationship Between Education Level and Life Expectancy
\end{itemize}


\section*{Data Sources to Consider }
\begin{itemize}
    \item \href{https://www.openintro.org/data/}{OpenIntro Datasets}
    \item \href{https://cran.r-project.org/web/packages/fivethirtyeight/vignettes/fivethirtyeight.html}{fivethirtyeight 
    R Package}
    \item \href{https://stat.ethz.ch/R-manual/R-devel/library/datasets/html/00Index.html}{datsets R package}
    \item \href{https://www.kaggle.com/datasets}{Kaggle datasets}
    \item \href{https://data.gov/}{US government's open data}
\end{itemize}

\end{document}
