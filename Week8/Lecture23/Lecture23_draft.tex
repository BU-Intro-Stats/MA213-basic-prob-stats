%%%%%%%%%%%%%%%%%%%%%%%%%%%%%%%%%%%%
% Slide options
%%%%%%%%%%%%%%%%%%%%%%%%%%%%%%%%%%%%

% Option 1: Slides with solutions

\documentclass[slidestop,compress,mathserif]{beamer}
\newcommand{\soln}[1]{\textit{#1}}
\newcommand{\solnGr}[1]{#1}

% Option 2: Handouts without solutions

%\documentclass[11pt,containsverbatim,handout]{beamer}
%\usepackage{pgfpages}
%\pgfpagesuselayout{4 on 1}[letterpaper,landscape,border shrink=5mm]
%\newcommand{\soln}[1]{ }
%\newcommand{\solnGr}{ }

%%%%%%%%%%%%%%%%%%%%%%%%%%%%%%%%%%%%
% Style
%%%%%%%%%%%%%%%%%%%%%%%%%%%%%%%%%%%%

\def\chp6@path{../../Chp 6}
\input{../../lec_style.tex}


%%%%%%%%%%%%%%%%%%%%%%%%%%%%%%%%%%%%
% Preamble
%%%%%%%%%%%%%%%%%%%%%%%%%%%%%%%%%%%%

\title[Chp 6: Inference for categorical data]{Chapter 6: Inference for categorical data}
\author{OpenIntro Statistics, 4th Edition}
\institute{$\:$ \\ {\footnotesize Slides developed by Mine \c{C}etinkaya-Rundel of OpenIntro. \\
The slides may be copied, edited, and/or shared via the \webLink{http://creativecommons.org/licenses/by-sa/3.0/us/}{CC BY-SA license.} \\
Some images may be included under fair use guidelines (educational purposes).}}
\date{}


%%%%%%%%%%%%%%%%%%%%%%%%%%%%%%%%%%%%
% Begin document
%%%%%%%%%%%%%%%%%%%%%%%%%%%%%%%%%%%%

\begin{document}


%%%%%%%%%%%%%%%%%%%%%%%%%%%%%%%%%%%%
% Title page
%%%%%%%%%%%%%%%%%%%%%%%%%%%%%%%%%%%%

{
\addtocounter{framenumber}{-1} 
{\removepagenumbers 
\usebackgroundtemplate{\includegraphics[width=\paperwidth]{../../OpenIntro_Grid_4_3-01.jpg}}
\begin{frame}

\hfill \includegraphics[width=20mm]{../../oiLogo_highres}

\titlepage

\end{frame}
}
}


%%%%%%%%%%%%%%%%%%%%%%%%%%%%%%%%%%%%
% Sections
%%%%%%%%%%%%%%%%%%%%%%%%%%%%%%%%%%%%

%%%%%%%%%%%%%%%%%%%%%%%%%%%%%%%%%%%%

\section{Review}

\begin{frame}
    \frametitle{Last time: chi-squared goodness-of-fit tests}
    
    Context:
    \begin{itemize}
        \item ...
    \end{itemize}
    
    Necessary conditions: 
    \begin{itemize}
        \item ...
    \end{itemize}

    Next, another kind of chi-squared test. But first, a challenge question:
\end{frame}

%%%%%%%%%%%%%%%%%%%%%%%%%%%%%%%%%%%%

\section{Edfinity Quiz: What hypotheses would you use for testing independence?}

%%%%%%%%%%%%%%%%%%%%%%%%%%%%%%%%%%%%

\section{Chi-square test of independence}

%%%%%%%%%%%%%%%%%%%%%%%%%%%%%%%%%%%

\subsection{Popular kids}

%%%%%%%%%%%%%%%%%%%%%%%%%%%%%%%%%%%

\begin{frame}
\frametitle{Popular kids}

\dq{In the dataset \texttt{popular}, students in grades 4-6 were asked whether good grades, athletic ability, or popularity was most important to them. A two-way table separating the students by grade and by choice of most important factor is shown below. Do these data provide evidence to suggest that goals vary by grade?}

\twocol{0.5}{0.5}
{
\begin{center}
\begin{tabular}{rrrr}
  \hline
 & Grades & Popular & Sports \\ 
  \hline
$4^{th}$ &  63 &  31 &  25 \\ 
$5^{th}$ &  88 &  55 &  33 \\ 
$6^{th}$ &  96 &  55 &  32 \\ 
   \hline
\end{tabular}
\end{center}
}
{
\begin{center}
\includegraphics[width=0.8\textwidth]{\chp6@path/6-4_chisq_indep/figures/popular/popular_mosaic}
\end{center}
}


\end{frame}

%%%%%%%%%%%%%%%%%%%%%%%%%%%%%%%%%%%

\begin{frame}
\frametitle{Chi-square test of independence}

\begin{itemize}
\item The hypotheses are:
\begin{itemize}
\item[$H_0$:] Grade and goals are independent. Goals do not vary by grade.
\item[$H_A$:] Grade and goals are dependent. Goals vary by grade.
\end{itemize}

\pause

\item The test statistic is calculated as
\[ \chi^2_{df} = \sum_{i = 1}^{k} \frac{(O - E)^2}{E} \quad \text{ where } \quad df = (R - 1) \times (C - 1), \]
where $k$ is the number of cells, $R$ is the number of rows, and $C$ is the number of columns.

\Note{We calculate $df$ differently for one-way and two-way tables.}

\pause

\item The p-value is the area under the $\chi^2_{df}$ curve, above the calculated test statistic.

\end{itemize}


\end{frame}

%%%%%%%%%%%%%%%%%%%%%%%%%%%%%%%%%%%

\subsection{Expected counts in two-way tables}

%%%%%%%%%%%%%%%%%%%%%%%%%%%%%%%%%%%

\begin{frame}
\frametitle{Expected counts in two-way tables}

\formula{Expected counts in two-way tables}
{
\[ \text{Expected Count} = \frac{(\text{row total}) \times (\text{column total})}{\text{table total}} \]
}

\pause

{\small
\begin{center}
\begin{tabular}{rrrr|r}
  \hline
 & Grades & Popular & Sports	& Total \\ 
  \hline
$4^{th}$ &  \orange{63} &  \green{31} &  25 	&119 \\ 
$5^{th}$ &  88 &  55 &  33	& 176 \\ 
$6^{th}$&  96 &  55 &  32	& 183 \\ 
   \hline
Total	& 247	& 141	& 90	& 478 \\
\end{tabular}
\end{center}
}

\pause

\[ \orange{$E_{row~1, col~1} = \frac{119 \times 247}{478} = 61$} \qquad \pause
 \green{$E_{row~1, col~2} = \frac{119 \times 141}{478} = 35$} \]

\end{frame}

%%%%%%%%%%%%%%%%%%%%%%%%%%%%%%%%%%%

\begin{frame}
\frametitle{Expected counts in two-way tables}

\pq{What is the expected count for the highlighted cell?}

{\small
\begin{center}
\begin{tabular}{rrrr|r}
  \hline
 & Grades & Popular & Sports	& Total \\ 
  \hline
$4^{th}$ &  63 &  31 &  25 	&119 \\ 
$5^{th}$ &  88 &  \orange{55} &  33	& 176 \\ 
$6^{th}$ &  96 &  55 &  32	& 183 \\ 
   \hline
Total	& 247	& 141	& 90	& 478 \\
\end{tabular}
\end{center}
}

\twocol{0.2}{0.8}
{
\begin{enumerate}[(a)]
\solnMult{$\frac{176 \times 141}{478}$}
\item $\frac{119 \times 141}{478}$
\item $\frac{176 \times 247}{478}$
\item $\frac{176 \times 478}{478}$
\end{enumerate}
}
{
\soln{\only<2>{
\orange{$\rightarrow$ 52\\
{\small more than expected \# of 5th graders \\
have a goal of being popular}}
\vspace{0.75cm}
}
}
}

\end{frame}

%%%%%%%%%%%%%%%%%%%%%%%%%%%%%%%%%%%

\begin{frame}
\frametitle{Calculating the test statistic in two-way tables}

Expected counts are shown in \ex{blue} next to the observed counts.
\begin{center}
\begin{tabular}{rrrr|r}
  \hline
 & Grades & Popular & Sports	& Total \\ 
  \hline
$4^{th}$ 	&  63 \ex{61} &  31 \ex{35} &  25 \ex{23}	&119 \\ 
$5^{th}$ 	&  88 \ex{91} &  55 \ex{52} &  33 \ex{33}	& 176 \\ 
$6^{th}$	&  96 \ex{95} &  55 \ex{54} &  32 \ex{34}	& 183 \\ 
   \hline
Total	& 247	& 141	& 90	& 478 \\
\end{tabular}
\end{center}

\vspace{0.5cm}

\pause

\begin{eqnarray*} 
\chi^2 &=& \sum \frac{(63 - 61)^2}{61} + \frac{(31 - 35)^2}{35} + \cdots + \frac{(32 - 34)^2}{34} = 1.3121 \\
\pause
df &=& (R - 1) \times (C - 1) = (3 - 1) \times (3 - 1) = 2 \times 2 = 4 
\end{eqnarray*}

\end{frame}

%%%%%%%%%%%%%%%%%%%%%%%%%%%%%%%%%%%

\section{Edfinity Quiz: Practice with two-way tables}

%%%%%%%%%%%%%%%%%%%%%%%%%%%%%%%%%%%

\subsection{Results}

%%%%%%%%%%%%%%%%%%%%%%%%%%%%%%%%%%%

\begin{frame}
\frametitle{Calculating the p-value}

\pq{Which of the following is the correct p-value for this hypothesis test?
\[ \chi^2 = 1.3121 \qquad df = 4 \]
}

\twocol{0.6}{0.4}{
\begin{center}
\includegraphics[width=0.67\textwidth]{\chp6@path/6-4_chisq_indep/figures/popular/popular}
\end{center}
}
{
{\small
\begin{enumerate}[(a)]
\setlength{\itemsep}{0in}
\solnMult{ more than 0.3}
\item between 0.3 and 0.2
\item between 0.2 and 0.1
\item between 0.1 and 0.05
\item less than 0.001
\end{enumerate}
}
}

\end{frame}

%%%%%%%%%%%%%%%%%%%%%%%%%%%%%%%%%%%

\begin{frame}
\frametitle{Conclusion}

\dq{Do these data provide evidence to suggest that goals vary by grade?}

\begin{itemize}

\item[$H_0$:] Grade and goals are independent. Goals do not vary by grade.

\item[$H_A$:] Grade and goals are dependent. Goals vary by grade. \\

\end{itemize}

$\:$ \\

\soln{\only<2>{Since p-value is high, we fail to reject $H_0$. The data do not provide convincing evidence that grade and goals are dependent. It doesn't appear that goals vary by grade.
}
}

\end{frame}

%%%%%%%%%%%%%%%%%%%%%%%%%%%%%%%%%%%

\section{Edfinity Quiz: Putting it all together}

%%%%%%%%%%%%%%%%%%%%%%%%%%%%%%%%%%%

%%%%%%%%%%%%%%%%%%%%%%%%%%%%%%%%%%%%
% End document
%%%%%%%%%%%%%%%%%%%%%%%%%%%%%%%%%%%%

\end{document}