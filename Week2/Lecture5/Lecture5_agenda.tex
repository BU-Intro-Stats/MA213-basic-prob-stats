
%%%%%%%%%%%%%%%%%%%%%%%%%%%%%%%%%%%%
% Recap/Agenda 
%%%%%%%%%%%%%%%%%%%%%%%%%%%%%%%%%%%%
% TODO better formatting
\begin{frame}
    \frametitle{Module 1: Exploratory Data Analysis and Study Design}
    \begin{itemize}
        \item \hl{Previously: } Considering categorial data (Chapter 2.2)
        \item \hl{This time: } Two case studies
        \item \hl{Reading: } Chapter 3.1 for next time
        \item \hl{Deadlines/Announcements: Homework 2 Due Monday} 
    \end{itemize}
    
\end{frame}

%%%%%%%%%%%%%%%%%%%%%%%%%%%%%%%%%%%%
% Learning objectives:
%%%%%%%%%%%%%%%%%%%%%%%%%%%%%%%%%%%%
\begin{frame}
    \frametitle{Learning Objectives}
    \begin{itemize}
        \item \textbf{M1, LO2: Evaluate Study Design and Its Implications:} Identify and explain experimental design choices (observational vs. experimental, sampling methods, blinding, potential biases), and judge whether results can be generalized to a population or used to infer causation. 
        \item \textbf{M1, LO3: Use R for Data Management and Exploration:} Utilize R to load, pre-process, and explore data through visualization and summarization techniques.
        \item \textbf{M1, LO4: Visualize and Describe Data Distributions:} Select appropriate visualizations (scatterplots, histograms, box plots, bar plots) to depict data, and describe distributions qualitatively (shape, center, spread, outliers) and quantitatively (mean, median, mode, range, IQR, standard deviation).
        \item \textbf{M1, L05: Conduct Hypothesis Testing Using Simulation:} Set up null and alternative hypotheses to test for independence between variables, and use simulation techniques to evaluate data support for these hypotheses.
    \end{itemize}
\end{frame}
