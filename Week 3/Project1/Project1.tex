\documentclass[11pt]{article}
\usepackage[top=1in,bottom=1in,right=1in,left=1in]{geometry}
\usepackage{pdflscape}
\usepackage{amsmath}
\usepackage{amsfonts}
\usepackage{graphicx}
\usepackage[english]{babel}
\usepackage{amsmath}
\usepackage{amssymb}
\usepackage{setspace}
\usepackage{array}
\usepackage{tcolorbox}
\usepackage{arydshln}
\usepackage{hyperref}
\usepackage{float}
\usepackage{tabularx} 
\usepackage{bm}
\usepackage{enumitem}
\usepackage{geometry}
\usepackage{fontawesome5}



\newcolumntype{L}[1]{>{\raggedright\let\newline\\\arraybackslash\hspace{0pt}}m{#1}}
\newcolumntype{C}[1]{>{\centering\let\newline\\\arraybackslash\hspace{0pt}}m{#1}}
\newcolumntype{R}[1]{>{\raggedleft\let\newline\\\arraybackslash\hspace{0pt}}m{#1}}

\doublespacing

\begin{document}
	
	
	
	\begin{center}		
		%\noindent\rule{16cm}{1pt}\vspace{0.5em}
        Boston University \\
        Department of Mathematics and Statistics \\ 
		\Large{\textbf{MA213 - Fall 2025}} \\ 
        \large{Basic Statistics and Probability} \\
        \large{Project 1 : Data Exploration} \\ 
		\noindent\rule{16cm}{2pt}
	\end{center}
	
	%%%%%%%%%%%%%%%%%%%%%%%%%%%%%%%%%%%%%%%%%%%%%%%%%%%%%%%%%%%%%%%%%%%%%%%%%%%%%%%%%%%%%%%%%
	%%%%%%%%%%%%%%%%%%%%%%%%%%%%%%%%%%%%%%%%%%%%%%%%%%%%%%%%%%%%%%%%%%%%%%%%%%%%%%%%%%%%%%%%%
	
% \large{\textbf{Project 1 : Exploration of the Data}} 
% \normalsize
% \section*{ \centering  Project 1 : Exploration of the Data}
% \begin{center}\rule{16cm}{1pt}\end{center}
% \textbf{Overview of the Project \\}  
\section*{Overview}
Your group will select a dataset of your interest and conduct exploratory data analysis.
This should include :
\begin{itemize}
    \item One numerical exploratory data anlysis
    \item One categorical exploratory data analysis 
\end{itemize} 
You will present your analyses through either:
\begin{itemize}
    \item An in-class presentation 
    \item A pre-reorded video presentation
\end{itemize}
\begin{center}\rule{16cm}{1pt}\end{center}
\section*{Suggested Outline}
\begin{itemize}
    \item Introduction
    \begin{enumerate}
        \item Introduce the dataset and its source
        \item Tell the story of your reasoning and possible hypothesis
        \item Define your variables of interest
    \end{enumerate}
    \item Data Analysis
    \begin{enumerate}
        \item Numerical Exploratory Data Analysis Part
        \begin{enumerate}
            \item Summary statistics 
            \item The relationship between two numerical variables 
            \item Distribution of variable(s) of interest
            \item Discuss shape, central tendency, spreadity and outliers. 
        \end{enumerate}
        \item Categorical Exploratory Data Analysis Part
        \begin{enumerate}
            \item Summarize categorical variables
            \item Contingency Table (two categorical varaibles)
            \item Visualizations: bar plots, pie charts, etc.
        \end{enumerate} 
    \end{enumerate}
    \item Conclusion
        \begin{enumerate}
            \item Summarize key insights
            \item Suggest future analysis or potential applications
        \end{enumerate} 
\end{itemize}
\begin{center}\rule{16cm}{1pt}\end{center}
\section*{Project requirements}
\begin{itemize}
    \item Source of data must be cited
    \item R code should be included (Rscript or Rmd files)
    \item References should be provided at the end
\end{itemize}
\begin{center}\rule{16cm}{1pt}\end{center}

\section*{Deadlines}
\begin{table}[h!]
    \centering
    \begin{tabularx}{\textwidth}{|l|X|l|}
    \hline
    \textbf{Item} & \textbf{Description} & \textbf{Due} \\
    \hline
    Deliverable & Brief introduction of the analysis and roles of group members & Week 5 (after Lab 3) \\
    \hline
    R Script or RMarkdown File & Include data preparation, analysis, and visualizations & Week 7 \\
    \hline
    Slide File & 4–5 slides (excluding Title and Reference slides); 5-minute presentation & Week 7 \\
    \hline
    \end{tabularx}
    \caption{Deliverable and Deadlines}
\end{table}

\newpage
\section*{Data Sources to Consider}
\begin{itemize}
    \item \href{https://www.openintro.org/data/}{OpenIntro Datasets}
    \item \href{https://cran.r-project.org/web/packages/fivethirtyeight/vignettes/fivethirtyeight.html}{fivethirtyeight 
    R Package}
    \item \href{https://stat.ethz.ch/R-manual/R-devel/library/datasets/html/00Index.html}{datsets R package}
    \item \href{https://www.kaggle.com/datasets}{Kaggle datasets}
    \item \href{https://data.gov/}{US government's open data}
\end{itemize}

% \subsection*{Possible topics can be considered}
% \begin{itemize}
%     \item 
% \end{itemize}


\newpage
\section*{Evaluation Criteria}

\begin{table}[h!]
    \centering
    \begin{tabularx}{\textwidth}{|XXX|}
        \hline
        \textbf{Pass} & \textbf{Almost Pass} & \textbf{Not Yet} \\
        \hline
    \end{tabularx}
\end{table}

\textit{Each item is marked with \faCheck, \faCheck$-$, or \faTimes:}



\begin{table}[h!]
    \centering

    % Legend Row
    \begin{tabularx}{\textwidth}{XXX}
        \faCheck = \textit{satisfied} &
        \faCheck{}$-$ = \textit{mostly satisfied} &
        \faTimes = \textit{not yet} \\
    \end{tabularx}

    % Evaluation Rows
    \begin{tabularx}{\textwidth}{cccX}
        \hline
        \faCheck & \faCheck{}$-$ & \faTimes  & Introduction section correctly summarize motivation, data, result and takeaways\\
        \faCheck & \faCheck{}$-$ & \faTimes &  Code correctly implemented (Results can be easily reproduced) \\
        \faCheck & \faCheck{}$-$ & \faTimes  & Conclusion section is coherent and the direction for future work is interesting and plausible \\
        \faCheck & \faCheck{}$-$ & \faTimes & Discussed shape, central tendency, spreadity in a correct manner  \\ 
        \faCheck & \faCheck{}$-$ & \faTimes & Plots and summaries support your explanation \\
        \faCheck & \faCheck{}$-$ & \faTimes & Well participated in the project as a group work \\ 
        \faCheck & \faCheck{}$-$ & \faTimes & Answered one question from the presentation \\
        \faCheck & \faCheck{}$-$ & \faTimes & Topic and the results are interesting and easy to understand \\ 
        \faCheck & \faCheck{}$-$ & \faTimes & Asked a relevent/valid question to other group \\
\hline
    \end{tabularx}

\end{table}




\section*{Grading Criteria}
\begin{itemize}
    \item \textbf{Pass:} \faCheck on 7 of 9 rubric items, no \faTimes marks
    \item \textbf{Almost Pass:} \faCheck on 3 of 9 rubric items, no \faTimes marks
\end{itemize}



\end{document}
