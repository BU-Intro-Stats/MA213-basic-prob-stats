%%%%%%%%%%%%%%%%%%%%%%%%%%%%%%%%%%%%
% Lesson Plan (50 minutes)
%%%%%%%%%%%%%%%%%%%%%%%%%%%%%%%%%%%%
\begin{frame}
    \frametitle{Lesson Plan}
    \begin{itemize}
        % last time: chi-squared GOF tests
        \item xx min Lecture: brief recap of last time, motivate tests of independence (maybe with an example)
        \item xx min Lecture: review the necessary conditions for a chi-squared test
        \item xx min Edfinity quiz (very short): what do you think the hypotheses might be in such a test? (open answer)
        \item xx min Lecture: explain and interpret the answers, then derive the test statistic and connect it to two-way tables
        \item xx min Edfinity quiz: practice computing expected counts with two-way tables
        \item xx min Lecture: deriving the test statistic and p-values from a two-way table
        \item xx min Edfinity quiz: practice, putting it all together
        \item xx min R Demonstration: (not sure about this one, maybe how to compute p-values from R?)
        % Might be cool to continue using the motivating example from last lecture here, and making up some table data for it
    \end{itemize}
\end{frame}
            
%%%%%%%%%%%%%%%%%%%%%%%%%%%%%%%%%%%%
% Learning objectives:
%%%%%%%%%%%%%%%%%%%%%%%%%%%%%%%%%%%%
\begin{frame}
    \frametitle{Learning Objectives}
    \begin{itemize}
        % Commenting out the R LO since I'm not sure what an R demo for this lecture could be
        %\item \textbf{M1, LO3: Use R for Data Management and Exploration:} Utilize R to load, pre-process, and explore data through visualization and summarization techniques.
        \item \textbf{M3, LO3: Calculate and Interpret Standard Error:} Calculate the standard error for proportions and interpret it as a measure of sampling variability.
        \item \textbf{M4, LO2: Design and Interpret Confidence Intervals:} Design, execute, and interpret confidence intervals for the population proportion.
        \item \textbf{M4, LO3: Conduct and Interpret Hypothesis Tests for Proportions:} Design, execute, and interpret hypothesis tests for population proportions.
        \item \textbf{M4, LO4: Conduct and Interpret Chi-Square Tests:} Assess whether the conditions for a chi-square test (goodness of fit or independence) are met, and if so, design, execute, and interpret the test.
    \end{itemize}
\end{frame}
    
%%%%%%%%%%%%%%%%%%%%%%%%%%%%%%%%%%%%
% TODO: Fill in recap slide(s)
% TODO: Edfinity quiz (or forum discussion?) on independence, prep board work or slides to derive the test statistic
% TODO: Edfinity quiz on computing expected counts
% TODO: Edfinity quiz on a complete two-way table problem
% TODO: (Optional) R demo e.g. another interesting problem worked through?
