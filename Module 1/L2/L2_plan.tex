

%%%%%%%%%%%%%%%%%%%%%%%%%%%%%%%%%%%%
% Lesson Plan (50 minutes)
%%%%%%%%%%%%%%%%%%%%%%%%%%%%%%%%%%%%
\begin{frame}
\frametitle{Lesson Plan}
\begin{itemize}
    \item xx min Exploring Data, what type of information...?
    \item xx min Some basic built-in R plot functions to show data... how to summarize?
    \item xx min Scatterplot, Scatterplots are useful for visualizing the relationship between two
    numerical variables.
    \item xx min data manipulation -- dplyr syntax? 
    \item xx min ggplot2 packages and  
    \item xx min Data explore by groups
\end{itemize}
\end{frame}


%%%%%%%%%%%%%%%%%%%%%%%%%%%%%%%%%%%%
% Learning objectives:
%%%%%%%%%%%%%%%%%%%%%%%%%%%%%%%%%%%%
\begin{frame}
\frametitle{Learning Objectives}

\begin{itemize}
    \item Use R for Data Management and Exploration: Utilize R to load, pre-process, and explore data through visualization and summarization techniques. [Q1, L1, L2]
    \item Classify and Analyze Variables: Categorize variables based on their types (e.g., numerical/categorical, continuous/discrete, ordinal), assess their association (positive, negative, or independent), and determine which make sense as explanatory vs. response variables. [Q1, L2] Core
\end{itemize}
\end{frame}

\begin{frame}
    \frametitle{Numerical Data}
    \begin{itemize}
        \item Scatterplots are useful for visualizing the relationship between two
        numerical variables.
        \item Dot plots; Useful for visualizing one numerical variable. 
        \item Sample mean 
        \item Stacked dot plot
        \item Histograms; and talk about Bind width
        \item Shape of a distribution : modality (unimodal, bimodal/multimodal, uniform)
        \item Shape of a distribution : skewness
        \item Outliers
        \item Variance and standard deviation 
        \item Median
        \item Q1, Q3, IQR
        \item Box plot
        \item 
    \end{itemize}
\end{frame}


\begin{frame}
    \frametitle{Categorical Data}
    
    \begin{itemize}
        \item Contingency Tables; A table that summarizes data for two categorical variables is called
        a contingency table.
        \item Bar plot
        \item Pie charts
        \item Simulation ??
    \end{itemize}
    \end{frame}




