%%%%%%%%%%%%%%%%%%%%%%%%%%%%%%%%%%%%
% Lesson Plan (50 minutes)
%%%%%%%%%%%%%%%%%%%%%%%%%%%%%%%%%%%%
\begin{frame}
    \frametitle{Lesson Plan}
    \begin{itemize}
        \item Review motivation from last time, maybe a related example to the paired-data example
        \item \begin{itemize}
            \item eg. if we did reading and writing scores last time, maybe this time we look at reading scores of two different schools (and want to see if they differ significantly)
            \item or, scores at the same school but different grades (and the question might be, is the higher grade achieving a higher score, i.e. students are improving over time?)
        \end{itemize}
        \item xx min Edfinity quiz: for this problem, what is the parameter of interest? what is the point estimate? how would you set up hypotheses? (like slides 3-5)
        \item xx min R Demonstration: computing the test statistic, p-value (like slides 6-9)
        \item xx min Edfinity quiz: what can you conclude from this hypothesis test? can you construct a CI? how do you interpret it? (like slides 10-13)
        \item xx min Lecture: Review the worklow and conclusions (like recap, slide 14)
    \end{itemize}
\end{frame}
% Note: a little concerned that all the lectures this week will feel a bit repetitive, since (as planned here) they all follow the same basic steps
% ...though maybe it's good to get the practice? what is the right balance between (productive) reinforcement and (boring) repetition?

%%%%%%%%%%%%%%%%%%%%%%%%%%%%%%%%%%%%
% Learning objectives:
%%%%%%%%%%%%%%%%%%%%%%%%%%%%%%%%%%%%
\begin{frame}
    \frametitle{Learning Objectives}
    \begin{itemize}
        \item \textbf{M1, LO3: Use R for Data Management and Exploration:} Utilize R to load, pre-process, and explore data through visualization and summarization techniques.
        \item \textbf{M3, LO1: Understand Point Estimates and Sampling Variability:} Define a sample statistic (point estimate) for a population parameter, and explain how it varies across different samples.
        \item \textbf{M3, LO3: Calculate and Interpret Standard Error:} Calculate the standard error for proportions and interpret it as a measure of sampling variability.
        \item \textbf{M3, LO4: Explain Hypothesis Testing and Its Limitations:} Discuss the use cases and potential issues with hypothesis testing, including the interpretation of results.
        \item \textbf{M3, LO6: Distinguish Statistical vs. Practical Significance:} Differentiate between statistical significance and practical significance, and explain the implications of each.
        \item \textbf{M4, LO2: Design and Interpret Confidence Intervals:} Design, execute, and interpret confidence intervals for the population proportion.
        \item \textbf{M4, LO6: Conduct and Interpret t-Tests:} Design, execute, and interpret t-tests for a single population mean, a difference of paired means, and a difference of independent means, calculating the standard error appropriately for each. Describe how to obtain a p-value for a t-test and a critical t-score for a confidence interval.    
    \end{itemize}
\end{frame}
    
%%%%%%%%%%%%%%%%%%%%%%%%%%%%%%%%%%%%
% TODO: Adapt drafted slides