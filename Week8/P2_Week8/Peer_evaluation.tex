\documentclass{article}

% --- LaTeX Packages ---
\usepackage[utf8]{inputenc} 
\usepackage[T1]{fontenc}    
\usepackage{geometry}       
\geometry{letterpaper, margin=1in} 

\usepackage{enumitem}      
\usepackage{tabularx}       
\usepackage{array}         

\begin{document}

% --- Peer Evaluation Form ---
\begin{center}
  {\Huge\bfseries Peer Evaluation Form \textsuperscript{*}} % Main title
  \vspace{0.5cm}

  \hrule % Horizontal line
  \vspace{0.5cm}
\end{center}


Please write the names of all your team members, INCLUDING YOURSELF, and rate the degree
to which each member fulfilled his/her responsibilities in completing the group assignments.
Use the following rating system:


\begin{itemize}[leftmargin=*, label=\textbullet]
    \item \textbf{Excellent}: Outstanding contribution, extra effort, and leadership.
    \item \textbf{Satisfactory}: Completed responsibilities adequately, met expectations.
    \item \textbf{Unsatisfactory}: Frequently missed or did incomplete work, did not meet expectations.
    \item \textbf{Failing}: Had no participation at all. No effort was made to attend meetings or complete responsibilities.
\end{itemize}

\vspace{0.5cm}

These ratings should reflect each individual's level of participation and effort and sense of responsibility, not his or her academic ability.

\vspace{0.5cm}

% Table for names and ratings
\begin{center}
\begin{tabularx}{0.8\textwidth}{>{\RaggedRight\arraybackslash}X|>{\centering\arraybackslash}X}

\textbf{\Large Name of Team Member} & \textbf{\Large Rating} \\
\hline
\vspace{1.5em} & \\ \\
\hline
\vspace{1.5em} & \\ \\
\hline
\vspace{1.5em} & \\ \\
\hline
\vspace{1.5em} & \\ \\
\hline
\vspace{1.5em} & \\ \\
\hline
\end{tabularx}
\end{center}

\vspace{1cm}



\begin{center}
\begin{tabularx}{0.8\textwidth}{X}
\textbf{\Large Comments} \\
\hline
\vspace{5cm} \\ \\ \\ \\ \\ \\ \\ \\ \\ \\ \\ \\
\hrule
\end{tabularx}
\end{center}


\hrule
\vspace{0.5cm}
\noindent * Adapted from Kaufman, D.M., Felder, R.M. and Fuller H (2000). Accounting for Individual Effort in Cooperative Learning Teams. \textit{Journal of Engineering Education}, April 2000, 133–140.

\end{document}