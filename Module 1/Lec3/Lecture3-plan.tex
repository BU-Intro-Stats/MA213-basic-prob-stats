%%%%%%%%%%%%%%%%%%%%%%%%%%%%%%%%%%%%
% Lesson Plan (50 minutes)
%%%%%%%%%%%%%%%%%%%%%%%%%%%%%%%%%%%%
\begin{frame}
    \frametitle{Lesson Plan}
    \begin{itemize}
        \item xx min Lecture: plotting data, means
        \item xx min R demonstration: examples of plots and computations
        \begin{itemize}
            \item  highlight: distribution modality, skewness, outliers
        \end{itemize}
        \item xx min Edfinity quiz (what kind of shapes do you expect these data to have...)
        \item xx min R demonstration: quiz answer(s)
        \item xx min Lecture: variance, stddev, median
        \item xx min R demonstration: Comparing shapes of distribution with different summary statistics (eg. changing variance, etc.)
        \item Transformations? Intensity maps?
    \end{itemize}
    \end{frame}
    
    %%%%%%%%%%%%%%%%%%%%%%%%%%%%%%%%%%%%
    % Learning objectives:
    %%%%%%%%%%%%%%%%%%%%%%%%%%%%%%%%%%%%
    \begin{frame}
    \frametitle{Learning Objectives}
    \begin{itemize}
        \item Categorize variables based on their types (e.g., numerical/categorical, continuous/discrete, ordinal), assess their association (positive, negative, or independent), and determine which make sense as explanatory vs. response variables. [Q1, L2]
        \item Select appropriate visualizations (scatterplots, histograms, box plots, bar plots) to depict data, and describe distributions qualitatively (shape, center, spread, outliers) and quantitatively (mean, median, mode, range, IQR, standard deviation). [Q1, P1] 
    \end{itemize}
    \end{frame}