
%%%%%%%%%%%%%%%%%%%%%%%%%%%%%%%%%%%%
% Recap/Agenda 
%%%%%%%%%%%%%%%%%%%%%%%%%%%%%%%%%%%%
% TODO better formatting
\begin{frame}
    \frametitle{Module 2: Probability, Random Variables, and Distributions}
    \begin{itemize}
        \item \hl{Previously: } Normal distribution (Chapter 4.1)
        \item \hl{This time: } Normal distribution, continued
        \item \hl{Reading: } Chapter 4.2 for next time
        \item \hl{Deadlines/Announcements: } 
        \begin{itemize}
            \item HW 4 due Monday
            \item Quiz 1 grades are up on Gradescope
            \item Next week: Quiz 1 retake qualifications and Almost Pass conversions
        \end{itemize}
    \end{itemize}
    
\end{frame}

\begin{frame}
    \frametitle{Quiz 1}
    \begin{itemize}
        \item Reminder of the \hl{\href{https://docs.google.com/spreadsheets/d/1yAqE1HUeVbe6xuhddNSIX7Y3zuvFHlsuIn8Ys7V9_a0}{Grade Tracker}}
        \item There were 4 Core learning objectives and 1 Auxiliary learning objective on Quiz 1
        \item Each was graded Pass / Almost Pass / Not Yet
        \begin{itemize}
            \item Pass: Correct response. You do not need to take this Learning Objective again, even if you retake the quiz to cover other objectives.
            \item Almost Pass: Minor errors (e.g. typos). You can turn an Almost Pass into a Pass by \hl{attending the Office Hours} for Prof Stephen or your discussion TF within a week of receiving your grade on the quiz, explaining the error, and fixing it. Correct explanations will be converted to ``Pass"; incorrect explanations will be converted to ``Not Yet''.
            \item Not Yet: Incorrect, incomplete, or insufficient. To turn a Not Yet into a Pass, you need to \hl{retake the quiz}.
        \end{itemize}
    \end{itemize}
\end{frame}

\begin{frame}
\frametitle{Quiz Retakes}
    \begin{itemize}
        \item You can retake up to three quizzes
        \item Retakes will happen at the end of the semester (during Final Exam period + one TBD date)
        \item In order to retake a quiz, you need to \hl{qualify} within one week of receiving your grade on the quiz
        \item To qualify, you need to:
        \begin{itemize}
            \item \hl{Print} The \hl{\href{https://learn.bu.edu/ultra/courses/_246685_1/outline/file/_16630786_1}{Retake Qualification Form}} (Available on the course website in the Discussions folder, or on Gradescope)
            \item Fill it out \hl{completely} (including the blank copy of the quiz)
            \item Bring it to \hl{Office Hours} for Prof Stephen or your discussion TF to get it signed
            \item Submit the signed form to \hl{Gradescope} (under ``Quiz 1 Retake Qualification'') by the deadline
        \end{itemize}
    \end{itemize}
\end{frame}

\begin{frame}
    \frametitle{Office Hours Appointments}
    \begin{itemize}
        \item To manage demand, we will be booking \hl{5 minute appointments} for Office Hours next week
        \begin{itemize}
            \item Prof Stephen (Mon 12:20-1:30pm, Thurs 9-10am)
            \item James Zheng Yang (Tues 12:30-2:30pm)
            \item Will Stride (Wed 3:00-5:00pm)
        \end{itemize}
        \item Appointments will be for both
        \begin{itemize}
            \item Converting Almost Pass to Pass
            \item Quiz retake qualifications
        \end{itemize}
        \item Bookings will be through Google Appointments -- \hl{links were sent out via email}
    \end{itemize}
\end{frame}

%%%%%%%%%%%%%%%%%%%%%%%%%%%%%%%%%%%%
% Learning objectives:
%%%%%%%%%%%%%%%%%%%%%%%%%%%%%%%%%%%%
\begin{frame}
    \frametitle{Learning Objectives}
    \begin{itemize}
        \item \textbf{M2 LO3: Compute Probabilities Using Various Tools:} Use logic, Venn diagrams, and probability rules to compute probabilities for events.
        \item \textbf{M2 LO4: Understand and Compute Expectations and Variances:} Explain the concepts of expectations and variances of random variables, and compute the expectation and variance of a linear combination of random variables.
        \item \textbf{M2 LO6: Assess Data Using the Normal Distribution:} Use the normal distribution to assess the "unusualness" of data points, apply the 68-95-99.7% rule, evaluate normality through histograms and q-q plots, and determine when a normal approximation to the binomial model is valid for calculating binomial probabilities.
    \end{itemize}
\end{frame}
