%%%%%%%%%%%%%%%%%%%%%%%%%%%%%%%%%%%%
% Lesson Plan (50 minutes)
%%%%%%%%%%%%%%%%%%%%%%%%%%%%%%%%%%%%
\begin{frame}
    \frametitle{Lesson Plan}
    \begin{itemize}
        \item 4 min Lecture (4 frames): Box plot and outliers
        \item 4 min R demo: Box plots with whiskers, outliers (similar to R demo from Lecture 3)
        \item 3 min Edfinity quiz: Outliers -- example where the outlier is likely an error
        \item 5 min Forum brainstorm: what to do with outliers
        \item 5 min: Applet demo: Robust statistics (e.g. mean vs median, IQR vs standard deviation) https://www.isi-stats.com/isi2nd/ISIapplets2021.html
        \item 3 min Lecture (3 frames) Robust Statistics 
        \item 3 min: Edfinity quiz: Mean >< Median
        \item 2 min Lecture (3 frames): Transformations and intensity maps
        \item 8 min Lecture (8 frames): contingency tables
        \item xx min Edfinity quiz(zes) or another active learning (contingency tables, interpreting plots from examples)
        \item xx min Lecture: Set up problem for Case Study next lecture
    \end{itemize}
\end{frame}

%%%%%%%%%%%%%%%%%%%%%%%%%%%%%%%%%%%%
% Learning objectives:
%%%%%%%%%%%%%%%%%%%%%%%%%%%%%%%%%%%%
\begin{frame}
    \frametitle{Learning Objectives}
    \begin{itemize}
        \item \textbf{M1, LO1: Classify and Analyze Variables:} Categorize variables based on their types (e.g., numerical/categorical, continuous/discrete, ordinal), assess their association (positive, negative, or independent), and determine which make sense as explanatory vs. response variables.
        \item \textbf{M1, LO3: Use R for Data Management and Exploration:} Utilize R to load, pre-process, and explore data through visualization and summarization techniques.
        \item \textbf{M1, LO4: Visualize and Describe Data Distributions:} Select appropriate visualizations (scatterplots, histograms, box plots, bar plots) to depict data, and describe distributions qualitatively (shape, center, spread, outliers) and quantitatively (mean, median, mode, range, IQR, standard deviation).
    \end{itemize}
\end{frame}

%%%%%%%%%%%%%%%%%%%%%%%%%%%%%%%%%%%%
% TODO: Add a slide introducing the robust statistics applet, and a link to it, with questions: 
    % How much does the mean vs median change when we add an outlier?
    % How does the IQR compare to the standard deviation?
% TODO: Edfinity Quiz on Mean/median difference for a skewed distribution, and change with outliers (start with similar mean/median, then add an outlier)