\documentclass[12pt]{article}
\usepackage[hidelinks]{hyperref}
\usepackage{resource/shortex}
\usepackage{mathtools,amssymb}
\usepackage{enumitem}
\usepackage{datetime}
\usepackage{fancyhdr}
\usepackage{ifthen}
\usepackage{pifont}
\usepackage{mathrsfs}
\usepackage[normalem]{ulem}
%\RequirePackage[usenames,dvipsnames]{color}


%% showing solutions 
\newboolean{showsols}
\setboolean{showsols}{false}
\newcommand{\showsolutions}{\setboolean{showsols}{true}}
\newlength{\solspace}
\setlength{\solspace}{16em}
\newcommand{\nosolspace}{\setlength{\solspace}{0em}}

\newcommand{\solution}[2]{\ifthenelse{\boolean{showsols}}{{\par\textcolor{red}{Rubric criterion: #1}\newline\textcolor{blue}{#2}}}{\vspace{\solspace}}}
\newcommand{\npforprint}{\ifthenelse{\boolean{showsols}}{}{\newpage}}
\newcommand{\vsforprint}[1]{\ifthenelse{\boolean{showsols}}{}{\vspace{#1}}}

%% grading
\newcommand{\grades}[1]{~\newline\noindent\framebox[\textwidth][c]{Pass \quad \quad Almost Pass \quad \quad Not Yet}}

\newcommand{\questiongrades}{\noindent\emph{Each item is marked $\checkmark$, $\checkmark\!-$, or \ding{55}:}\begin{center}\emph{$\checkmark$ = capable  $~~$ $\checkmark\!-$ = mostly capable  $~~$ \ding{55} = not yet}\end{center} }

\newcommand{\triAssessments}{$\checkmark\quad\checkmark\!-\quad$\ding{55}\quad}
\newcommand{\biAssessments}{$\checkmark\quad$\ding{55}\quad}

%% line for writing name and BUID
\newcommand{\nameBUID}{\noindent\textbf{Name:}\underline{\phantom{XXXXXXXXXXXXXXXXXXX}}\hfill\textbf{BUID:}\underline{\phantom{XXXXXXXXXXXXX}}\newline}


%% formatting 
%\allowdisplaybreaks[3]
\renewcommand{\headrulewidth}{.4mm} % header line width
\renewcommand{\arraystretch}{1.25}


%% page style 
\pagestyle{fancy}
\fancyhf{}
\rhead{Fall 2025}
\lhead{CAS MA 213: Basic Statistics and Probability}
%\rfoot{\thepage}
\setlength{\parskip}{0.5em}

%--------------------
\begin{document}

\begin{center}
\textbf{\Large Final Project \\ Statistical Data Analysis} 
\end{center}

%%%%%%%%%%%%%%%%%%%%%%%%%%%%%%%%%%%%%%%%%%%%%%%%%%%%%%%%%%%%%%%%%%%%%%%%%%%%%%%%%%%%%%%%%
%%%%%%%%%%%%%%%%%%%%%%%%%%%%%%%%%%%%%%%%%%%%%%%%%%%%%%%%%%%%%%%%%%%%%%%%%%%%%%%%%%%%%%%%%

\section*{Final Project Overview}

In the final project, you will produce a written report of a Statistical Data Analysis using topics you learned in this course.  
The report should be 10--15 pages (11pt font, 1.5 or single-spaced, standard margins), including tables and figures.  
You can choose the study question, which may cover various fields such as social science, medicine, sports, natural science, etc.  
You may use datasets of your interest (please cite them correctly).  
Therefore, finding a good data source and an interesting question is crucial.  
You will work in groups of four people on this project.

\subsection*{Suggested Outline}

The following is a suggested outline of your report:  
\begin{enumerate}
    \item \textbf{Title and Abstract:} Your report should have a title and include a brief abstract (e.g., 100--200 words) summarizing the goal, execution, and conclusions of your project.  
    \item \textbf{Introduction:} Summarize the motivation behind your project and the goals of your work. You may describe your null and alternative hypotheses.  
    \item \textbf{Methods:} Describe the problem and the data, including how they were collected, at the beginning of this section.  
    \item \textbf{Results:} Outline the main steps you followed to model and analyze your data, along with the key findings. Include descriptive statistics, visualizations, diagnostic checks, and important stages in developing your final model. Use graphs such as histograms or scatterplots, and statistical measures such as point estimates, confidence intervals, and hypothesis tests. Ensure all visuals are clearly labeled and easy to interpret.  
    \item \textbf{Discussion:} Discuss how your goals were achieved, potential improvements, and possible future work to enhance the results or answer further questions.  
    \item \textbf{Supplement section:} You may include your R code, additional figures, and tables not part of the main report. There is no page limit for this section.  
    \item \textbf{References:} List any references at the end of the report, and ensure citations are placed appropriately in the text.  
\end{enumerate}

\section*{Deliverables}

\begin{itemize}
    \item Deliverable 1: Project outline (due next Friday)  
    \item Deliverable 2: Project progress report  
    \item Deliverable 3: Final project report  
\end{itemize}

\subsection*{Possible examples of the project}

\begin{itemize}
    \item Comparing the proportion of smokers across different age groups  
    \item Testing the association between gender and preference for a product (two-proportion test)  
    \item Hypothesis testing on the effect of a new drug on patient recovery rates  
    \item Comparing pass rates between two different teaching methods  
\end{itemize}

\section*{Data Sources to Consider}

\begin{itemize}
    \item \href{https://www.openintro.org/data/}{OpenIntro Datasets}  
    \item \href{https://cran.r-project.org/web/packages/fivethirtyeight/vignettes/fivethirtyeight.html}{fivethirtyeight R package}  
    \item \href{https://stat.ethz.ch/R-manual/R-devel/library/datasets/html/00Index.html}{datasets R package}  
    \item \href{https://www.kaggle.com/datasets}{Kaggle Datasets}  
    \item \href{https://data.gov/}{US Government’s Open Data}  
\end{itemize}

\end{document}