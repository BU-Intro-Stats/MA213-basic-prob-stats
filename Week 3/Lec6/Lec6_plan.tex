%%%%%%%%%%%%%%%%%%%%%%%%%%%%%%%%%%%%
% Lesson Plan (50 minutes)
%%%%%%%%%%%%%%%%%%%%%%%%%%%%%%%%%%%%
\begin{frame}
    \frametitle{Lesson Plan}
    \begin{itemize}
        \item xx min Lecture: Probability, Law of Large Numbers
        \item xx min R demonstration: LLN
        \item xx min Edfinity quiz (LLN concept check)
        \item xx min Lecture: disjoint outcomes
        \item xx min Edfinity quiz: computing probabilities of disjoint and non-disjoint events
        \item xx min Lecture: Sample space, independence
        \item xx min Edfinity quiz (complements, checking for independence)
        \item xx min Lecture: recap + putting everything together
    \end{itemize}
\end{frame}

%%%%%%%%%%%%%%%%%%%%%%%%%%%%%%%%%%%%
% Learning objectives:
%%%%%%%%%%%%%%%%%%%%%%%%%%%%%%%%%%%%
\begin{frame}
    \frametitle{Learning Objectives}
    \begin{itemize}
        \item \textbf{M1, LO3: Use R for Data Management and Exploration:} Utilize R to load, pre-process, and explore data through visualization and summarization techniques.
        \item \textbf{M2, LO1: Validate and Explain Probability Distributions:} Assess the validity of a probability distribution using the concepts of outcome, sample space, and probability properties (e.g., disjoint outcomes, probabilities between 0 and 1, and total probabilities summing to 1).
        \item \textbf{M2, LO2: Apply the Law of Large Numbers and Its Implications:} Explain the Law of Large Numbers, why it holds, and its implications for predicting long-term averages in probability and statistics.
        \item \textbf{Compute Probabilities Using Various Tools:} Use logic, Venn diagrams, and probability rules to compute probabilities for events.
    \end{itemize}
\end{frame}

% TODO: Adapt the drafted lecture slides
