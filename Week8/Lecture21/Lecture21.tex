%%%%%%%%%%%%%%%%%%%%%%%%%%%%%%%%%%%%
% Slide options
%%%%%%%%%%%%%%%%%%%%%%%%%%%%%%%%%%%%

% Option 1: Slides with solutions

\documentclass[slidestop,compress,mathserif]{beamer}
\newcommand{\soln}[1]{\textit{#1}}
\newcommand{\solnGr}[1]{#1}

% Option 2: Handouts without solutions

%\documentclass[11pt,containsverbatim,handout]{beamer}
%\usepackage{pgfpages}
%\pgfpagesuselayout{4 on 1}[letterpaper,landscape,border shrink=5mm]
%\newcommand{\soln}[1]{ }
%\newcommand{\solnGr}{ }

%%%%%%%%%%%%%%%%%%%%%%%%%%%%%%%%%%%%
% Style
%%%%%%%%%%%%%%%%%%%%%%%%%%%%%%%%%%%%

\def\chp6@path{../../Chp 6}
\input{../../lec_style.tex}

%%%%%%%%%%%%%%%%%%%%%%%%%%%%%%%%%%%%
% Preamble
%%%%%%%%%%%%%%%%%%%%%%%%%%%%%%%%%%%%

\title[Lecture 21]{MA213: Lecture 21}
\subtitle{Module 4: Inference}
\author{OpenIntro Statistics, 4th Edition}
\institute{$\:$ \\ {\footnotesize Based on slides developed by Mine \c{C}etinkaya-Rundel of OpenIntro. \\
The slides may be copied, edited, and/or shared via the \webLink{http://creativecommons.org/licenses/by-sa/3.0/us/}{CC BY-SA license.} \\
Some images may be included under fair use guidelines (educational purposes).}}
\date{}


%%%%%%%%%%%%%%%%%%%%%%%%%%%%%%%%%%%%
% Begin document
%%%%%%%%%%%%%%%%%%%%%%%%%%%%%%%%%%%%

\begin{document}


%%%%%%%%%%%%%%%%%%%%%%%%%%%%%%%%%%%%
% Title page
%%%%%%%%%%%%%%%%%%%%%%%%%%%%%%%%%%%%

{
\addtocounter{framenumber}{-1} 
{\removepagenumbers 
\usebackgroundtemplate{\includegraphics[width=\paperwidth]{../../OpenIntro_Grid_4_3-01.jpg}}
\begin{frame}

\hfill \includegraphics[width=20mm]{../../oiLogo_highres}

\titlepage

\end{frame}
}
}


%%%%%%%%%%%%%%%%%%%%%%%%%%%%%%%%%%%%
% Recap/Agenda 
%%%%%%%%%%%%%%%%%%%%%%%%%%%%%%%%%%%%
% TODO better formatting
\begin{frame}
    \frametitle{Module 3: Foundations for inference}
    \begin{itemize}
        \item \hl{Previously: }Inference for a single proportion (Chapter 6.1)
        \item \hl{This time: }Difference of two proportions (Chapter 6.2)
        \item \hl{Reading: }Chapter 6.3 for next time
        \item \hl{Deadlines/Announcements: }
    \end{itemize}
    
\end{frame}

%%%%%%%%%%%%%%%%%%%%%%%%%%%%%%%%%%%%
% Learning objectives:
%%%%%%%%%%%%%%%%%%%%%%%%%%%%%%%%%%%%
\begin{frame}
    \frametitle{Learning Objectives}
    \begin{itemize}
        \item \textbf{M4, LO2: Design and Interpret Confidence Intervals:} Design, execute, and interpret confidence intervals for the population proportion.
        \item \textbf{M4, LO3: Conduct and Interpret Hypothesis Tests for Proportions:} Design, execute, and interpret hypothesis tests for population proportions.
    \end{itemize}
\end{frame}
    

%%%%%%%%%%%%%%%%%%%%%%%%%%%%%%%%%%%%
% Sections
%%%%%%%%%%%%%%%%%%%%%%%%%%%%%%%%%%%%

%%%%%%%%%%%%%%%%%%%%%%%%%%%%%%%%%%%%

\section{Review: Inference for one proportion}

\begin{frame}
\frametitle{6.1 Inference for a single proportion}
We used inference for a single proportion in Chapter 5 to motivate confidence intervals and hypothesis tests.

Chapter 6.1 provides a review of inference for a single proportion, including:
    \begin{itemize}
        \item The sampling distribution of the sample proportion
        \item Confidence intervals for a single proportion
        \item Hypothesis tests for a single proportion
        \item (New) Choosing a sample size for a desired margin of error
    \end{itemize}

\end{frame}

%%%%%%%%%%%%%%%%%%%%%%%%%%%%%%%%%%

\begin{frame}
\frametitle{Sampling distribution of \(\hat{p}\)}

The sampling distribution for \(\hat{p}\) based on a sample of size \(n\) from a population with true proportion \(p\) is nearly normal when:

\begin{enumerate}
  \item The sample's observations are independent (e.g., from a simple random sample).
  \item We expect at least 10 successes and 10 failures in the sample, i.e. \(np \ge 10\) and \(n(1-p) \ge 10\) (the success-failure condition).
\end{enumerate}

When these conditions are met, \(\hat{p}\) is approximately normal with
\[ \hat{p} \sim Normal \pr{ mean = p, SE = \sqrt{\frac{p~(1-p)}{n}} } \]

\end{frame}

%%%%%%%%%%%%%%%%%%%%%%%%%%%%%%%%%%

\begin{frame}
\frametitle{Confidence interval for a single proportion}

If a confidence interval is appropriate:

\begin{description}
  \item[\hl{Prepare.}] Identify $\hat{p}$ and $n$, and choose the confidence level (e.g. 95\%).
  \item[\hl{Check.}] Verify conditions for approximate normality: independence (random sample / 10\% rule) and success-failure. For confidence intervals use $\hat{p}$ in the success-failure check: $\;n\hat{p}\ge10\;$ and $\;n(1-\hat{p})\ge10$.
  \item[\hl{Calculate.}] Compute the standard error using $\hat{p}$ and find the critical value $z^\star$:
  \[
    SE(\hat{p})=\sqrt{\dfrac{\hat{p}(1-\hat{p})}{n}}, \qquad
    \text{CI: }\hat{p}\pm z^\star \,SE(\hat{p}).
  \]
  \item[\hl{Conclude.}] Interpret the interval in context (what the confidence level means for the parameter).
\end{description}

\end{frame}

%%%%%%%%%%%%%%%%%%%%%%%%%%%%%%%%%%

\begin{frame}
\frametitle{Hypothesis testing for a single proportion}

If a hypothesis test is appropriate:

\begin{small}
\begin{description}
  \item[\hl{Prepare.}] State the parameter, write $H_0$ and $H_A$ (one or two-sided), choose significance level $\alpha$, and record $\hat{p}$ and $n$.
  \item[\hl{Check.}] Verify independence (random sample / 10\% rule) and the success-failure condition \emph{using the null value} $p_0$: \[
    np_0 \ge 10 \quad\text{and}\quad n(1-p_0)\ge 10.
  \]
  \item[\hl{Calculate.}] Use $p_0$ to compute the standard error and then the Z statistic:
  \[
    SE = \sqrt{\dfrac{p_0(1-p_0)}{n}}, \qquad
    Z=\dfrac{\hat{p}-p_0}{SE}.
  \]
  Find the p-value corresponding to $Z$ (one- or two-sided).
  \item[\hl{Conclude.}] Compare the p-value to $\alpha$: reject $H_0$ if p-value $<\alpha$, otherwise fail to reject. State the result in context.
\end{description}
\end{small}
\end{frame}

%%%%%%%%%%%%%%%%%%%%%%%%%%%%%%%%%%

\begin{frame}
\frametitle{CI vs. HT for proportions}

\begin{itemize}

\item Success-failure condition:
\begin{itemize}
\item CI: At least 10 \orange{observed} successes and failures
\item HT: At least 10 \orange{expected} successes and failures, calculated using the null value
\end{itemize}

\item Standard error:
\begin{itemize}
\item CI: calculate using observed sample proportion: $SE = \sqrt{\frac{\hat{p}(1-\hat{p})}{n}}$
\item HT: calculate using the null value: $SE = \sqrt{\frac{p_0(1-p_0)}{n}}$
\end{itemize}

\end{itemize}

\end{frame}

%%%%%%%%%%%%%%%%%%%%%%%%%%%%%%%%%%%%

\section{Difference of two proportions}

%%%%%%%%%%%%%%%%%%%%%%%%%%%%%%%%%%%%

\begin{frame}
\frametitle{Melting ice cap}

\pq{Scientists predict that global warming may have big effects on the polar regions within the next 100 years. One of the possible effects is that the northern ice cap may completely melt. Would this bother you a great deal, some, a little, or not at all if it actually happened?}

\begin{enumerate}[(a)]
\item A great deal
\item Some
\item A little
\item Not at all
\end{enumerate}

\end{frame}

%%%%%%%%%%%%%%%%%%%%%%%%%%%%%%%%%%%

\begin{frame}
\frametitle{Results from the GSS}

The GSS asks the same question, below are the distributions of responses from the 2010 GSS as well as from a group of introductory statistics students at Duke University: \\

\begin{center}
\begin{tabular}{l r r}
\hline
				& GSS	& Duke \\
\hline
A great deal		& 454	& 69 \\
Some			& 124 	& 30\\
A little			& 52 		& 4\\
Not at all			& 50 		& 2 \\
\hline
Total				& 680 	& 105\\
\hline
\end{tabular}
\end{center}

\end{frame}

%%%%%%%%%%%%%%%%%%%%%%%%%%%%%%%%%%

\begin{frame}
\frametitle{Parameter and point estimate}

\begin{itemize}

\item \hl{Parameter of interest:} Difference between the proportions of \orange{all} Duke students and \orange{all} Americans who would be bothered a great deal by the northern ice cap completely melting.
\[ \mathhl{ p_{Duke} - p_{US} }\]

\pause

\item \hl{Point estimate:} Difference between the proportions of \orange{sampled} Duke students and \orange{sampled} Americans who would be bothered a great deal by the northern ice cap completely melting.
\[ \mathhl{ \hat{p}_{Duke} - \hat{p}_{US} }\]

\end{itemize}

\end{frame}

%%%%%%%%%%%%%%%%%%%%%%%%%%%%%%%%%%%

\begin{frame}
\frametitle{Inference for comparing proportions}

\begin{itemize}

\item The details are the same as before...

\pause

\item CI: \textcolor{orange}{$point~estimate \pm margin~of~error$}

\pause

\item HT: Use \textcolor{orange}{$Z = \frac{point~estimate - null~value}{SE}$} to find appropriate p-value.

\pause

\item We just need the appropriate standard error of the point estimate ($SE_{ \hat{p}_{Duke} - \hat{p}_{US}}$), which is the only new concept.

\end{itemize}

\pause

\formula{Standard error of the difference between two sample proportions}
{
\[ SE_{(\hat{p}_1 - \hat{p}_2)} = \sqrt{ \frac{p_1(1-p_1)}{n_1} + \frac{p_2(1-p_2)}{n_2} } \]
}

\end{frame}

%%%%%%%%%%%%%%%%%%%%%%%%%%%%%%%%%%%%%

\subsection{Confidence intervals for difference of proportions}

%%%%%%%%%%%%%%%%%%%%%%%%%%%%%%%%%%%%%

\begin{frame}
\frametitle{Conditions for CI for difference of proportions}

\begin{enumerate}

\item \hl{Independence within groups: }
\begin{itemize}
\item The US group is sampled randomly and we're assuming that the Duke group represents a random sample as well.
\pause
\item 105 $<$ 10\% of all Duke students and 680 $<$ 10\% of all Americans.
\end{itemize}
\pause
We can assume that the attitudes of Duke students in the sample are independent of each other, and attitudes of US residents in the sample are independent of each other as well.

\pause

\item \hl{Independence between groups: }
The sampled Duke students and the US residents are independent of each other.

\pause

\item \hl{Success-failure:} \\
At least 10 observed successes and 10 observed failures in the two groups.

\end{enumerate}

\end{frame}

%%%%%%%%%%%%%%%%%%%%%%%%%%%%%%%%%%%%%

\begin{frame}
\frametitle{}

\dq{Construct a 95\% confidence interval for the difference between the proportions of Duke students and Americans who would be bothered a great deal by the melting of the northern ice cap \orange{($p_{Duke} - p_{US}$)}.}

{\footnotesize
\begin{center}
\begin{tabular}{l | c c}
Data			& Duke		& US \\
\hline
A great deal	& 69			& 454 \\
Not a great deal& 36			& 226 \\
\hline
Total			& 105		& 680 \\
\hline
\pause
$\hat{p}$		& 0.657		& 0.668
\end{tabular}
\end{center}
}

\pause

\soln{
\begin{eqnarray*}
&& (\hat{p}_{Duke} - \hat{p}_{US}) \pm z^\star \times \sqrt{ \frac{ \hat{p}_{Duke} (1 - \hat{p}_{Duke})}{n_{Duke} } + \frac{ \hat{p}_{US} (1 -  \hat{p}_{US})}{n_{US} } }  \\
\pause
&=& (0.657 - 0.668) \pause \pm 1.96 \pause \times \sqrt{ \frac{0.657 \times 0.343}{105} + \frac{0.668 \times 0.332}{680} } \\
\pause
&=& -0.011 \pm \pause 1.96 \times 0.0497 \\
\pause
&=& -0.011 \pm 0.097 \\
\pause
&=& (-0.108, 0.086)
\end{eqnarray*}
}


\end{frame}

%%%%%%%%%%%%%%%%%%%%%%%%%%%%%%%%%%
\section{Edfinity Quiz: Interpreting the CI}
%%%%%%%%%%%%%%%%%%%%%%%%%%%%%%%%%%

%%%%%%%%%%%%%%%%%%%%%%%%%%%%%%%%%%%
\section{R Demonstration: Sample size for proportions}
%%%%%%%%%%%%%%%%%%%%%%%%%%%%%%%%%%%

\begin{frame}{Comparing Two Online Ads: Small Study}
\small
Imagine a company tests two versions of an online ad by showing each to 10{,}000 users.

\medskip
\setlength{\tabcolsep}{4pt}
\renewcommand{\arraystretch}{1.1}
\begin{tabular}{lccc}
\textbf{Ad} & Clicked & Total & \(\hat p\) \\
\hline
A & 510 & 10{,}000 & 0.051 \\
B & 550 & 10{,}000 & 0.055 \\
\end{tabular}

\medskip
\[
\hat{p}_B - \hat{p}_A = 0.004
\]

\[
SE = \sqrt{\frac{0.051(1-0.051)}{10{,}000} + \frac{0.055(1-0.055)}{10{,}000}} \approx 0.0031
\]

\[
95\%~CI = 0.004 \pm 1.96(0.0031) = (-0.002,\, 0.010)
\]

\textbf{Interpretation:} The interval includes 0, so we \textbf{do not have statistical evidence}
that Ad B performs better than Ad A.
\end{frame}

%%%%%%%%%%%%%%%%%%%%%%%%%%%%%%%%%%%%%%

\begin{frame}{Comparing Two Online Ads: Large Study}
\small
Now imagine they did the same thing, but each ad shown to 1{,}000{,}000 users. And they observed the same proportions (5.1\% vs 5.5\%).

\medskip
\[
SE = \sqrt{\frac{0.051(1-0.051)}{1{,}000{,}000} + \frac{0.055(1-0.055)}{1{,}000{,}000}}
      \approx 0.00031
\]
\[
95\%~CI = 0.004 \pm 1.96(0.00031) = (0.0034,\, 0.0046)
\]

\textbf{Interpretation:} The interval no longer includes 0.  
The difference is \textbf{statistically significant.}

\medskip
\textbf{But...} the effect size is only \(\mathbf{0.004}\) (0.4 percentage points) --  
does that matter in practice?
\end{frame}

%%%%%%%%%%%%%%%%%%%%%%%%%%%%%%%%%%%%%

\begin{frame}{Statistical vs.\ Practical Significance}
\small
\begin{center}
\begin{tabular}{lccc}
\textbf{Scenario} & \textbf{Sample Size} & \textbf{95\% CI} & \textbf{Stat.\ Sig.?} \\
\hline
Small study & 10{,}000 per ad & (-0.002,\; 0.010) & No \\
Large study & 1{,}000{,}000 per ad & (0.0034,\; 0.0046) & Yes \\
\end{tabular}
\end{center}

\medskip
\textbf{Takeaways:}
\begin{itemize}
  \item Statistical significance depends on both \hl{effect size} and \hl{sample size}.
  \item Practical significance depends on whether the effect is \hl{large enough to matter}.
  \item With large enough data, even tiny effects can be statistically significant.
\end{itemize}

\bigskip
\textit{Statistical significance tells you it's unlikely to be due to chance.  \\
Practical significance tells you if it matters.}
\end{frame}


%%%%%%%%%%%%%%%%%%%%%%%%%%%%%%%%%%%%%

\section{HT for comparing proportions}

%%%%%%%%%%%%%%%%%%%%%%%%%%%%%%%%%%%%%

\begin{frame}

\pq{Which of the following is the correct set of hypotheses for testing if the proportion of all Duke students who would be bothered a great deal by the melting of the northern ice cap differs from the proportion of all Americans who do?}

\begin{enumerate}[(a)]
\solnMult{ $H_0:  p_{Duke} = p_{US}$ \\
$H_A:  p_{Duke} \ne p_{US}$ }
\item $H_0:  \hat{p}_{Duke} = \hat{p}_{US}$ \\
$H_A:  \hat{p}_{Duke} \ne \hat{p}_{US}$
\solnMult{ $H_0:  p_{Duke} - p_{US} = 0$ \\
$H_A:  p_{Duke} - p_{US} \ne 0$ }
\item $H_0:  p_{Duke} = p_{US}$ \\
$H_A:  p_{Duke} < p_{US}$
\end{enumerate}

\soln{
\only<2>{\orange{Both (a) and (c) are correct.}}
}

\end{frame}

%%%%%%%%%%%%%%%%%%%%%%%%%%%%%%%%%%
\section{Edfinity Quiz: HT for difference in proportions}
%%%%%%%%%%%%%%%%%%%%%%%%%%%%%%%%%%

\begin{frame}
\frametitle{Flashback to working with one proportion}

\begin{itemize}

\item When constructing a confidence interval for a population proportion, we check if the \orange{observed} number of successes and failures are at least 10.
\[ n\hat{p} \ge 10 \qquad \qquad n(1-\hat{p}) \ge 10 \]

\pause

\item When conducting a hypothesis test for a population proportion, we check if the \orange{expected} number of successes and failures are at least 10.
\[ np_0 \ge 10 \qquad \qquad n(1-p_0) \ge 10 \]

\end{itemize}

\end{frame}

%%%%%%%%%%%%%%%%%%%%%%%%%%%%%%%%%%%

\begin{frame}
\frametitle{Pooled estimate of a proportion}

\begin{itemize}

\item In the case of comparing two proportions where $H_0: p_1 = p_2$, there isn't a given null value we can use to calculated the \orange{expected} number of successes and failures in each sample.

\pause

\item Therefore, we need to first find a common (\hl{pooled}) proportion for the two groups, and use that in our analysis.

\pause

\item This simply means finding the proportion of total successes among the total number of observations.

\end{itemize}

$\:$ \\

\formula{Pooled estimate of a proportion}
{ \[ \hat{p} = \frac{\#~of~successes_1 + \#~of~successes_2}{n_1 + n_2} \] }

\end{frame}

%%%%%%%%%%%%%%%%%%%%%%%%%%%%%%%%%%%

\begin{frame}
\frametitle{}

\dq{Calculate the estimated \underline{pooled proportion} of Duke students and Americans who would be bothered a great deal by the melting of the northern ice cap. Which sample proportion ($\hat{p}_{Duke}$ or $\hat{p}_{US}$) the pooled estimate is closer to? Why?}

{\footnotesize
\begin{center}
\begin{tabular}{l | c c}
Data			& Duke		& US \\
\hline
A great deal	& 69			& 454 \\
Not a great deal& 36			& 226 \\
\hline
Total			& 105		& 680 \\
\hline
$\hat{p}$		& 0.657		& 0.668
\end{tabular}
\end{center}
}

\pause

\soln{
\begin{eqnarray*}
\hat{p} &=& \frac{\#~of~successes_1 + \#~of~successes_2}{n_1 + n_2} \\
\pause
&=& \frac{69+454}{105+680} \pause = \frac{523}{785} \pause = 0.666
\end{eqnarray*}
}

\end{frame}
%%%%%%%%%%%%%%%%%%%%%%%%%%%%%%%%%%%

\begin{frame}
\frametitle{}

\dq{Do these data suggest that the proportion of all Duke students who would be bothered a great deal by the melting of the northern ice cap differs from the proportion of all Americans who do? Calculate the test statistic, the p-value, and interpret your conclusion in context of the data.}

{\footnotesize
\begin{center}
\begin{tabular}{l | c c}
Data			& Duke		& US \\
\hline
A great deal	& 69			& 454 \\
Not a great deal& 36			& 226 \\
\hline
Total			& 105		& 680 \\
\hline
$\hat{p}$		& 0.657		& 0.668
\end{tabular}
\end{center}
}

\pause

\soln{
\begin{eqnarray*}
Z &=& \frac{(\hat{p}_{Duke} - \hat{p}_{US})}{\sqrt{ \frac{ \hat{p} (1 - \hat{p})}{n_{Duke} } + \frac{ \hat{p} (1 -  \hat{p})}{n_{US} } }} \\
\pause 
&=& \frac{(0.657 - 0.668)}{\sqrt{ \frac{0.666 \times 0.334}{105} + \frac{0.666 \times 0.334}{680} }} = \pause \frac{-0.011}{0.0495} \pause = -0.22 \\
\pause
p-value &=& 2 \times P(Z < -0.22) \pause = 2 \times 0.41 = 0.82
\end{eqnarray*}
}

\end{frame}


%%%%%%%%%%%%%%%%%%%%%%%%%%%%%%%%%%%

\subsection{Recap}

%%%%%%%%%%%%%%%%%%%%%%%%%%%%%%%%%%%

\begin{frame}
\frametitle{Recap - comparing two proportions}

\begin{itemize}

\item Population parameter: $(p_1 - p_2)$, point estimate: $(\hat{p}_1 - \hat{p}_2)$

\pause

\item Conditions:
\pause
\begin{itemize}
\item independence within groups \\
- random sample and 10\% condition met for both groups
\item independence between groups
\item at least 10 successes and failures in each group\\ 
- if not $\rightarrow$ randomization (i.e. simulation-based inference)
\end{itemize}

\pause

\item $SE_{(\hat{p}_1 - \hat{p}_2)} = \sqrt{ \frac{p_1(1-p_1)}{n_1} + \frac{p_2(1-p_2)}{n_2} }$
\begin{itemize}
\item for CI: use $\hat{p}_1$ and $\hat{p}_2$
\item for HT:
\begin{itemize}
\item when $H_0: p_1 = p_2$: use $\hat{p}_{pool} = \frac{\#~suc_1 + \#suc_2}{n_1 + n_2}$
\item when $H_0: p_1 - p_2 = $ \textit{(some value other than 0)}: use $\hat{p}_1$ and $\hat{p}_2$ \\
- this is pretty rare
\end{itemize}
\end{itemize}

\end{itemize}

\end{frame}

%%%%%%%%%%%%%%%%%%%%%%%%%%%%%%%%%%%

% \begin{frame}
% \frametitle{Reference - standard error calculations}

% \begin{center}
% \begin{tabular}{l | l | l}
% 			& one sample					& two samples \\ 
% \hline
% & & \\
% mean		& $SE = \frac{s}{\sqrt{n}}$			& $SE = \sqrt{ \frac{s_1^2}{n_1} + \frac{s_2^2}{n_2}}$ \\
% & & \\
% \hline
% & & \\
% proportion		& $SE = \sqrt{ \frac{p(1-p)}{n} }$	& $SE = \sqrt{ \frac{p_1(1-p_1)}{n_1} + \frac{p_2(1-p_2)}{n_2} }$	 \\	
% & & \\
% \end{tabular}
% \end{center}

% \pause

% \begin{itemize}

% \item When working with means, it's very rare that $\sigma$ is known, so we usually use $s$.

% \pause

% \item When working with proportions, 
% \begin{itemize}
% \item if doing a hypothesis test, $p$ comes from the null hypothesis
% \item if constructing a confidence interval, use $\hat{p}$ instead
% \end{itemize}

% \end{itemize}

% \end{frame}


%%%%%%%%%%%%%%%%%%%%%%%%%%%%%%%%%%%%
% End document
%%%%%%%%%%%%%%%%%%%%%%%%%%%%%%%%%%%%

\end{document}