\documentclass[12pt]{article}

\usepackage{amsmath,amsthm,amssymb,xcolor}

\begin{document}

%%%%%%%%%%%%%%%%%%%%%%%%%%%%%%%%%%%%%%%%%%

\title{MA 213 Learning Objectives}
\date{}
\maketitle

%%%%%%%%%%%%%%%%%%%%%%%%

% \section{Global Module}

% %%%%%%%%%%%%%%%%%%%%%%%%

% \begin{enumerate}
%     \item I can carry out a complete, reproducible statistical workflow in R (load data, run exploratory analyses, transform variables, run statistical analyses, display and interpret results) using the inference methods from the course with a team.
%     \item Given R code for a statistical analysis, I can explain what it does and why (in terms of the methods from the course), and identify both programmatic and statistical errors.
%     \item When solving probability and statistics problems, I can support my answers by writing out the steps using the notation and conventions of statistical exposition.
%     \item I can recognize whether a statistical workflow is appropriate for the given data and data analysis goals, and explain the results of a statistical analysis to stakeholders:
    
%     \begin{enumerate}
%         \item as a presentation, and 
%         \item in a written report.
%     \end{enumerate}    
% \end{enumerate}

% In addition to the global module, each module of the course has its own learning objectives that will be assessed throughout the semester.

%%%%%%%%%%%%%%%%%%%%%%%%

\section{Module 1: Exploratory Data Analysis and Study Design}
(Chapters 1 and 2; 5 objectives)

%%%%%%%%%%%%%%%%%%%%%%%%

\begin{enumerate}
    \item \textbf{Classify and Analyze Variables:} Categorize variables based on their types (e.g., numerical/categorical, continuous/discrete, ordinal), assess their association (positive, negative, or independent), and determine which make sense as explanatory vs. response variables. [Q1, L2] \textcolor{red}{Core}
    \item \textbf{Evaluate Study Design and Its Implications:} Identify and explain experimental design choices (observational vs. experimental, sampling methods, blinding, potential biases), and judge whether results can be generalized to a population or used to infer causation. [Q1, P1] \textcolor{red}{Core}
    \item \textbf{Use R for Data Management and Exploration:} Utilize R to load, pre-process, and explore data through visualization and summarization techniques. [Q1, L1, L2] \textcolor{red}{Core}
    \item \textbf{Visualize and Describe Data Distributions:} Select appropriate visualizations (scatterplots, histograms, box plots, bar plots) to depict data, and describe distributions qualitatively (shape, center, spread, outliers) and quantitatively (mean, median, mode, range, IQR, standard deviation). [Q1, P1] \textcolor{red}{Core}
    \item \textbf{Conduct Hypothesis Testing Using Simulation:} Set up null and alternative hypotheses to test for independence between variables, and use simulation techniques to evaluate data support for these hypotheses. [Q1, L3] \textcolor{blue}{Auxiliary}
\end{enumerate}

\newpage

%%%%%%%%%%%%%%%%%%%%%%%%

\section{Module 2: Probability, Random Variables, and Distributions}
(Chapters 3 and 4; 6 objectives)

%%%%%%%%%%%%%%%%%%%%%%%%

\begin{enumerate}
    \item \textbf{Validate and Explain Probability Distributions:} Assess the validity of a probability distribution using the concepts of outcome, sample space, and probability properties (e.g., disjoint outcomes, probabilities between 0 and 1, and total probabilities summing to 1). [Q2, L3]  \textcolor{red}{Core}
    \item \textbf{Apply the Law of Large Numbers and Its Implications:} Explain the Law of Large Numbers, why it holds, and its implications for predicting long-term averages in probability and statistics. [Q2, L4] \textcolor{red}{Core}
    \item \textbf{Compute Probabilities Using Various Tools:} Use logic, Venn diagrams, and probability rules to compute probabilities for events. [Q2, L3] \textcolor{red}{Core}
    \item \textbf{Understand and Compute Expectations and Variances:} Explain the concepts of expectations and variances of random variables, and compute the expectation and variance of a linear combination of random variables. [Q2, L3] \textcolor{red}{Core} 
    \item \textbf{Model Data Using Bernoulli, Geometric, and Binomial Distributions:} Recognize when to appropriately model data using the Bernoulli, geometric, and binomial distributions, and compute quantities of interest such as mean, standard deviation, and tail probabilities. [Q2, L4] \textcolor{red}{Core}
    \item \textbf{Assess Data Using the Normal Distribution:} Use the normal distribution to assess the "unusualness" of data points, apply the 68-95-99.7\% rule, evaluate normality through histograms and q-q plots, and determine when a normal approximation to the binomial model is valid for calculating binomial probabilities. [Q2, L4] \textcolor{red}{Core}
\end{enumerate}

\newpage

%%%%%%%%%%%%%%%%%%%%%%%%

\section{Module 3: Foundations for Inference}
(Chapter 5; 6 objectives)

%%%%%%%%%%%%%%%%%%%%%%%%

\begin{enumerate}
    \item \textbf{Understand Point Estimates and Sampling Variability:} Define a sample statistic (point estimate) for a population parameter, and explain how it varies across different samples. [Q3, L4] \textcolor{red}{Core} 
    \item \textbf{Visualize and Interpret Sampling Distributions:} Draw and interpret sampling distributions for a point estimate (e.g., population proportion) across different sample sizes, explaining how the distribution changes as the sample size increases. [Q3, L4] \textcolor{red}{Core}
    \item \textbf{Calculate and Interpret Standard Error:} Calculate the standard error for proportions and interpret it as a measure of sampling variability. [Q3, L4] \textcolor{red}{Core}
    \item \textbf{Explain Hypothesis Testing and Its Limitations:} Discuss the use cases and potential issues with hypothesis testing, including the interpretation of results. [Q3] \textcolor{red}{Core}
    \item \textbf{Understand Errors and Significance Levels:} Identify Type I and Type II errors and explain how they are influenced by changes in the significance level. [Q3, L5] \textcolor{red}{Core}
    \item \textbf{Distinguish Statistical vs. Practical Significance:} Differentiate between statistical significance and practical significance, and explain the implications of each. [Q3] \textcolor{red}{Core}
\end{enumerate}

\newpage

%%%%%%%%%%%%%%%%%%%%%%%%

\section{Module 4: Inference}
(Chapters 6 and 7; 8 objectives)

%%%%%%%%%%%%%%%%%%%%%%%%

\begin{enumerate}
    \item \textbf{Calculate Sample Size for Confidence Intervals:} Calculate the required minimum sample size for a given margin of error and confidence level. [Q4, L5] \textcolor{blue}{Auxiliary}
    \item \textbf{Design and Interpret Confidence Intervals:} Design, execute, and interpret confidence intervals for the population proportion. [Q4, L5] \textcolor{red}{Core}
    \item \textbf{Conduct and Interpret Hypothesis Tests for Proportions:} Design, execute, and interpret hypothesis tests for population proportions. [Q4, L5] \textcolor{red}{Core}
    \item \textbf{Conduct and Interpret Chi-Square Tests:} Assess whether the conditions for a chi-square test (goodness of fit or independence) are met, and if so, design, execute, and interpret the test. [Q4, L5] \textcolor{blue}{Auxiliary}
    \item \textbf{Explain and Use the t-Distribution:} Explain how the t-distribution differs from the normal distribution and why it is used for population mean inference. [Q4, L5] \textcolor{blue}{Auxiliary}
    \item \textbf{Conduct and Interpret t-Tests:} Design, execute, and interpret t-tests for a single population mean, a difference of paired means, and a difference of independent means, calculating the standard error appropriately for each. Describe how to obtain a p-value for a t-test and a critical t-score for a confidence interval. [Q4, L5] \textcolor{red}{Core}
    \item \textbf{Calculate Test Power and Evaluate Factors:} Calculate the power of a test for a given effect size and significance level, and explain how the power would change with variations in effect size, sample size, significance level, or standard error. [Q4, L5] \textcolor{blue}{Auxiliary}
    \item \textbf{Conduct and Interpret ANOVA:} Assess whether conditions for an ANOVA are met, and if so, design, execute, and interpret the test to compare sample means across several groups. [Q4, L5] \textcolor{blue}{Auxiliary} 
\end{enumerate}

\newpage

%%%%%%%%%%%%%%%%%%%%%%%%

\section{Module 5: Linear Regression}
(Chapter 8; 5 objectives)

%%%%%%%%%%%%%%%%%%%%%%%%

\begin{enumerate}
    \item \textbf{Describe and Assess Relationships Between Two Variables:} Describe the association between two numerical variables in a scatter plot in terms of direction, shape (linear or nonlinear), and strength, and assess whether linear regression is an appropriate model. [Q5, L6] \textcolor{blue}{Auxiliary} 
    \item \textbf{Compute and Interpret Correlation and R$^2$} Compute and interpret correlation coefficients and R$^2$ values, while recognizing that correlation does not imply causation. [Q5, L6] \textcolor{blue}{Auxiliary}
    \item \textbf{Fit and Interpret Linear Models Using Least Squares:} Fit the intercept and slope of a linear model to data using the least squares method, interpret the fitted values, and use the model to predict responses to new inputs. [Q5, L6] \textcolor{blue}{Auxiliary}
    \item \textbf{Explain and Assess Model Fit:} Explain the least squares fitting procedure, assess whether the fit is unduly influenced by any particular points, and distinguish between interpolation and extrapolation in predictions. [Q5] \textcolor{blue}{Auxiliary}
    \item \textbf{Perform Inference for Regression Coefficients:} Use fit summary and parameter estimates (e.g., $\beta_1$ and $\sigma^2$) to perform hypothesis tests or construct confidence intervals for the slope, and interpret the results. [Q5, L6] \textcolor{blue}{Auxiliary} 
\end{enumerate}

\newpage

%%%%%%%%%%%%%%%%%%%%%%%%

\section{Module 6: Global Module}
(Shared with MA 214; 4 objectives)

%%%%%%%%%%%%%%%%%%%%%%%%

\begin{enumerate}
    \item I can carry out a complete, reproducible statistical workflow in R (load data, run exploratory analyses, transform variables, run statistical analyses, display and interpret results) using the inference methods from the course. [P1, P2] \textcolor{red}{Core} 
    \item Given R code for a statistical analysis, I can explain what it does and why (in terms of the methods from the course), and identify both programmatic and statistical errors. [Q1, Q4] \textcolor{red}{Core} 
    \item When solving probability and statistics problems, I can support my answers by writing out the steps using the notation and conventions of statistical exposition. [Quizzes] \textcolor{red}{Core} 
    \item I can recognize whether a statistical workflow is appropriate for the given data and data analysis goals, and explain the results of a statistical analysis to stakeholders
    \begin{enumerate}
        \item as a presentation, and [P1] \textcolor{red}{Core} 
        \item in a written report. [P2] \textcolor{red}{Core}
    \end{enumerate}    
\end{enumerate}

%%%%%%%%%%%%%%%%%%%%%%%%%%%%%%%%%%%%%%%%%%

\end{document}