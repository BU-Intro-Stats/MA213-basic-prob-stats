
%%%%%%%%%%%%%%%%%%%%%%%%%%%%%%%%%%%%
% Recap/Agenda 
%%%%%%%%%%%%%%%%%%%%%%%%%%%%%%%%%%%%
% TODO better formatting
\begin{frame}
    \frametitle{Module 3: Foundations for inference}
    \begin{itemize}
        \item \hl{Previously: }Hypothesis testing for a proportion (Chapter 5.3)
        \item \hl{This time: }Inference for a single proportion (Chapter 6.1)
        \item \hl{Reading: }Chapter 6.2 for next time
        \item \hl{Deadlines/Announcements: }
        \begin{enumerate}
            \item HW 7 due today
            \item Quiz 2 retake qualifications due Friday (last appointments on Thursday 9-10am)
        \end{enumerate}
    \end{itemize}
    
\end{frame}

%%%%%%%%%%%%%%%%%%%%%%%%%%%%%%%%%%%%
% Learning objectives:
%%%%%%%%%%%%%%%%%%%%%%%%%%%%%%%%%%%%
\begin{frame}
    \frametitle{Learning Objectives}
    \begin{itemize}
        \item \textbf{M4, LO1: Calculate Sample Size for Confidence Intervals:} Calculate the required minimum sample size for a given margin of error and confidence level.
        \item \textbf{M4, LO2: Design and Interpret Confidence Intervals:} Design, execute, and interpret confidence intervals for the population proportion.
        \item \textbf{M4, LO3: Conduct and Interpret Hypothesis Tests for Proportions:} Design, execute, and interpret hypothesis tests for population proportions.
    \end{itemize}
\end{frame}
