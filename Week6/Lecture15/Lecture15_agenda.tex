
%%%%%%%%%%%%%%%%%%%%%%%%%%%%%%%%%%%%
% Recap/Agenda 
%%%%%%%%%%%%%%%%%%%%%%%%%%%%%%%%%%%%
% TODO better formatting
\begin{frame}
    \frametitle{Module 3: Foundations for inference}
    \begin{itemize}
        \item \hl{Previously: }Probability distributions and random variables (Chapter 4)
        \item \hl{This time: }Point estimates and sampling variability (Chapter 5.1)
        \item \hl{Reading: }Chapter 5.1 for next time
        \item \hl{Deadlines/Announcements: }
        \begin{itemize}
            \item HW 5 due today
            \item Next week: Monday 10/13 is a Holiday, Tuesday is Monday schedule (we will have lecture)
            \item Quiz 2 \hl{in class on Wednesday 10/14}, \textbf{Not in discussion sections}
            \begin{itemize}
                \item If you have accommodations through DAS, you can schedule your quiz through Accommodate
            \end{itemize}
            \item No discussion sections next week
        \end{itemize}
    \end{itemize}
    
\end{frame}

%%%%%%%%%%%%%%%%%%%%%%%%%%%%%%%%%%%%
% Learning objectives:
%%%%%%%%%%%%%%%%%%%%%%%%%%%%%%%%%%%%
\begin{frame}
\frametitle{Learning Objectives}
\begin{itemize}
    \item \textbf{M3 LO1: Understand Point Estimates and Sampling Variability:} Define a sample statistic (point estimate) for a population parameter, and explain how it varies across different samples. 
    \item \textbf{M3 LO2: Visualize and Interpret Sampling Distributions:} Draw and interpret sampling distributions for a point estimate (e.g., population proportion) across different sample sizes, explaining how the distribution changes as the sample size increases. 
\end{itemize}
\end{frame}
