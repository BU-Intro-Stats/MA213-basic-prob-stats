%%%%%%%%%%%%%%%%%%%%%%%%%%%%%%%%%%%%
% Lesson Plan (50 minutes)
%%%%%%%%%%%%%%%%%%%%%%%%%%%%%%%%%%%%
\begin{frame}
    \frametitle{Lesson Plan}
    \begin{itemize}
        \item xx min Lecture: recap inference for one proportion, motivate difference of two proportions
        \item xx min Lecture: difference of proportions
        \item xx min Edfinity quiz: experimental design, null/alternative hypotheses examples
        \item xx min Lecture: review quiz answers
        \item xx min R Demonstration: examples - how does the sample size affect inference this time?
        \item xx min Lecture: pooled proportion
    \end{itemize}
\end{frame}
            
%%%%%%%%%%%%%%%%%%%%%%%%%%%%%%%%%%%%
% Learning objectives:
%%%%%%%%%%%%%%%%%%%%%%%%%%%%%%%%%%%%
\begin{frame}
    \frametitle{Learning Objectives}
    \begin{itemize}
        \item \textbf{M1, LO3: Use R for Data Management and Exploration:} Utilize R to load, pre-process, and explore data through visualization and summarization techniques.
        \item \textbf{M3, LO3: Calculate and Interpret Standard Error:} Calculate the standard error for proportions and interpret it as a measure of sampling variability.
        \item \textbf{M4, LO1: Calculate Sample Size for Confidence Intervals:} Calculate the required minimum sample size for a given margin of error and confidence level.
        \item \textbf{M4, LO2: Design and Interpret Confidence Intervals:} Design, execute, and interpret confidence intervals for the population proportion.
        \item \textbf{M4, LO3: Conduct and Interpret Hypothesis Tests for Proportions:} Design, execute, and interpret hypothesis tests for population proportions.
    \end{itemize}
\end{frame}
    
%%%%%%%%%%%%%%%%%%%%%%%%%%%%%%%%%%%%
% TODO: Adapt drafted slides