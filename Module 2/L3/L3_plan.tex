

%%%%%%%%%%%%%%%%%%%%%%%%%%%%%%%%%%%%
% Lesson Plan (50 minutes)
%%%%%%%%%%%%%%%%%%%%%%%%%%%%%%%%%%%%
\begin{frame}
\frametitle{Lesson Plan}
\begin{itemize}
    \item xx min Purpose of Simulation (maybe visualization)
    \item xx min review of $E(X)$, $Var(X)$ and linear combination of RV
    \item xx min Excercise using R (Rmd file)
\end{itemize}
\end{frame}


%%%%%%%%%%%%%%%%%%%%%%%%%%%%%%%%%%%%
% Learning objectives:
%%%%%%%%%%%%%%%%%%%%%%%%%%%%%%%%%%%%
\begin{frame}
\frametitle{Learning Objectives}

\begin{itemize}
    \item Validate and Explain Probability Distributions: Assess the validity of a probability distribution using the concepts of outcome, sample space, and probability properties (e.g., disjoint outcomes, probabilities between 0 and 1, and total probabilities summing to 1). [Q2, L3] 
    \item Compute Probabilities Using Various Tools: Use logic, Venn diagrams, and probability rules to compute probabilities for events. [Q2, L3] 
    \item Understand and Compute Expectations and Variances: Explain the concepts of expectations and variances of random variables, and compute the expectation and variance of a linear combination of random variables. [Q2, L3] 
\end{itemize}
\end{frame}


\begin{frame}
    \frametitle{Learning Objectives Con't}

    \begin{itemize}
        \item Conduct Hypothesis Testing Using Simulation: Set up null and alternative hypotheses to test for independence between variables, and use simulation techniques to evaluate data support for these hypotheses. [Q1, L3]
        \item Simulating $E[a+bX]$, $Var[a+bX]$, $E[X+Y]$, $Var[X+Y]$ (independent) -- learning how to simulate and the relationships between sampling and the probability distributions

    \end{itemize}
\end{frame}






%%%%%%%%%%%%%%%%%%%%%%%%%%%%%%%%%%%%
% Excercise :
%%%%%%%%%%%%%%%%%%%%%%%%%%%%%%%%%%%%
\begin{frame}
    \frametitle{Excercise using R}
    \begin{enumerate}
        \item Show lists of dstn functions in R
        \item Generate random sample from different group/individual depending on the context
        \item how to obtain $E(X)$, $Var(X)$ in R? 
        \item Generate sample of $a+bX$
        \item Obtain $E(a+bX)$, $Var(a+bX)$ and compare from R and theoretical values
    \end{enumerate}
    go over from $2, \dots, 5$ for $X+Y$
\end{frame}
