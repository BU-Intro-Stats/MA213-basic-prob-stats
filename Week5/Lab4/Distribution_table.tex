\documentclass[14pt]{article}
\usepackage{booktabs}    % For enhanced table formatting
\usepackage{array}       % For improved array and tabular environments
\usepackage{geometry}    % To adjust page margins
\usepackage{multirow}    % For multi-row cells
\usepackage{caption}     % For customizing captions

\geometry{a4paper, margin=1in}

\begin{document}

\title{Distribution of the Sum of Two Dice}
\author{}
\date{}
\maketitle

\begin{table}[htbp]
\centering
\large
\setlength{\tabcolsep}{14pt}  % Increase column separation
\caption{Outcomes, Number of Ways, and Probabilities for the Sum of Two Six-Sided Dice (Fill in the Blanks)}
\label{table:sum_two_dice_empty}
\begin{tabular}{@{}l *{11}{c@{\hspace{1cm}}} @{}} 
\toprule
\textbf{Category} & \textbf{2} & \textbf{3} & \textbf{4} & \textbf{5} & \textbf{6} & \textbf{7} & \textbf{8} & \textbf{9} & \textbf{10} & \textbf{11} & \textbf{12} \\ \midrule
\textbf{\# of Ways} & & & & & & & & & & & \\
 & & & & & & & & & & & \\ \\ \\ 
 & & & & & & & & & & & \\ \\ \\
 & & & & & & & & & & & \\ \\ \\
 & & & & & & & & & & & \\ \\ \\
 & & & & & & & & & & & \\ \midrule
\textbf{$P(X_1 + X_2)$} & & & & & & & & & & & \\ \bottomrule
\end{tabular}
\end{table}

               

% \newpage

% % \begin{table}[h!]
% % \centering
% % \caption{Outcomes, Number of Ways, and Probabilities for the Sum of Two Six-Sided Dice}
% % \label{table:sum_two_dice}
% % \begin{tabular}{@{}l *{11}{c} @{}} % 1 + 11 columns
% % \toprule
% %  & \textbf{2} & \textbf{3} & \textbf{4} & \textbf{5} & \textbf{6} & \textbf{7} & \textbf{8} & \textbf{9} & \textbf{10} & \textbf{11} & \textbf{12} \\ \midrule
% % \textbf{\# of Ways} & (1,1) & (1,2) & (1,3) & (1,4) & (1,5) & (1,6) & (2,6) & (3,6) & (4,6) & (5,6) & (6,6) \\
% %  & & (2,1) & (2,2) & (2,3) & (2,4) & (2,5) & (2,4) & (3,5) & (4,5) & (5,5) & \\
% %  & & & (3,1) & (3,2) & (3,3) & (3,4) & (3,3) & (3,4) & (3,3) & & \\
% %  & & & & (4,1) & (4,2) & (4,3) & (4,2) & (4,3) & & & \\
% %  & & & & & (5,1) & (5,2) & (5,3) & & & \\
% %  & & & & & & (6,1) & (6,2) & & & \\ \midrule
% % \textbf{$P(X_1 + X_2)$} & $\frac{1}{36}$ & $\frac{2}{36}$ & $\frac{3}{36}$ & $\frac{4}{36}$ & $\frac{5}{36}$ & $\frac{6}{36}$ & $\frac{5}{36}$ & $\frac{4}{36}$ & $\frac{3}{36}$ & $\frac{2}{36}$ & $\frac{1}{36}$ \\ \bottomrule
% % \end{tabular}
% % \end{table}

% % $$
% % \begin{aligned}
% % E(X_1 + X_2) &= 2 \times \frac{1}{36} + 3 \times \frac{2}{36} + 4 \times \frac{3}{36} + \dots + 12 \times \frac{1}{36} \\
% %                &= \mu = 7 
% % \end{aligned}
% % $$

% % $$
% % \begin{align}
% %     Var(X_1+X_2) &= (2-7)^2 \frac{1}{36} + (3-7)^2 \frac{2}{36} + \dots + (12-7)^2 \frac{1}{36} \\
% %                 &= \sigma^2 = 5.83
% % \end{align}
% % $$

\end{document}