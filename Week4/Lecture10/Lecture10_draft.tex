%%%%%%%%%%%%%%%%%%%%%%%%%%%%%%%%%%%%
% Slide options
%%%%%%%%%%%%%%%%%%%%%%%%%%%%%%%%%%%%

% Option 1: Slides with solutions

\documentclass[slidestop,compress,mathserif]{beamer}
\newcommand{\soln}[1]{\textit{#1}}
\newcommand{\solnGr}[1]{#1}

% Option 2: Handouts without solutions

%\documentclass[11pt,containsverbatim,handout]{beamer}
%\usepackage{pgfpages}
%\pgfpagesuselayout{4 on 1}[letterpaper,landscape,border shrink=5mm]
%\newcommand{\soln}[1]{ }
%\newcommand{\solnGr}{ }


%%%%%%%%%%%%%%%%%%%%%%%%%%%%%%%%%%%%
% Style
%%%%%%%%%%%%%%%%%%%%%%%%%%%%%%%%%%%%

\input{../../lec_style.tex}


%%%%%%%%%%%%%%%%%%%%%%%%%%%%%%%%%%%%
% Preamble
%%%%%%%%%%%%%%%%%%%%%%%%%%%%%%%%%%%%

\title[Chp 4: Distributions of RVs]{Chapter 4: Distributions of random variables}
\author{OpenIntro Statistics, 4th Edition}
\institute{$\:$ \\ {\footnotesize Slides developed by Mine \c{C}etinkaya-Rundel of OpenIntro. \\
The slides may be copied, edited, and/or shared via the \webLink{http://creativecommons.org/licenses/by-sa/3.0/us/}{CC BY-SA license.} \\
Some images may be included under fair use guidelines (educational purposes).}}
\date{}

%%%%%%%%%%%%%%%%%%%%%%%%%%%%%%%%%%%%
% Begin document
%%%%%%%%%%%%%%%%%%%%%%%%%%%%%%%%%%%%

\begin{document}


%%%%%%%%%%%%%%%%%%%%%%%%%%%%%%%%%%%%
% Title page
%%%%%%%%%%%%%%%%%%%%%%%%%%%%%%%%%%%%

{
\addtocounter{framenumber}{-1} 
{\removepagenumbers 
\usebackgroundtemplate{\includegraphics[width=\paperwidth]{../../OpenIntro_Grid_4_3-01.jpg}}
\begin{frame}

\hfill \includegraphics[width=20mm]{../../oiLogo_highres}

\titlepage

\end{frame}
}
}


%%%%%%%%%%%%%%%%%%%%%%%%%%%%%%%%%%%%
% Sections
%%%%%%%%%%%%%%%%%%%%%%%%%%%%%%%%%%%%

%%%%%%%%%%%%%%%%%%%%%%%%%%%%%%%%%%%%

\section{Normal distribution}

%%%%%%%%%%%%%%%%%%%%%%%%%%%%%%%%%%%%

\begin{frame}
\frametitle{Normal distribution}

\begin{itemize}

\item Unimodal and symmetric, bell shaped curve

\item Many variables are nearly normal, but none are exactly normal

\item Denoted as \mathhl{N(\mu,\sigma)} $\rightarrow$ Normal with mean $\mu$ and standard deviation $\sigma$

\end{itemize}

\begin{center}
%includegraphics[width=0.7\textwidth]{4-1_normal_distribution/figures/simpleNormal/simpleNormal}
\end{center}

\end{frame}

%%%%%%%%%%%%%%%%%%%%%%%%%%%%%%%%%%%%

\begin{frame}
\frametitle{Heights of males}

\twocol{0.5}{0.5}{
%includegraphics[width=\textwidth]{4-1_normal_distribution/figures/ok_cupid/ok_cupid_men} \\
}
{
\pause
{\footnotesize``The male heights on OkCupid very nearly follow the expected normal distribution -- except the whole thing is shifted to the right of where it should be. Almost universally guys like to add a couple inches." 

``You can also see a more subtle vanity at work: starting at roughly 5' 8", the top of the dotted curve tilts even further rightward. This means that guys as they get closer to six feet round up a bit more than usual, stretching for that coveted psychological benchmark."
}
}

\ct{\webURL{http://blog.okcupid.com/index.php/the-biggest-lies-in-online-dating/}}

\end{frame}

%%%%%%%%%%%%%%%%%%%%%%%%%%%%%%%%%%%%

\begin{frame}
\frametitle{Heights of females}

\twocol{0.5}{0.5}{
%includegraphics[width=\textwidth]{4-1_normal_distribution/figures/ok_cupid/ok_cupid_women} \\
}
{
\pause
{\footnotesize ``When we looked into the data for women, we were surprised to see height exaggeration was just as widespread, though without the lurch towards a benchmark height."
}
}

\vfill

\ct{\webURL{http://blog.okcupid.com/index.php/the-biggest-lies-in-online-dating/}}

\end{frame}

%%%%%%%%%%%%%%%%%%%%%%%%%%%%%%%%%%%%

\subsection{Normal distribution model}

%%%%%%%%%%%%%%%%%%%%%%%%%%%%%%%%%%%%

% \begin{frame}
% \frametitle{Normal distributions with different parameters}

% \vspace{-0.5cm}
% \begin{center}
% $\mu$: mean, $\sigma$: standard deviation
% \[N(\mu = 0, \sigma = 1) \hspace{1.4cm} N(\mu = 19, \sigma = 4) \]
% %includegraphics[width=0.6\textwidth]{4-1_normal_distribution/figures/twoSampleNormals/twoSampleNormals} \\
% %includegraphics[width=0.6\textwidth]{4-1_normal_distribution/figures/twoSampleNormalsStacked/twoSampleNormalsStacked}
% \end{center}

% \end{frame}

%%%%%%%%%%%%%%%%%%%%%%%%%%%%%%%%%%%%

\section{R Demonstration: Normal distributions with different parameters}

%%%%%%%%%%%%%%%%%%%%%%%%%%%%%%%%%%%%

\subsection{Standardizing with Z scores}

%%%%%%%%%%%%%%%%%%%%%%%%%%%%%%%%%%%%

\begin{frame}
\frametitle{}

\dq{SAT scores are distributed nearly normally with mean 1500 and standard deviation 300. ACT scores are distributed nearly normally with mean 21 and standard deviation 5. A college admissions officer wants to determine which of the two applicants scored better on their standardized test with respect to the other test takers: Pam, who earned an 1800 on her SAT, or Jim, who scored a 24 on his ACT?}

\begin{center}
%includegraphics[width=\textwidth]{4-1_normal_distribution/figures/satActNormals/satActNormals}
\end{center}

\end{frame}

%%%%%%%%%%%%%%%%%%%%%%%%%%%%%%%%%%%%

\begin{frame}
\frametitle{Standardizing with Z scores}

Since we cannot just compare these two raw scores, we instead compare how many standard deviations beyond the mean each observation is.

\begin{itemize}

\item Pam's score is $\frac{1800 - 1500}{300} = 1$ standard deviation above the mean.

\item Jim's score is $\frac{24 - 21}{5} = 0.6$ standard deviations above the mean.

\end{itemize}

\begin{center}
%includegraphics[width=0.7\textwidth]{4-1_normal_distribution/figures/satActNormals/satActNormalsStandardized}
\end{center}

\end{frame}

%%%%%%%%%%%%%%%%%%%%%%%%%%%%%%%%%%%%

\begin{frame}
\frametitle{Standardizing with Z scores (cont.)}

\begin{itemize}

\item These are called \hl{standardized} scores, or \hl{Z scores}.

\item Z score of an observation is the number of standard deviations it falls above or below the mean.
\formula{\[Z = \frac{observation - mean}{SD}\]}

\item Z scores are defined for distributions of any shape, but only when the distribution is normal can we use Z scores to calculate percentiles.

\item Observations that are more than 2 SD away from the mean ($|Z| > 2$) are usually considered unusual.

\end{itemize}

\end{frame}

%%%%%%%%%%%%%%%%%%%%%%%%%%%%%%%%%%%%

\begin{frame}
\frametitle{Percentiles}

\begin{itemize}

\item \hl{Percentile} is the percentage of observations that fall below a given data point. 

\item Graphically, percentile is the area below the probability distribution curve to the left of that observation.

\end{itemize}

\begin{center}
%includegraphics[width=0.7\textwidth]{4-1_normal_distribution/figures/satBelow1800/satBelow1800}
\end{center}

\end{frame}

\subsection{Finding tail areas}

%%%%%%%%%%%%%%%%%%%%%%%%%%%%%%%%%%%%

\begin{frame}[fragile]
\frametitle{Calculating percentiles - using computation}

There are many ways to compute percentiles/areas under the curve:

\begin{itemize}
\item R:
\end{itemize}
\begin{beamerboxesrounded}[shadow = false, lower = code body]{}
{\small \begin{verbatim}
> pnorm(1800, mean = 1500, sd = 300)
[1] 0.8413447
\end{verbatim}
}
\end{beamerboxesrounded}
\begin{itemize}
\item Applet: {\small \webURL{https://gallery.shinyapps.io/dist_calc/}}
\begin{center}
%includegraphics[width=0.8\textwidth]{4-1_normal_distribution/figures/applet}
\end{center}

\end{itemize}


\end{frame}

%%%%%%%%%%%%%%%%%%%%%%%%%%%%%%%%%%%%

\begin{frame}[fragile]
\frametitle{Calculating percentiles - using tables}

{\footnotesize
\begin{tabular}{c | >{\columncolor[gray]{0.6}[0pt]}rrrrr | rrrrr |}
  \cline{2-11}
&&&& \multicolumn{4}{c}{Second decimal place of $Z$} &&& \\
  \cline{2-11}
$Z$ & 0.00 & 0.01 & 0.02 & 0.03 & 0.04 & 0.05 & 0.06 & 0.07 & 0.08 & 0.09 \\
  \hline
  \hline
0.0 & \tiny{0.5000} & \tiny{0.5040} & \tiny{0.5080} & \tiny{0.5120} & \tiny{0.5160} & \tiny{0.5199} & \tiny{0.5239} & \tiny{0.5279} & \tiny{0.5319} & \tiny{0.5359} \\
  0.1 & \tiny{0.5398} & \tiny{0.5438} & \tiny{0.5478} & \tiny{0.5517} & \tiny{0.5557} & \tiny{0.5596} & \tiny{0.5636} & \tiny{0.5675} & \tiny{0.5714} & \tiny{0.5753} \\
  0.2 & \tiny{0.5793} & \tiny{0.5832} & \tiny{0.5871} & \tiny{0.5910} & \tiny{0.5948} & \tiny{0.5987} & \tiny{0.6026} & \tiny{0.6064} & \tiny{0.6103} & \tiny{0.6141} \\
  0.3 & \tiny{0.6179} & \tiny{0.6217} & \tiny{0.6255} & \tiny{0.6293} & \tiny{0.6331} & \tiny{0.6368} & \tiny{0.6406} & \tiny{0.6443} & \tiny{0.6480} & \tiny{0.6517} \\
  0.4 & \tiny{0.6554} & \tiny{0.6591} & \tiny{0.6628} & \tiny{0.6664} & \tiny{0.6700} & \tiny{0.6736} & \tiny{0.6772} & \tiny{0.6808} & \tiny{0.6844} & \tiny{0.6879} \\
  \hline
  0.5 & \tiny{0.6915} & \tiny{0.6950} & \tiny{0.6985} & \tiny{0.7019} & \tiny{0.7054} & \tiny{0.7088} & \tiny{0.7123} & \tiny{0.7157} & \tiny{0.7190} & \tiny{0.7224} \\
  0.6 & \tiny{0.7257} & \tiny{0.7291} & \tiny{0.7324} & \tiny{0.7357} & \tiny{0.7389} & \tiny{0.7422} & \tiny{0.7454} & \tiny{0.7486} & \tiny{0.7517} & \tiny{0.7549} \\
  0.7 & \tiny{0.7580} & \tiny{0.7611} & \tiny{0.7642} & \tiny{0.7673} & \tiny{0.7704} & \tiny{0.7734} & \tiny{0.7764} & \tiny{0.7794} & \tiny{0.7823} & \tiny{0.7852} \\
  0.8 & \tiny{0.7881} & \tiny{0.7910} & \tiny{0.7939} & \tiny{0.7967} & \tiny{0.7995} & \tiny{0.8023} & \tiny{0.8051} & \tiny{0.8078} & \tiny{0.8106} & \tiny{0.8133} \\
  0.9 & \tiny{0.8159} & \tiny{0.8186} & \tiny{0.8212} & \tiny{0.8238} & \tiny{0.8264} & \tiny{0.8289} & \tiny{0.8315} & \tiny{0.8340} & \tiny{0.8365} & \tiny{0.8389} \\
  \hline
  \hline
\rowcolor[gray]{.6}
  1.0 & \orange{\tiny{0.8413}} & \tiny{0.8438} & \tiny{0.8461} & \tiny{0.8485} & \tiny{0.8508} & \tiny{0.8531} & \tiny{0.8554} & \tiny{0.8577} & \tiny{0.8599} & \tiny{0.8621} \\
  1.1 & \tiny{0.8643} & \tiny{0.8665} & \tiny{0.8686} & \tiny{0.8708} & \tiny{0.8729} & \tiny{0.8749} & \tiny{0.8770} & \tiny{0.8790} & \tiny{0.8810} & \tiny{0.8830} \\
  1.2 & \tiny{0.8849} & \tiny{0.8869} & \tiny{0.8888} & \tiny{0.8907} & \tiny{0.8925} & \tiny{0.8944} & \tiny{0.8962} & \tiny{0.8980} & \tiny{0.8997} & \tiny{0.9015} \\
\end{tabular}
}

\end{frame}

%%%%%%%%%%%%%%%%%%%%%%%%%%%%%%%%%%%%

\subsection{Normal probability examples}

%%%%%%%%%%%%%%%%%%%%%%%%%%%%%%%%%%%%

\begin{frame}
\frametitle{Six sigma}

``The term \textit{six sigma process} comes from the notion that if one has six standard deviations between the process mean and the nearest specification limit, as shown in the graph, practically no items will fail to meet specifications."

\begin{center}
%includegraphics[width=0.35\textwidth]{4-1_normal_distribution/figures/sixsigma}
\end{center}

\ct{\webURL{http://en.wikipedia.org/wiki/Six_Sigma}}

\end{frame}

%%%%%%%%%%%%%%%%%%%%%%%%%%%%%%%%%%%%

\begin{frame}
\frametitle{Quality control}

\dq{{\small At Heinz ketchup factory the amounts which go into bottles of ketchup are supposed to be normally distributed with mean 36 oz. and standard deviation 0.11 oz. Once every 30 minutes a bottle is selected from the production line, and its contents are noted precisely. If the amount of ketchup in the bottle is below 35.8 oz. or above 36.2 oz., then the bottle fails the quality control inspection. What percent of bottles have less than 35.8 ounces of ketchup?}}

\soln{\pause
Let $X$ = amount of ketchup in a bottle: $X \sim N(\mu = 36, \sigma = 0.11)$ \\
\pause
\twocol{0.4}{0.6}{
\begin{center}
%includegraphics[width=\textwidth]{4-1_normal_distribution/figures/ketchup/ketchupLT358}
\end{center}
}
{
\pause
\[ Z = \frac{35.8 - 36}{0.11} = -1.82 \]
}
}

\end{frame}

%%%%%%%%%%%%%%%%%%%%%%%%%%%%%%%%%%%%

\section{Edfinity Quiz: Z scores, percentiles}

%%%%%%%%%%%%%%%%%%%%%%%%%%%%%%%%%%%%

\section{R Demonstration: Computing probabilities}

%%%%%%%%%%%%%%%%%%%%%%%%%%%%%%%%%%%%

% \begin{frame}[fragile]
% \frametitle{Finding the exact probability - using R}

% \begin{beamerboxesrounded}[shadow = true, lower = code body]{}
% {\small \begin{verbatim}
% > pnorm(-1.82, mean = 0, sd = 1)
% [1] 0.0344
% \end{verbatim}
% }
% \end{beamerboxesrounded}

% $\:$ \\
% \pause

% OR

% $\:$ \\
% \pause

% \begin{beamerboxesrounded}[shadow = true, lower = code body]{}
% {\small \begin{verbatim}
% > pnorm(35.8, mean = 36, sd = 0.11)
% [1] 0.0345
% \end{verbatim}
% }
% \end{beamerboxesrounded}


% \end{frame}

%%%%%%%%%%%%%%%%%%%%%%%%%%%%%%%%%%%%%

% \begin{frame}
% \frametitle{Practice}

% \pq{What percent of bottles \underline{pass} the quality control inspection?}

% \vspace{-0.5cm}
% \begin{multicols}{2}
% \begin{enumerate}[(a)]
% \item 1.82\%
% \item 3.44\%
% \item 6.88\%
% \solnMult{93.12\%}
% \item 96.56\%
% \item[]
% \end{enumerate}
% \end{multicols}

% \soln{
% \vspace{-0.5cm}
% \pause
% \begin{columns}[c]
% \column{0.3\textwidth}
% \pause
% %includegraphics[width=\textwidth]{4-1_normal_distribution/figures/ketchup/ketchupBET}
% \column{0.05\textwidth}
% =
% \pause
% \column{0.3\textwidth}
% %includegraphics[width=\textwidth]{4-1_normal_distribution/figures/ketchup/ketchupLT362}
% \column{0.05\textwidth}
% -
% \pause
% \column{0.3\textwidth}
% %includegraphics[width=\textwidth]{4-1_normal_distribution/figures/ketchup/ketchupLT358}
% \end{columns}
% \pause
% \begin{eqnarray*}
% Z_{35.8} &=& \frac{35.8 - 36}{0.11} = -1.82 \\ \pause
% Z_{36.2} &=& \frac{36.2 - 36}{0.11} = 1.82 \\ \pause
% P(35.8 < X < 36.2) &=& P(-1.82 < Z < 1.82) = 0.9656 - 0.0344 = 0.9312
% \end{eqnarray*}
% }

% \end{frame}

%%%%%%%%%%%%%%%%%%%%%%%%%%%%%%%%%%%%

\section{Edfinity Quiz: Cutoff points}

%%%%%%%%%%%%%%%%%%%%%%%%%%%%%%%%%%%%

% \begin{frame}[fragile]
% \frametitle{Finding cutoff points}

% \dq{Body temperatures of healthy humans are distributed nearly normally with mean 98.2$\degree$F and standard deviation 0.73$\degree$F. What is the cutoff for the lowest 3\% of human body temperatures?}

% \pause

% \twocol{0.3}{0.7}
% {
% %includegraphics[width=\textwidth]{4-1_normal_distribution/figures/temp/tempLOW3PERC}
% }
% {
% \pause
% \begin{eqnarray*}
% P(X < x) &=& 0.03 \rightarrow P(Z < \orange{-1.88}) = 0.03 \\ \pause
% Z &=& \frac{obs~-~mean}{SD} \rightarrow \frac{x - 98.2}{0.73} = -1.88 \\ \pause
% x &=& (-1.88 \times 0.73) + 98.2 = 96.8\degree F
% \end{eqnarray*}
% }
% $\:$ \\
% \begin{beamerboxesrounded}[shadow = true, lower = code body]{}
% {\small \begin{verbatim}
% > qnorm(0.03)
% [1] -1.880794
% \end{verbatim}
% }
% \end{beamerboxesrounded}

% \ct{Mackowiak, Wasserman, and Levine (1992), \textit{A Critical Appraisal of 98.6 Degrees F, the Upper Limit of the Normal Body Temperature, and Other Legacies of Carl Reinhold August Wunderlick}.}

% \end{frame}

%%%%%%%%%%%%%%%%%%%%%%%%%%%%%%%%%%%

% \begin{frame}[fragile]
% \frametitle{Practice}

% \pq{Body temperatures of healthy humans are distributed nearly normally with mean 98.2$\degree$F and standard deviation 0.73$\degree$F. What is the cutoff for the highest 10\% of human body temperatures?}

% \vspace{-0.5cm}
% \begin{multicols}{2}
% \begin{enumerate}[(a)]
% \item 97.3$\degree$F
% \solnMult{99.1$\degree$F}
% \item 99.4$\degree$F
% \item 99.6$\degree$F
% \end{enumerate}
% \end{multicols}

% \soln{
% \vspace{-0.5cm}
% \pause
% \twocol{0.3}{0.7}
% {
% %includegraphics[width=\textwidth]{4-1_normal_distribution/figures/temp/tempHIGH10PERC}
% }
% {
% \pause
% \begin{eqnarray*}
% P(X > x) &=& 0.10 \rightarrow P(Z < \orange{1.28}) = 0.90 \\ \pause
% Z &=& \frac{obs~-~mean}{SD} \rightarrow \frac{x - 98.2}{0.73} = 1.28 \\ \pause
% x &=& (1.28 \times 0.73) + 98.2 = 99.1
% \end{eqnarray*}
% }
% }

% \end{frame}

%%%%%%%%%%%%%%%%%%%%%%%%%%%%%%%%%%%%
% End document
%%%%%%%%%%%%%%%%%%%%%%%%%%%%%%%%%%%%

\end{document}