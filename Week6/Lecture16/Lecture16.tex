%%%%%%%%%%%%%%%%%%%%%%%%%%%%%%%%%%%%
% Slide options
%%%%%%%%%%%%%%%%%%%%%%%%%%%%%%%%%%%%

% Option 1: Slides with solutions

\documentclass[t,compress,mathserif]{beamer}
\newcommand{\soln}[1]{\textit{#1}}
\newcommand{\solnGr}[1]{#1}

% Option 2: Handouts without solutions

%\documentclass[11pt,containsverbatim,handout]{beamer}
%\usepackage{pgfpages}
%\pgfpagesuselayout{4 on 1}[letterpaper,landscape,border shrink=5mm]
%\newcommand{\soln}[1]{ }
%\newcommand{\solnGr}{ }


%%%%%%%%%%%%%%%%%%%%%%%%%%%%%%%%%%%%
% Style
%%%%%%%%%%%%%%%%%%%%%%%%%%%%%%%%%%%%

\def\chp5@path{../../Chp 5}
\input{../../lec_style.tex}

%%%%%%%%%%%%%%%%%%%%%%%%%%%%%%%%%%%%
% Preamble
%%%%%%%%%%%%%%%%%%%%%%%%%%%%%%%%%%%%

\title[Lecture 16]{MA213: Lecture 16}
\subtitle{Module 3: Foundations for inference}
\author{OpenIntro Statistics, 4th Edition}
\institute{$\:$ \\ {\footnotesize Based on slides developed by Mine \c{C}etinkaya-Rundel of OpenIntro. \\
The slides may be copied, edited, and/or shared via the \webLink{http://creativecommons.org/licenses/by-sa/3.0/us/}{CC BY-SA license.} \\
Some images may be included under fair use guidelines (educational purposes).}}
\date{}


%%%%%%%%%%%%%%%%%%%%%%%%%%%%%%%%%%%%
% Begin document
%%%%%%%%%%%%%%%%%%%%%%%%%%%%%%%%%%%%

\begin{document}

%%%%%%%%%%%%%%%%%%%%%%%%%%%%%%%%%%%%
% Title page
%%%%%%%%%%%%%%%%%%%%%%%%%%%%%%%%%%%%

{
\addtocounter{framenumber}{-1} 
{\removepagenumbers 
\usebackgroundtemplate{\includegraphics[width=\paperwidth]{../../OpenIntro_Grid_4_3-01.jpg}}
\begin{frame}

    \hfill \includegraphics[width=20mm]{../../oiLogo_highres}
    \titlepage

\end{frame}
}
}


%%%%%%%%%%%%%%%%%%%%%%%%%%%%%%%%%%%%
% Recap/Agenda 
%%%%%%%%%%%%%%%%%%%%%%%%%%%%%%%%%%%%
% TODO better formatting
\begin{frame}
    \frametitle{Module 3: Foundations for inference}
    \begin{itemize}
        \item \hl{Previously: }Point estimates and sampling variability (Chapter 5.1)
        \item \hl{This time: }Point estimates and sampling variability, continued (Chapter 5.1)
        \item \hl{Reading: }Chapter 5.2 for next time
        \item \hl{Deadlines/Announcements: }
        \begin{itemize}
            \item Reminder: Monday is a Holiday, Tuesday is Monday schedule
            \item Quiz 2 \hl{in class on Wednesday}, \textbf{Not in discussion sections}
            \item No discussion sections next week
        \end{itemize}
    \end{itemize}
    
\end{frame}
    
%%%%%%%%%%%%%%%%%%%%%%%%%%%%%%%%%%%%
% Learning objectives:
%%%%%%%%%%%%%%%%%%%%%%%%%%%%%%%%%%%%
\begin{frame}
\frametitle{Learning Objectives}
\begin{itemize}
    \item \textbf{M3 LO2: Visualize and Interpret Sampling Distributions:} Draw and interpret sampling distributions for a point estimate (e.g., population proportion) across different sample sizes, explaining how the distribution changes as the sample size increases. [Q3, L4]     \item \textbf{M6 LO1: Validate and Explain Probability Distributions:} Assess the validity of a probability distribution using the concepts of outcome, sample space, and probability properties (e.g., disjoint outcomes, probabilities between 0 and 1, and total probabilities summing to 1).
    \item \textbf{M3 LO3: Calculate and Interpret Standard Error:} Calculate the standard error for proportions and interpret it as a measure of sampling variability. [Q3, L4]
\end{itemize}
\end{frame}


%%%%%%%%%%%%%%%%%%%%%%%%%%%%%%%%%%%%
% Sections
%%%%%%%%%%%%%%%%%%%%%%%%%%%%%%%%%%%%

\section{Sampling distributions}

\begin{frame}
    \frametitle{}
    
    \dq{Suppose that you don't have access to the population of all American adults, which is a quite likely scenario. In order to estimate the proportion of American adults who support solar power expansion, you might sample from the population and use your sample proportion as the best guess for the unknown population proportion.}
    
    \begin{itemize}
    
    \item Sample 1000 American adults from the population, and record whether they support or not solar power expansion.
    
    \item Find the sample proportion.
    
    \item Repeat this process many times and plot the distribution of the sample proportions obtained by this experimental design.
    
    \end{itemize}
    
A \hl{Sampling Distribution} is the distribution of a statistic (e.g., sample proportion), where the randomness comes from sampling randomly from a population.
\end{frame}
%%%%%%%%%%%%%%%%%%%%%%%%%%%%%%%%%%%%

\begin{frame}
    \frametitle{Edfinity Quiz}
    \begin{center}
        \includegraphics[width=0.75\textwidth]{../Lecture15/sampling_distribution.png}
    \end{center}
    
    \soln{
    \begin{table}[h!]
        \centering
        \begin{tabular}{|c|c|c|c|}
        \hline
         & Middle? & Symm? & Wider/Narrower/Same? \\ 
        \hline
        p=0.88, n=1000 & $\sim0.88$ & Yes & Same \\ 
        \hline
        p=0.5, n=100 & $\sim0.5$ & Yes & Wider \\ 
        \hline
        p=0.99, n=100 & $\sim0.99$ & No & Same? \\ 
        \hline
        \end{tabular}
    \end{table}
    }
\end{frame}

%%%%%%%%%%%%%%%%%%%%%%%%%%%%%%%%%%%%

\begin{frame}
    \frametitle{Edfinity Quiz}
    What would happen if the sample size were 10?
\end{frame}

%%%%%%%%%%%%%%%%%%%%%%%%%%%%%%%%%%%%

%%%%%%%%%%%%%%%%%%%%%%%%%%%%%%%%%%

\section{R Demo: What would happen if...}

%%%%%%%%%%%%%%%%%%%%%%%%%%%%%%%%

\section{Board work: Deriving theoretical sampling distribution of the sample proportion}

\begin{frame}
\frametitle{Summary: theoretical sampling distribution of the sample proportion}
\begin{itemize}
    \item For a sample of size $n$ and population proportion $p$, the number of successes $X \sim \text{Binomial}(n, p)$
    \item The sample proportion is a rescaled version of $X$: $\hat{p} = \frac{X}{n}$
    \pause
    \item So the sampling distribution of $\hat{p}$ is just like the binomial distribution, but for values from $0$ to $1$ instead of $0$ to $n$
    \pause
    \item $Pr(\hat{p} = \frac{k}{n}) = Pr(X = k) = \dbinom{n}{k} p^k (1-p)^{n-k}$ for $\frac{k}{n}$ in $\left\{0, \frac{1}{n}, \frac{2}{n}, \ldots, \frac{n-1}{n}, 1\right\}$
    \pause
    \item $E[\hat{p}] = p$
    \item $Var[\hat{p}] = \frac{p(1-p)}{n}$
\end{itemize}
The \hl{Standard Error} is the standard deviation of the sampling distribution. Here, $SE_{\hat{p}} = \sqrt{\frac{p(1-p)}{n}}$
\end{frame}
%%%%%%%%%%%%%%%%%%%%%%%%%%%%%%%%%%

\begin{frame}
    \frametitle{Rule of succession}

    There can be multiple statistics that estimate the same population parameter
    \begin{itemize}

        \item \hl{The Rule Of Succession} (Laplace, 18th Century)
        \item What is the probability that the sun will rise tomorrow, given that it has risen every day for the last 5000 years?
        \item 0 is an unsatisfying answer
        \item Approach: Pretend we have seen one success and one failure
    
    \end{itemize}

    Sample proportion:
     \begin{align}
        \hat{p}=\frac{k}{n}
    \end{align}
 
    Rule of Succession:
    \begin{align}
        \hat{p}^*=\frac{k +1}{n+2}
    \end{align}
 
\end{frame}


%%%%%%%%%%%%%%%%%%%%%%%%%%%%%%%%%%%%

\section{R Demo: Comparing different estimators based on their sampling distributions}

%%%%%%%%%%%%%%%%%%%%%%%%%%%%%%%%%%
\begin{frame}
    \frametitle{Comparing different estimators based on their sampling distributions}

    \begin{itemize}
        \item We can use simulations to compare different estimators for the same population parameter
        \item In general, we want an estimator that
        \begin{itemize}
            \item is unbiased, i.e. the mean of the sampling distribution is equal to the population parameter
            \item has low variability, i.e. the standard error is small
        \end{itemize}
        \item But sometimes it is helpful to have some bias for small samples, as long as the bias goes away for large samples (e.g. the rule of succession)
        \item And often, there is a trade-off between bias and variability
    \end{itemize}
\end{frame}

%%%%%%%%%%%%%%%%%%%%%%%%%%%%%%%%

\section{Sampling distribution for the sample mean of a Normal population}

%%%%%%%%%%%%%%%%%%%%%%%%%%%%%%%%

\section{R Demo: Motivate CLT}

%%%%%%%%%%%%%%%%%%%%%%%%%%%%%%%%%%

%%%%%%%%%%%%%%%%%%%%%%%%%%%%%%%%

\section{Board work: Binomial distribution converges to Normal}

%%%%%%%%%%%%%%%%%%%%%%%%%%%%%%%%%%

%%%%%%%%%%%%%%%%%%%%%%%%%%%%%%%%

\section{R Demo: Binomial to Normal}

%%%%%%%%%%%%%%%%%%%%%%%%%%%%%%%%%%

\section{Central Limit Theorem (Ch. 5.1-5.2)}

%%%%%%%%%%%%%%%%%%%%%%%%%%%%%%%%%%%%

\begin{frame}
\frametitle{Central Limit Theorem}

\formula{Central limit theorem}
{Sample proportions will be nearly normally distributed with mean equal to the population proportion, $p$, and standard error equal to $\sqrt{\frac{p~(1-p)}{n}}$.
\[ \hat{p} \sim N \pr{ mean = p, SE = \sqrt{\frac{p~(1-p)}{n}} } \]
}

\begin{itemize}

\item It wasn't a coincidence that the sampling distribution we saw earlier was symmetric, and centered at the true population proportion.

\item The standard error $SE =  \sqrt{\frac{p~(1-p)}{n}}$ is what we derived on the board for $\hat{p}$. Note that as $n$ increases $SE$ decreases.
\begin{itemize}
\item As $n$ increases samples will yield more consistent $\hat{p}$s, i.e. variability among $\hat{p}$s will be lower.
\end{itemize}

\end{itemize}

\end{frame}

%%%%%%%%%%%%%%%%%%%%%%%%%%%%%%%%%%%%

\begin{frame}
\frametitle{CLT - conditions}

Certain conditions must be met for the CLT to apply:

\begin{enumerate}

\item \hlGr{Independence:} Sampled observations must be independent. \\

This is difficult to verify, but is more likely if
\begin{itemize}
\item random sampling/assignment is used, and
\item if sampling without replacement, $n$ $<$ 10\% of the population.
\end{itemize}

\pause

\item \hlGr{Sample size:} There should be at least 10 expected successes and 10 expected failures in the observed sample.

This is difficult to verify if you don't know the population proportion (or can't assume a value for it). In those cases we look for the number of observed successes and failures to be at least 10.

\end{enumerate}

\end{frame}

%%%%%%%%%%%%%%%%%%%%%%%%%%%%%%%%%%

\begin{frame}
\frametitle{When the conditions are not met...}

\begin{itemize}

\item When either $np$ or $n(1-p)$ is small, the distribution is more discrete.
\item When $np$ or $n(1-p)$ $<$ 10, the distribution is more skewed.
\item The larger both $np$ and $n(1-p)$, the more normal the distribution.
\item When $np$ and $n(1-p)$ are both very large, the discreteness of the distribution is hardly evident, and the distribution looks much more like a normal distribution.

\end{itemize}

\end{frame}

%%%%%%%%%%%%%%%%%%%%%%%%%%%%%%%%

\section{Edfinity quiz}

%%%%%%%%%%%%%%%%%%%%%%%%%%%%%%%%%%%%

%%%%%%%%%%%%%%%%%%%%%%%%%%%%%%%%%%%%

\subsection{Extending the framework for other statistics}

%%%%%%%%%%%%%%%%%%%%%%%%%%%%%%%%%%%%

\begin{frame}
\frametitle{Extending the framework for other statistics}

\begin{itemize}

\item The strategy of using a sample statistic to estimate a parameter is quite common, and it's a strategy that we can apply to other statistics besides a proportion.

\begin{itemize}
\item Take a random sample of students at a college and ask them how many extracurricular activities they are involved in to estimate the average number of extra curricular activities all students in this college are interested in.
\end{itemize}

\item The principles and general ideas are from this chapter apply to other parameters as well, even if the details change a little. 

\end{itemize}

\end{frame}

%%%%%%%%%%%%%%%%%%%%%%%%%%%%%%%%%%%%
% End document
%%%%%%%%%%%%%%%%%%%%%%%%%%%%%%%%%%%%

\end{document}