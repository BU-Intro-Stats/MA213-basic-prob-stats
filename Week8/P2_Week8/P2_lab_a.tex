\documentclass[12pt]{article}
\usepackage{resource/shortex}
\usepackage{mathtools,amssymb}
\usepackage{enumitem}
\usepackage{datetime}
\usepackage{fancyhdr}
\usepackage{ifthen}
\usepackage{pifont}
\usepackage{mathrsfs}
\usepackage[normalem]{ulem}
%\RequirePackage[usenames,dvipsnames]{color}


%% showing solutions 
\newboolean{showsols}
\setboolean{showsols}{false}
\newcommand{\showsolutions}{\setboolean{showsols}{true}}
\newlength{\solspace}
\setlength{\solspace}{16em}
\newcommand{\nosolspace}{\setlength{\solspace}{0em}}

\newcommand{\solution}[2]{\ifthenelse{\boolean{showsols}}{{\par\textcolor{red}{Rubric criterion: #1}\newline\textcolor{blue}{#2}}}{\vspace{\solspace}}}
\newcommand{\npforprint}{\ifthenelse{\boolean{showsols}}{}{\newpage}}
\newcommand{\vsforprint}[1]{\ifthenelse{\boolean{showsols}}{}{\vspace{#1}}}

%% grading
\newcommand{\grades}[1]{~\newline\noindent\framebox[\textwidth][c]{Pass \quad \quad Almost Pass \quad \quad Not Yet}}

\newcommand{\questiongrades}{\noindent\emph{Each item is marked $\checkmark$, $\checkmark\!-$, or \ding{55}:}\begin{center}\emph{$\checkmark$ = capable  $~~$ $\checkmark\!-$ = mostly capable  $~~$ \ding{55} = not yet}\end{center} }

\newcommand{\triAssessments}{$\checkmark\quad\checkmark\!-\quad$\ding{55}\quad}
\newcommand{\biAssessments}{$\checkmark\quad$\ding{55}\quad}

%% line for writing name and BUID
\newcommand{\nameBUID}{\noindent\textbf{Name:}\underline{\phantom{XXXXXXXXXXXXXXXXXXX}}\hfill\textbf{BUID:}\underline{\phantom{XXXXXXXXXXXXX}}\newline}


%% formatting 
%\allowdisplaybreaks[3]
\renewcommand{\headrulewidth}{.4mm} % header line width
\renewcommand{\arraystretch}{1.25}


%% page style 
\pagestyle{fancy}
\fancyhf{}
\rhead{Fall 2025}
\lhead{CAS MA 213: Basic Statistics and Probability}
%\rfoot{\thepage}
\usepackage[hidelinks]{hyperref}
\setlength{\parskip}{0.5em}

%--------------------
\begin{document}

\begin{center}
\textbf{\Large Project 2 Guide \\ Week 8 } 
\end{center}

\section*{Lab Activities}

\subsection*{1. Group introductions}
Meet your group members and briefly introduce yourselves.  
Discuss your backgrounds, interests, and any relevant experience with data analysis or programming.

\vspace{4\baselineskip}

\subsection*{2. Discuss possible project topics}
Brainstorm and share ideas for your project topics. Consider these questions:  
\begin{itemize}
    \item What fields or subjects are you interested in (e.g., health, social science, sports)?  
    \item What kinds of data do you want to work with (categorical, numerical, or both)?  
    \item What types of hypothesis tests or comparisons might be relevant based on the data?  
\end{itemize}

\vspace{2\baselineskip}

\subsection*{3. Identify potential data sources}
Review suggested data sources provided in the project overview or propose your own.  
Check the availability, size, and variables in the datasets.

\vspace{1\baselineskip}

\noindent Some useful datasets to consider:  
\begin{itemize}
    \item \href{https://www.openintro.org/data/}{OpenIntro Datasets}  
    \item \href{https://cran.r-project.org/web/packages/fivethirtyeight/vignettes/fivethirtyeight.html}{fivethirtyeight R package}  
    \item \href{https://www.kaggle.com/datasets}{Kaggle datasets}  
\end{itemize}

\vspace{2\baselineskip}

\subsection*{4. Formulate preliminary research questions and hypotheses}
Start drafting clear, focused research questions that your data can help answer.  
Define your null and alternative hypotheses for these questions.

\vspace{3\baselineskip}

%%% Edit this section as needed %%%
% \subsection*{5. Assign group roles and responsibilities}

% Discuss the roles to organize your group work efficiently.  
% Common roles include but are not limited to:  
% \begin{itemize}
%     \item \textbf{Project Manager}: Oversees the project timeline, coordinates group meetings, and ensures deadlines are met.  
%     \item \textbf{Data Specialist}: Responsible for data collection, cleaning, and preprocessing.  
%     \item \textbf{Statistical Analyst}: Performs statistical tests, visualizations, and interprets results.  
%     \item \textbf{Report Writer}: Drafts and formats the written report, ensuring clarity and coherence.  
% \end{itemize}

% Feel free to assign multiple roles per person or adjust as needed based on your group’s strengths.

% Based on the project outline due next week, you can organize your group work as the following:
% \begin{itemize}
%     \item \textbf{Decide the project topic}  
%     \item \textbf{Find the data baed on the topic}
%     \item \textbf{Writing outline} 
%     \item \textbf{Review the outline and provide feedback}
% \end{itemize}

\subsection*{5. Assign Group Roles and Responsibilities}

Discuss the roles to organize your group work efficiently.  
Common roles include, but are not limited to:  
\begin{itemize}
    \item \textbf{Project Manager}: Oversees the project timeline, coordinates group meetings, and ensures deadlines are met.  
    \item \textbf{Data Specialist}: Responsible for data collection, cleaning, and preprocessing.  
    \item \textbf{Statistical Analyst}: Performs statistical tests, creates visualizations, and interprets results.  
    \item \textbf{Report Writer}: Drafts and formats the written report, ensuring clarity and coherence.  
\end{itemize}

Feel free to assign multiple roles to each person or adjust roles based on your group’s strengths.

Based on the project outline due next week, you can organize your group work as follows:
\begin{itemize}
    \item \textbf{Decide on the project topic}  
    \item \textbf{Find data based on the topic}
    \item \textbf{Write the project outline} 
    \item \textbf{Review the outline and provide feedback}
\end{itemize}


\vspace{2\baselineskip}

\subsection*{6. Plan your group's next steps}

Decide how to split preliminary work among group members, such as:  
\begin{itemize}
    \item Gathering and cleaning data  
    \item Exploring data summaries and visualizations  
    \item Reading background literature or similar studies  
    \item Drafting the project outline  
\end{itemize}

\vspace{2\baselineskip}

\section*{Post Lab Activities}

\begin{itemize}
    \item Prepare and submit your project outline by next Friday at 8:00 AM. The outline should include:  
    \begin{itemize}
        \item Project title and brief description  
        \item Research question(s) and hypotheses  
        \item Description of dataset(s) including sources and key variables  
        \item Preliminary analysis plan  
    \end{itemize}
    \item Schedule your group's next meetings and decide on communication methods.  
\end{itemize}



\end{document}