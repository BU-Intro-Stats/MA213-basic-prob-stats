%%%%%%%%%%%%%%%%%%%%%%%%%%%%%%%%%%%%

\section{Negative binomial distribution}

%%%%%%%%%%%%%%%%%%%%%%%%%%%%%%%%%%%%

\begin{frame}
\frametitle{Negative binomial distribution}

\begin{itemize}

\item The \hl{negative binomial distribution} describes the probability of observing the $k^{th}$ success on the $n^{th}$ trial.

\item The following four conditions are useful for identifying a negative binomial case:
\begin{enumerate}
\item The trials are independent.
\item Each trial outcome can be classified as a success or failure.
\item The probability of success ($p$) is the same for each trial.
\item The last trial must be a success.
\end{enumerate}
Note that the first three conditions are common to the binomial distribution.

\end{itemize}

\vfill

\formula{Negative binomial distribution}
{
P($k^{th}$ success on the $n^{th}$ trial) = ${n-1 \choose k-1}~p^k~(1-p)^{n-k}$, \\
where $p$ is the probability that an individual trial is a success. All trials are assumed to be independent.
}

\end{frame}

%%%%%%%%%%%%%%%%%%%%%%%%%%%%%%%%%%%%

\begin{frame}

\dq{A college student working at a psychology lab is asked to recruit 10 couples to participate in a study. She decides to stand outside the student center and ask every 5$^{th}$ person leaving the building whether they are in a relationship and, if so, whether they would like to participate in the study with their significant other. Suppose the probability of finding such a person is 10\%. What is the probability that she will need to ask 30 people before she hits her goal?}

\pause

Given: $p = 0.10$, $k = 10$, $n = 30$. We are asked to find the probability of $10^{th}$ success on the $30^{th}$ trial, therefore we use the negative binomial distribution.

\pause

\begin{eqnarray*}
P(\text{$10^{th}$ success on the $30^{th}$ trial}) &=& {29 \choose 9} \times 0.10^{10} \times 0.90^{20} \\
\pause
&=& 10,015,005 \times 0.10^{10} \times 0.90^{20} \\
\pause
&=& 0.00012
\end{eqnarray*}

\end{frame}

%%%%%%%%%%%%%%%%%%%%%%%%%%%%%%%%%%%%

\begin{frame}
\frametitle{Binomial vs. negative binomial}

\dq{How is the negative binomial distribution different from the binomial distribution?}

\pause

\begin{itemize}

\item In the binomial case, we typically have a fixed number of trials and instead consider the number of successes. 

\item In the negative binomial case, we examine how many trials it takes to observe a fixed number of successes and require that the last observation be a success.

\end{itemize}

\end{frame}

%%%%%%%%%%%%%%%%%%%%%%%%%%%%%%%%%%%%

\begin{frame}
\frametitle{Practice}

\pq{Which of the following describes a case where we would use the negative binomial distribution to calculate the desired probability?}

\begin{enumerate}[(a)]

\item Probability that a 5 year old boy is taller than 42 inches.

\item Probability that 3 out of 10 softball throws are successful.

\item Probability of being dealt a straight flush hand in poker.

\item Probability of missing 8 shots before the first hit.

\solnMult{Probability of hitting the ball for the 3${rd}$ time on the 8$^{th}$ try.}

\end{enumerate}

\end{frame}

%%%%%%%%%%%%%%%%%%%%%%%%%%%%%%%%%%%%
