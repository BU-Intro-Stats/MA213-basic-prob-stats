
%%%%%%%%%%%%%%%%%%%%%%%%%%%%%%%%%%%%
% Recap/Agenda 
%%%%%%%%%%%%%%%%%%%%%%%%%%%%%%%%%%%%
% TODO better formatting
\begin{frame}
    \frametitle{Module 3: Foundations for inference}
    \begin{itemize}
        \item \hl{Previously: }Point estimates and sampling variability (Chapter 5.1)
        \item \hl{This time: }Point estimates and sampling variability, continued (Chapter 5.1)
        \item \hl{Reading: }Chapter 5.2 for next time
        \item \hl{Deadlines/Announcements: }
        \begin{itemize}
            \item Reminder: Monday is a Holiday, Tuesday is Monday schedule
            \item Quiz 2 \hl{in class on Wednesday}, \textbf{Not in discussion sections}
            \item No discussion sections next week
        \end{itemize}
    \end{itemize}
    
\end{frame}
    
%%%%%%%%%%%%%%%%%%%%%%%%%%%%%%%%%%%%
% Learning objectives:
%%%%%%%%%%%%%%%%%%%%%%%%%%%%%%%%%%%%
\begin{frame}
\frametitle{Learning Objectives}
\begin{itemize}
    \item \textbf{M3 LO2: Visualize and Interpret Sampling Distributions:} Draw and interpret sampling distributions for a point estimate (e.g., population proportion) across different sample sizes, explaining how the distribution changes as the sample size increases. [Q3, L4]     \item \textbf{M6 LO1: Validate and Explain Probability Distributions:} Assess the validity of a probability distribution using the concepts of outcome, sample space, and probability properties (e.g., disjoint outcomes, probabilities between 0 and 1, and total probabilities summing to 1).
    \item \textbf{M3 LO3: Calculate and Interpret Standard Error:} Calculate the standard error for proportions and interpret it as a measure of sampling variability. [Q3, L4]
\end{itemize}
\end{frame}
