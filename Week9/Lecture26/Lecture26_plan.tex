%%%%%%%%%%%%%%%%%%%%%%%%%%%%%%%%%%%%
% Lesson Plan (50 minutes)
%%%%%%%%%%%%%%%%%%%%%%%%%%%%%%%%%%%%
\begin{frame}
    \frametitle{Lesson Plan}
    \begin{itemize}
        \item xx min Lecture: Review motivation from last time, introduce the new example for this lecture (i.e. diamonds)
        \item xx min Edfinity quiz: for this problem, what is the parameter of interest? what is the point estimate? how would you set up hypotheses?
        \item xx min Lecture: Review quiz answers
        \item xx min Lecture: Reviewing the conditions for the hypothesis test % LV: should this be before the quizzes?
        \item xx min R Demonstration: Computing the test statistic, p-value (like slides 7-10)
        \item xx min Edfinity Quiz: CI concept check (like slide 12); What can you conclude from this hypothesis test? Can you construct a CI and interpret it? (like slides 10-13)
        \item xx min Lecture: Review the worklow for constructing a CI, and the conclusions of the test (like synthesis, slide 11; slide 14) 
        \item xx min Think/pair/share: inference on means similarities/differences (one sample, paired, independent) 
        \item xx min Lecture: Recap concepts (like slide 15)
    \end{itemize}
\end{frame}

%%%%%%%%%%%%%%%%%%%%%%%%%%%%%%%%%%%%
% Learning objectives:
%%%%%%%%%%%%%%%%%%%%%%%%%%%%%%%%%%%%
\begin{frame}
    \frametitle{Learning Objectives}
    \begin{itemize}
        \item \textbf{M1, LO3: Use R for Data Management and Exploration:} Utilize R to load, pre-process, and explore data through visualization and summarization techniques.
        \item \textbf{M3, LO1: Understand Point Estimates and Sampling Variability:} Define a sample statistic (point estimate) for a population parameter, and explain how it varies across different samples.
        \item \textbf{M3, LO3: Calculate and Interpret Standard Error:} Calculate the standard error for proportions and interpret it as a measure of sampling variability.
        \item \textbf{M3, LO4: Explain Hypothesis Testing and Its Limitations:} Discuss the use cases and potential issues with hypothesis testing, including the interpretation of results.
        \item \textbf{M3, LO6: Distinguish Statistical vs. Practical Significance:} Differentiate between statistical significance and practical significance, and explain the implications of each.
        \item \textbf{M4, LO2: Design and Interpret Confidence Intervals:} Design, execute, and interpret confidence intervals for the population proportion.
        \item \textbf{M4, LO6: Conduct and Interpret t-Tests:} Design, execute, and interpret t-tests for a single population mean, a difference of paired means, and a difference of independent means, calculating the standard error appropriately for each. Describe how to obtain a p-value for a t-test and a critical t-score for a confidence interval.    
    \end{itemize}
\end{frame}
    
%%%%%%%%%%%%%%%%%%%%%%%%%%%%%%%%%%%%
% TODO: Fill in slides for motivating two sample problem
% TODO: Edfinity quiz on problem setup
% TODO: R demo for computing the test statistic and p-value
% TODO: Edfinity quiz on conclusions of HT, setting up a CI (possibly think/pair/share)
% TODO: Design think/pair/share and summary slide on the similarities/differences between the mean HT/CIs. E.g. a chart to fill in test statistic, standard error, degrees of freedom, assumptions/conditions