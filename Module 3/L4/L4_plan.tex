

%%%%%%%%%%%%%%%%%%%%%%%%%%%%%%%%%%%%
% Lesson Plan (50 minutes)
%%%%%%%%%%%%%%%%%%%%%%%%%%%%%%%%%%%%
\begin{frame}
\frametitle{Lesson Plan}
\begin{itemize}
    \item xx min Purpose of Simulation (maybe visualization)
    \item xx min 
    \item xx min Excercise using R 
\end{itemize}
\end{frame}


%%%%%%%%%%%%%%%%%%%%%%%%%%%%%%%%%%%%
% Learning objectives:
%%%%%%%%%%%%%%%%%%%%%%%%%%%%%%%%%%%%
\begin{frame}
\frametitle{Learning Objectives}

\begin{itemize}
    \item Use the normal distribution to assess the "unusualness" of data points, apply the 68-95-99.7\% rule, evaluate normality through histograms and q-q plots, and determine when a normal approximation to the binomial model is valid for calculating binomial probabilities. [Q2, L4] 
    \item Understand Point Estimates and Sampling Variability: Define a sample statistic (point estimate) for a population parameter, and explain how it varies across different samples. [Q3, L4] 
    \item Visualize and Interpret Sampling Distributions: Draw and interpret sampling distributions for a point estimate (e.g., population proportion) across different sample sizes, explaining how the distribution changes as the sample size increases. [Q3, L4] 
\end{itemize}
\end{frame}


\begin{frame}
    \frametitle{Learning Objectives Con't}

    \begin{itemize}
        \item Calculate and Interpret Standard Error: Calculate the standard error for proportions and interpret it as a measure of sampling variability. [Q3, L4]
        \item   

    \end{itemize}
\end{frame}






%%%%%%%%%%%%%%%%%%%%%%%%%%%%%%%%%%%%
% Excercise :
%%%%%%%%%%%%%%%%%%%%%%%%%%%%%%%%%%%%
\begin{frame}
    \frametitle{Excercise using R}
    \begin{enumerate}
        \item Show lists of dstn functions in R
        \item Give them populations with the same mean and variance, but very different shapes.
        \item Have them plot histograms of the sample mean
        \item Ask about the empirical mean and variance of the sample mean. (notice same for each distribution)
        \item Ask about the shape as the sample grows
        \item For superstars, let them derive the Normal and plot it on the histogram
    \end{enumerate}
\end{frame}
