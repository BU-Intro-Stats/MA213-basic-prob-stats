%%%%%%%%%%%%%%%%%%%%%%%%%%%%%%%%%%%%
% Lesson Plan (50 minutes)
%%%%%%%%%%%%%%%%%%%%%%%%%%%%%%%%%%%%
\begin{frame}
    \frametitle{Lesson Plan}
    \begin{itemize}
        \item xx min Lecture: Intro to Exponential distribution (?)
        \item xx min Lecture: Poisson distribution, link with Exponential (?)
        \begin{itemize}
            \item Probability mass function
            \item Formulas for expectation and variance (the same!)
            \item Variability
        \end{itemize}
        \item xx min R Demonstration: sampling from Poisson, different parameters
        \item xx min Edfinity quiz: when does modeling with a Poisson distribution make sense?
        \item xx min Lecture: review quiz answers
    \end{itemize}
\end{frame}
    
%%%%%%%%%%%%%%%%%%%%%%%%%%%%%%%%%%%%
% Learning objectives:
%%%%%%%%%%%%%%%%%%%%%%%%%%%%%%%%%%%%
\begin{frame}
    \frametitle{Learning Objectives}
    \begin{itemize}
        \item \textbf{M1 LO3: Use R for Data Management and Exploration:} Utilize R to load, pre-process, and explore data through visualization and summarization techniques.
        \item \textbf{M2 LO1: Validate and Explain Probability Distributions:} Assess the validity of a probability distribution using the concepts of outcome, sample space, and probability properties (e.g., disjoint outcomes, probabilities between 0 and 1, and total probabilities summing to 1).
        \item \textbf{M2 LO4: Understand and Compute Expectations and Variances:} Explain the concepts of expectations and variances of random variables, and compute the expectation and variance of a linear combination of random variables.
        %\item \textbf{M2 LO5: Model Data Using Bernoulli, Geometric, and Binomial Distributions:} Recognize when to appropriately model data using the Bernoulli, geometric, and binomial distributions, and compute quantities of interest such as mean, standard deviation, and tail probabilities.
        \item \textbf{M2 LO6: Assess Data Using the Normal Distribution:} Use the normal distribution to assess the "unusualness" of data points, apply the 68-95-99.7% rule, evaluate normality through histograms and q-q plots, and determine when a normal approximation to the binomial model is valid for calculating binomial probabilities.
    \end{itemize}
\end{frame}

%%%%%%%%%%%%%%%%%%%%%%%%%%%%%%%%%%%%
% TODO: Adapt drafted slides
