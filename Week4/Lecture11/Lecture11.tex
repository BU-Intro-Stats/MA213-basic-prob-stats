%%%%%%%%%%%%%%%%%%%%%%%%%%%%%%%%%%%%
% Slide options
%%%%%%%%%%%%%%%%%%%%%%%%%%%%%%%%%%%%

% Option 1: Slides with solutions

\documentclass[slidestop,compress,mathserif]{beamer}
\newcommand{\soln}[1]{\textit{#1}}
\newcommand{\solnGr}[1]{#1}

% Option 2: Handouts without solutions

%\documentclass[11pt,containsverbatim,handout]{beamer}
%\usepackage{pgfpages}
%\pgfpagesuselayout{4 on 1}[letterpaper,landscape,border shrink=5mm]
%\newcommand{\soln}[1]{ }
%\newcommand{\solnGr}{ }


%%%%%%%%%%%%%%%%%%%%%%%%%%%%%%%%%%%%
% Style
%%%%%%%%%%%%%%%%%%%%%%%%%%%%%%%%%%%%

\def\chpiv@path{../../Chp 4}
\input{../../lec_style.tex}


%%%%%%%%%%%%%%%%%%%%%%%%%%%%%%%%%%%%
% Preamble
%%%%%%%%%%%%%%%%%%%%%%%%%%%%%%%%%%%%

\title[Lecture 11]{MA213: Lecture 11}
\subtitle{Module 2: Probability, Random Variables, and Distributions}
\author{OpenIntro Statistics, 4th Edition}
\institute{$\:$ \\ {\footnotesize Based on slides developed by Mine \c{C}etinkaya-Rundel of OpenIntro. \\
The slides may be copied, edited, and/or shared via the \webLink{http://creativecommons.org/licenses/by-sa/3.0/us/}{CC BY-SA license.} \\
Some images may be included under fair use guidelines (educational purposes).}}
\date{}

%%%%%%%%%%%%%%%%%%%%%%%%%%%%%%%%%%%%
% Begin document
%%%%%%%%%%%%%%%%%%%%%%%%%%%%%%%%%%%%

\begin{document}


%%%%%%%%%%%%%%%%%%%%%%%%%%%%%%%%%%%%
% Title page
%%%%%%%%%%%%%%%%%%%%%%%%%%%%%%%%%%%%

{
\addtocounter{framenumber}{-1} 
{\removepagenumbers 
\usebackgroundtemplate{\includegraphics[width=\paperwidth]{../../OpenIntro_Grid_4_3-01.jpg}}
\begin{frame}

\hfill \includegraphics[width=20mm]{../../oiLogo_highres}

\titlepage

\end{frame}
}
}


%%%%%%%%%%%%%%%%%%%%%%%%%%%%%%%%%%%%
% Recap/Agenda 
%%%%%%%%%%%%%%%%%%%%%%%%%%%%%%%%%%%%
% TODO better formatting
\begin{frame}
    \frametitle{Module 2: Probability, Random Variables, and Distributions}
    \begin{itemize}
        \item \hl{Previously: } Normal distribution (Chapter 4.1)
        \item \hl{This time: } Normal distribution, continued
        \item \hl{Reading: } Chapter 4.2 for next time
        \item \hl{Deadlines/Announcements: } 
        \begin{itemize}
            \item HW 4 due Monday
            \item Quiz 1 grades are up on Gradescope
            \item Next week: Quiz 1 retake qualifications and Almost Pass conversions
        \end{itemize}
    \end{itemize}
    
\end{frame}

\begin{frame}
    \frametitle{Quiz 1}
    \begin{itemize}
        \item Reminder of the \hl{\href{https://docs.google.com/spreadsheets/d/1yAqE1HUeVbe6xuhddNSIX7Y3zuvFHlsuIn8Ys7V9_a0}{Grade Tracker}}
        \item There were 4 Core learning objectives and 1 Auxiliary learning objective on Quiz 1
        \item Each was graded Pass / Almost Pass / Not Yet
        \begin{itemize}
            \item Pass: Correct response. You do not need to take this Learning Objective again, even if you retake the quiz to cover other objectives.
            \item Almost Pass: Minor errors (e.g. typos). You can turn an Almost Pass into a Pass by \hl{attending the Office Hours} for Prof Stephen or your discussion TF within a week of receiving your grade on the quiz, explaining the error, and fixing it. Correct explanations will be converted to ``Pass"; incorrect explanations will be converted to ``Not Yet''.
            \item Not Yet: Incorrect, incomplete, or insufficient. To turn a Not Yet into a Pass, you need to \hl{retake the quiz}.
        \end{itemize}
    \end{itemize}
\end{frame}

\begin{frame}
\frametitle{Quiz Retakes}
    \begin{itemize}
        \item You can retake up to three quizzes
        \item Retakes will happen at the end of the semester (during Final Exam period + one TBD date)
        \item In order to retake a quiz, you need to \hl{qualify} within one week of receiving your grade on the quiz
        \item To qualify, you need to:
        \begin{itemize}
            \item \hl{Print} The \hl{\href{https://learn.bu.edu/ultra/courses/_246685_1/outline/file/_16630786_1}{Retake Qualification Form}} (Available on the course website in the Discussions folder, or on Gradescope)
            \item Fill it out \hl{completely} (including the blank copy of the quiz)
            \item Bring it to \hl{Office Hours} for Prof Stephen or your discussion TF to get it signed
            \item Submit the signed form to \hl{Gradescope} (under ``Quiz 1 Retake Qualification'') by the deadline
        \end{itemize}
    \end{itemize}
\end{frame}

\begin{frame}
    \frametitle{Office Hours Appointments}
    \begin{itemize}
        \item To manage demand, we are booking \hl{5 minute appointments} for Office Hours next week
        \begin{itemize}
            \item Prof Stephen (Mon 12:20-1:30pm, Thurs 9-10am)
            \item James Zheng Yang (Tues 12:30-2:30pm)
            \item Will Stride (Wed 3:00-5:00pm)
        \end{itemize}
        \item Appointments will be for both
        \begin{itemize}
            \item Converting Almost Pass to Pass
            \item Quiz retake qualifications
        \end{itemize}
        \item Bookings will be through Google Appointments -- \hl{links were sent out via email}
    \end{itemize}
\end{frame}

%%%%%%%%%%%%%%%%%%%%%%%%%%%%%%%%%%%%
% Learning objectives:
%%%%%%%%%%%%%%%%%%%%%%%%%%%%%%%%%%%%
\begin{frame}
    \frametitle{Learning Objectives}
    \begin{itemize}
        \item \textbf{M2 LO3: Compute Probabilities Using Various Tools:} Use logic, Venn diagrams, and probability rules to compute probabilities for events.
        \item \textbf{M2 LO4: Understand and Compute Expectations and Variances:} Explain the concepts of expectations and variances of random variables, and compute the expectation and variance of a linear combination of random variables.
        \item \textbf{M2 LO6: Assess Data Using the Normal Distribution:} Use the normal distribution to assess the "unusualness" of data points, apply the 68-95-99.7% rule, evaluate normality through histograms and q-q plots, and determine when a normal approximation to the binomial model is valid for calculating binomial probabilities.
    \end{itemize}
\end{frame}


%%%%%%%%%%%%%%%%%%%%%%%%%%%%%%%%%%%%
% Sections
%%%%%%%%%%%%%%%%%%%%%%%%%%%%%%%%%%%%

\section{Normal distribution}

\begin{frame}
\frametitle{Normal distribution}

\begin{itemize}

\item Unimodal and symmetric, bell shaped curve

\item Many variables are nearly normal, but none are exactly normal

\item Denoted as \mathhl{N(\mu,\sigma)} $\rightarrow$ Normal with mean $\mu$ and standard deviation $\sigma$

\end{itemize}

\begin{center}
\includegraphics[width=0.7\textwidth]{\chpiv@path/4-1_normal_distribution/figures/simpleNormal/simpleNormal}
\end{center}

\end{frame}

%%%%%%%%%%%%%%%%%%%%%%%%%%%%%%%%%%%%

\begin{frame}
\frametitle{Normal distributions with different parameters}

\vspace{-0.5cm}
\begin{center}
$\mu$: mean, $\sigma$: standard deviation
\[N(\mu = 0, \sigma = 1) \hspace{1.4cm} N(\mu = 19, \sigma = 4) \]
\includegraphics[width=0.6\textwidth]{\chpiv@path/4-1_normal_distribution/figures/twoSampleNormals/twoSampleNormals} \\
\includegraphics[width=0.6\textwidth]{\chpiv@path/4-1_normal_distribution/figures/twoSampleNormalsStacked/twoSampleNormalsStacked}
\end{center}

\end{frame}

\begin{frame}
\frametitle{Standardizing with Z scores}

\begin{itemize}

\item Z score of an observation is the number of standard deviations it falls above or below the mean.
\formula{\[Z = \frac{observation - mean}{SD}\]}
\pause

\item Z scores are defined for distributions of any shape, but only when the distribution is normal can we use Z scores to calculate percentiles.
\item If $X \sim N(\mu, \sigma)$, then $Z = \frac{X - \mu}{\sigma} \sim N(0, 1)$.

\pause
\item Observations that are more than 2 SD away from the mean ($|Z| > 2$) are usually considered unusual.

\end{itemize}

\end{frame}

%%%%%%%%%%%%%%%%%%%%%%%%%%%%%%%%%%%%

\begin{frame}
\frametitle{Percentiles}

\begin{itemize}

\item Recall from Chapter 2: the $p$th \hl{percentile} is a number such that $p\%$ of the observations fall below that number and $(100 - p)\%$ are equal to or above it

\item For continuous distributions, you can compute an observations's percentile by finding the area below the probability distribution curve to the left of that observation.

\end{itemize}

\begin{center}
\includegraphics[width=0.7\textwidth]{\chpiv@path/4-1_normal_distribution/figures/satBelow1800/satBelow1800}
\end{center}

\end{frame}

\begin{frame}
\frametitle{Cumulative distribution function (CDF)}
\begin{itemize}
\item The \hl{cumulative distribution function (CDF)} of a random variable $X$ is the function $F(x) = P(X \leq x)$.
\item The CDF gives the probability that the random variable takes on a value less than or equal to $x$.
\item For continuous distributions, the CDF is the area under the probability density function (PDF) curve to the left of $x$, $F(x) = \int_{-\infty}^{x} f(t) dt$.
\end{itemize}

\begin{center}
\begin{minipage}{0.48\textwidth}
\begin{center}
\textbf{Probability Density Function (PDF)}\\
\includegraphics[width=\textwidth]{\chpiv@path/4-1_normal_distribution/figures/satBelow1800/satBelow1800}
\end{center}
\end{minipage}
\hfill
\begin{minipage}{0.48\textwidth}
\begin{center}
\textbf{Cumulative Distribution Function (CDF)}\\
\includegraphics[width=\textwidth]{\chpiv@path/4-1_normal_distribution/figures/satBelow1800/satBelow1800_CDF}
\end{center}
\end{minipage}
\end{center}
\end{frame}

\begin{frame}
\frametitle{Quality control}

\dq{{\small At Heinz ketchup factory the amounts which go into bottles of ketchup are supposed to be normally distributed with mean 36 oz. and standard deviation 0.11 oz. Once every 30 minutes a bottle is selected from the production line, and its contents are noted precisely. If the amount of ketchup in the bottle is below 35.8 oz. or above 36.2 oz., then the bottle fails the quality control inspection. What percent of bottles have less than 35.8 ounces of ketchup?}}

\soln{\pause
Let $X$ = amount of ketchup in a bottle: $X \sim N(\mu = 36, \sigma = 0.11)$ \\
\pause
\twocol{0.4}{0.6}{
\begin{center}
\includegraphics[width=\textwidth]{\chpiv@path/4-1_normal_distribution/figures/ketchup/ketchupLT358}
\end{center}
}
{
\pause
\vspace{2em}
\[ Z = \frac{35.8 - 36}{0.11} = -1.82 \]
\pause
\vspace{-1em}
\begin{eqnarray*}
    pnorm(-1.82)&=&0.03440\\
    pnorm(35.8,36,0.11)&=&0.03440
\end{eqnarray*}
}
}

\end{frame}

%%%%%%%%%%%%%%%%%%%%%%%%%%%%%%%%%%%%

\begin{frame}
\frametitle{Practice}

\pq{What percent of bottles \underline{pass} the quality control inspection?}

\vspace{-0.5cm}
\begin{multicols}{2}
\begin{enumerate}[(a)]
\item 1.82\%
\item 3.44\%
\item 6.88\%
\solnMult{93.12\%}
\item 96.56\%
\item[]
\end{enumerate}
\end{multicols}

\soln{
\vspace{-0.5cm}
\pause
\begin{columns}[c]
\column{0.3\textwidth}
\pause
\includegraphics[width=\textwidth]{\chpiv@path/4-1_normal_distribution/figures/ketchup/ketchupBET}
\column{0.05\textwidth}
=
\pause
\column{0.3\textwidth}
\includegraphics[width=\textwidth]{\chpiv@path/4-1_normal_distribution/figures/ketchup/ketchupLT362}
\column{0.05\textwidth}
-
\pause
\column{0.3\textwidth}
\includegraphics[width=\textwidth]{\chpiv@path/4-1_normal_distribution/figures/ketchup/ketchupLT358}
\end{columns}
\pause
% \begin{eqnarray*}
% Z_{35.8} = \frac{35.8 - 36}{0.11} = -1.82 \hspace{1.5cm}
% Z_{36.2} = \frac{36.2 - 36}{0.11} = 1.82 \\ \pause
% \end{eqnarray*}
\begin{small}
\begin{eqnarray*}
P(35.8 < X < 36.2) &=& pnorm(36.2, 35.8, 0.11) - pnorm(35.8, 36, 0.11) \\
&=& 0.9656 - 0.0344 = 0.9312 \\ \pause
P(-1.82 < Z < 1.82) &=& pnorm(-1.82) - pnorm(1.82) \\
&=& 0.9656 - 0.0344 = 0.9312
\end{eqnarray*}
\end{small}
}

\end{frame}

%%%%%%%%%%%%%%%%%%%%%%%%%%%%%%%%%%%%

\section{Edfinity Quiz}

%%%%%%%%%%%%%%%%%%%%%%%%%%%%%%%%%%%%

\begin{frame}[fragile]
\frametitle{Finding cutoff points}

\dq{Body temperatures of healthy humans are distributed nearly normally with mean 98.2$\degree$F and standard deviation 0.73$\degree$F. What is the cutoff for the lowest 3\% of human body temperatures?}

\pause

\twocol{0.3}{0.7}
{
\includegraphics[width=\textwidth]{\chpiv@path/4-1_normal_distribution/figures/temp/tempLOW3PERC}
}
{
\pause
\begin{eqnarray*}
P(X < x) &=& 0.03 \rightarrow P(Z < \orange{-1.88}) = 0.03 \\ \pause
Z &=& \frac{obs~-~mean}{SD} \rightarrow \frac{x - 98.2}{0.73} = -1.88 \\ \pause
x &=& (-1.88 \times 0.73) + 98.2 = 96.8\degree F
\end{eqnarray*}
}
$\:$ \\
\begin{beamerboxesrounded}[shadow = true, lower = code body]{}
{\small \begin{verbatim}
> qnorm(0.03)
[1] -1.880794
\end{verbatim}
}
\end{beamerboxesrounded}

\ct{Mackowiak, Wasserman, and Levine (1992), \textit{A Critical Appraisal of 98.6 Degrees F, the Upper Limit of the Normal Body Temperature, and Other Legacies of Carl Reinhold August Wunderlick}.}

\end{frame}

%%%%%%%%%%%%%%%%%%%%%%%%%%%%%%%%%%%

\begin{frame}[fragile]
\frametitle{Practice}

\pq{Body temperatures of healthy humans are distributed nearly normally with mean 98.2$\degree$F and standard deviation 0.73$\degree$F. What is the cutoff for the highest 10\% of human body temperatures?}

\vspace{-0.5cm}
\begin{multicols}{2}
\begin{enumerate}[(a)]
\item 97.3$\degree$F
\solnMult{99.1$\degree$F}
\item 99.4$\degree$F
\item 99.6$\degree$F
\end{enumerate}
\end{multicols}

\soln{
\vspace{-0.5cm}
\pause
\vspace{-0.3cm}
\begin{center}
\includegraphics[width=0.3\textwidth]{\chpiv@path/4-1_normal_distribution/figures/temp/tempHIGH10PERC}
\end{center}
\vspace{-0.3cm}
\pause
\begin{small}
\begin{eqnarray*}
    P(X > x) = 0.10 &\rightarrow& P(X < x) = 0.90 \\
    &\rightarrow& x = qnorm(0.90, 98.2, 0.73) = 99.1\\ \pause
    P(Z<z) = 0.90 &\rightarrow& z = qnorm(0.90) = 1.28 \\
    &\rightarrow& x= (1.28 \times 0.73) + 98.2 = 99.1
\end{eqnarray*}
\end{small}
}

\end{frame}

%%%%%%%%%%%%%%%%%%%%%%%%%%%%%%%%%%%%

\section{R Demonstration: qnorm} 

%%%%%%%%%%%%%%%%%%%%%%%%%%%%%%%%%%%%

\section{Edfinity quiz: cutoff points}

% %%%%%%%%%%%%%%%%%%%%%%%%%%%%%%%%%%%

% \subsection{68-95-99.7 rule}

% %%%%%%%%%%%%%%%%%%%%%%%%%%%%%%%%%%%%

% \begin{frame}
% \frametitle{68-95-99.7 Rule}

% \begin{itemize}

% \item For nearly normally distributed data, 
% \begin{itemize}
% \item about 68\% falls within 1 SD of the mean,
% \item about 95\% falls within 2 SD of the mean,
% \item about 99.7\% falls within 3 SD of the mean.
% \end{itemize}

% \item It is possible for observations to fall 4, 5, or more standard deviations away from the mean, but these occurrences are very rare if the data are nearly normal.

% \end{itemize}

% \begin{center}
% \includegraphics[width=0.7\textwidth]{\chpiv@path/4-1_normal_distribution/figures/6895997/6895997}
% \end{center}

% \end{frame}

% %%%%%%%%%%%%%%%%%%%%%%%%%%%%%%%%%%%%

% \begin{frame}
% \frametitle{Describing variability using the 68-95-99.7 Rule}

% SAT scores are distributed nearly normally with mean 1500 and standard deviation 300.

% \pause
% \begin{itemize}

% \item $\sim$68\% of students score between 1200 and 1800 on the SAT. 

% \item $\sim$95\% of students score between 900 and 2100 on the SAT. 

% \item $\sim$99.7\% of students score between 600 and 2400 on the SAT. 

% \end{itemize}

% \begin{center}
% \includegraphics[width=0.65\textwidth]{\chpiv@path/4-1_normal_distribution/figures/sat_empirical/sat_empirical}
% \end{center}

% \end{frame}

% %%%%%%%%%%%%%%%%%%%%%%%%%%%%%%%%%%%%

% \section{Edfinity quiz: Geometric logic}  

% %%%%%%%%%%%%%%%%%%%%%%%%%%%%%%%%%%%%

% \begin{frame}[fragile]
% \frametitle{Number of hours of sleep on school nights}

% \only<1 | handout:0>{
% \begin{center}
% \includegraphics[width=0.75\textwidth]{\chpiv@path/4-1_normal_distribution/figures/sleep/sleep-hist} 
% \end{center}
% \vspace{-0.25cm}
% \begin{itemize}
% \item Mean = 6.88 hours, SD = 0.92 hrs
% \item[] \textcolor{white}{72\% of the data are within 1 SD of the mean: $6.88 \pm 0.93$}
% \item[] \textcolor{white}{92\% of the data are within 2 SD of the mean: $6.88 \pm 2 \times 0.93$}
% \item[] \textcolor{white}{99\% of the data are within 3 SD of the mean: $6.88 \pm 3 \times 0.93$}
% \end{itemize}
% }

% \only<2 | handout:0>{
% \begin{center}
% \includegraphics[width=0.75\textwidth]{\chpiv@path/4-1_normal_distribution/figures/sleep/sleep-hist-sd1} 
% \end{center}
% \vspace{-0.25cm}
% \begin{itemize}
% \item Mean = 6.88 hours, SD = 0.92 hrs
% \item 72\% of the data are within 1 SD of the mean: $6.88 \pm 0.93$
% \item[] \textcolor{white}{92\% of the data are within 2 SD of the mean: $6.88 \pm 2 \times 0.93$}
% \item[] \textcolor{white}{99\% of the data are within 3 SD of the mean: $6.88 \pm 3 \times 0.93$}
% \end{itemize}
% }

% \only<3 | handout:0>{
% \begin{center}
% \includegraphics[width=0.75\textwidth]{\chpiv@path/4-1_normal_distribution/figures/sleep/sleep-hist-sd2} 
% \end{center}
% \vspace{-0.25cm}
% \begin{itemize}
% \item Mean = 6.88 hours, SD = 0.92 hrs
% \item 72\% of the data are within 1 SD of the mean: $6.88 \pm 0.93$
% \item 92\% of the data are within 1 SD of the mean: $6.88 \pm 2 \times 0.93$
% \item[] \textcolor{white}{99\% of the data are within 3 SD of the mean: $6.88 \pm 3 \times 0.93$}
% \end{itemize}
% }

% \only<4>{
% \begin{center}
% \includegraphics[width=0.75\textwidth]{\chpiv@path/4-1_normal_distribution/figures/sleep/sleep-hist-sd3} 
% \end{center}
% \vspace{-0.25cm}
% \begin{itemize}
% \item Mean = 6.88 hours, SD = 0.92 hrs
% \item 72\% of the data are within 1 SD of the mean: $6.88 \pm 0.93$
% \item 92\% of the data are within 2 SD of the mean: $6.88 \pm 2 \times 0.93$
% \item 99\% of the data are within 3 SD of the mean: $6.88 \pm 3 \times 0.93$
% \end{itemize}
% }

% \end{frame}

%%%%%%%%%%%%%%%%%%%%%%%%%%%%%%%%%%%%

% \begin{frame}
% \frametitle{Practice}

% \pq{Which of the following is \underline{false}?}

% \begin{enumerate}[(a)]
% \item Majority of Z scores in a right skewed distribution are negative.
% \solnMult{In skewed distributions the Z score of the mean might be different than 0.}
% \item For a normal distribution, IQR is less than $2 \times SD$.
% \item Z scores are helpful for determining how unusual a data point is compared to the rest of the data in the distribution.
% \end{enumerate}

% \end{frame}

%%%%%%%%%%%%%%%%%%%%%%%%%%%%%%%%%%%%

% \section{R Demonstration: Variability with 68-95-99.7 rule}

%%%%%%%%%%%%%%%%%%%%%%%%%%%%%%%%%%%%

% \section{Edfinity quiz: Variability, concept review}
% Thinking like the last Practice slide, commented out above  
% ES: Make it into a think/pair/share?

%%%%%%%%%%%%%%%%%%%%%%%%%%%%%%%%%%%%
% End document
%%%%%%%%%%%%%%%%%%%%%%%%%%%%%%%%%%%%

\end{document}