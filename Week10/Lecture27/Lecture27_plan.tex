%%%%%%%%%%%%%%%%%%%%%%%%%%%%%%%%%%%%
% Lesson Plan (50 minutes)
%%%%%%%%%%%%%%%%%%%%%%%%%%%%%%%%%%%%
\begin{frame}
    \frametitle{Lesson Plan}
    \begin{itemize}
        \item xx min Lecture: last week we did a lot of tests and inference - but how strong are our conclusions? is there a way to measure that?
        \item xx min Lecture: first, review decision errors (like slides 2-3)
        \item xx min Edfinity quiz (short): decision errors, error rates practice (review)
        \item xx min Lecture: set up a motivating example, calculate SE and minimum effect size (like slides 4-6)
        \item xx min Lecture: finally, define power
        \item xx min Edfinity quiz: power calculation and (think/pair/share?) intuition (e.g. how would the power change if we increased the sample size / effect size / alpha / standard deviation?)
        % suggestion from ES for quiz above: "what do you think might increase the power of this test? (open answer)
        \item xx min Board work/lecture: Quiz solutions
        \item xx min Lecture: computing sample size
        \item xx min Edfinity quiz: what sample size will lead to a power of XX\% for this test? (like slide 50, maybe 2 such examples) 
        \item xx min Lecture: review quiz answer(s), recap procedure to computing power and how to achieve desired power (like slide 8)
    \end{itemize}
\end{frame}

%%%%%%%%%%%%%%%%%%%%%%%%%%%%%%%%%%%%
% Learning objectives:
%%%%%%%%%%%%%%%%%%%%%%%%%%%%%%%%%%%%
\begin{frame}
    \frametitle{Learning Objectives}
    \begin{itemize}
        % Commenting out R LO because there are none currently planned
        %\item \textbf{M1, LO3: Use R for Data Management and Exploration:} Utilize R to load, pre-process, and explore data through visualization and summarization techniques.
        \item \textbf{M3, LO1: Understand Point Estimates and Sampling Variability:} Define a sample statistic (point estimate) for a population parameter, and explain how it varies across different samples.
        \item \textbf{M3, LO3: Calculate and Interpret Standard Error:} Calculate the standard error for proportions and interpret it as a measure of sampling variability.
        \item \textbf{M3, LO4: Explain Hypothesis Testing and Its Limitations:} Discuss the use cases and potential issues with hypothesis testing, including the interpretation of results.
        \item \textbf{M3, LO6: Distinguish Statistical vs. Practical Significance:} Differentiate between statistical significance and practical significance, and explain the implications of each.
        \item \textbf{M4, LO2: Design and Interpret Confidence Intervals:} Design, execute, and interpret confidence intervals for the population proportion.
        \item \textbf{M4, LO6: Conduct and Interpret t-Tests:} Design, execute, and interpret t-tests for a single population mean, a difference of paired means, and a difference of independent means, calculating the standard error appropriately for each. Describe how to obtain a p-value for a t-test and a critical t-score for a confidence interval.    
        \item \textbf{M4, LO7: Calculate Test Power and Evaluate Factors:} Calculate the power of a test for a given effect size and significance level, and explain how the power would change with variations in effect size, sample size, significance level, or standard error.
    \end{itemize}
\end{frame}
    
%%%%%%%%%%%%%%%%%%%%%%%%%%%%%%%%%%%%
% TODO: Edfinity Quiz on decision errors
% TODO: Add a slide on how to choose the effect size (prior similar studies, pilot study, minimum relevant effect)
% TODO: Edfinity Quiz on power calculation and (think/pair/share?) intuition (e.g. how would the power change if we increased the sample size / effect size / alpha / standard deviation?)
% TODO: Prepare board work on what would happen if...
% TODO: Edfinity Quiz on sample size calculations (break it up into steps)
% TODO: Make slides for what if... scenarios (based on MA213)
